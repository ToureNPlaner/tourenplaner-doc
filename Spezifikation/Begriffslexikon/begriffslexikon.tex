%begriffslexikon
\documentclass[a4paper,10pt,titlepage]{article}

\usepackage[utf8]{inputenc}
\usepackage[ngerman]{babel}
\usepackage{a4wide}
\usepackage{booktabs}
\usepackage{float}
\usepackage{paralist}
\usepackage{array}
\usepackage{ifthen}
\usepackage[pdfborder={0 0 0}]{hyperref}

\author{ToureNPlaner Team}
\title{ToureNPlaner Begriffslexikon\\ Stupro 2011/2012 \\ Version 1.0}

\makeatletter
\newcommand\novspace{\@minipagetrue}

\newenvironment{owncompactitem}{%
\compactitem
}{%
\@finalstrut\@arstrutbox
\@nameuse{endcompactitem}%
\aftergroup\let\aftergroup\@finalstrut\aftergroup\@gobble
}
\newenvironment{owncompactenum}{%
\compactenum
}{%
\@finalstrut\@arstrutbox
\@nameuse{endcompactenum}%
\aftergroup\let\aftergroup\@finalstrut\aftergroup\@gobble
}
\makeatother

\newcommand{\begriff}[7]
{\paragraph{#1}$~$ \\
\begin{tabular}{|p{0.2\textwidth}|p{0.7\textwidth}|}
\hline
  Bedeutung & #2\\\hline
  Abgrenzung & #3\\\hline
  Gültigkeit & #4\\\hline
  Bezeichnung & #5\\\hline
  Unklarheiten & #6\\\hline
  Querverweise & #7\\\hline
 \end{tabular}
}

\newcommand{\verweis}[2]
{\paragraph{#1}$\rightarrow$ #2
}

\begin{document}

\pagenumbering{alph}
\maketitle
\pagenumbering{roman}
\setcounter{page}{1}
%\tableofcontents
\clearpage
\pagenumbering{arabic}
\setcounter{page}{1}

%\section{Begriffslexikon}

%\begriff{Begriff}{Bedeutung}{Abgrenzung}{Gültigkeit}{Bezeichnung}{Unklarheiten}{Querverweise}

%\begriff
{%Begriff
}{%Bedeutung
}{%Abgrenzung
}{%Gültigkeit
}{%Bezeichnung
}{%Unklarheiten
}{%Querverweise
}

\verweis
{Abrechnungs-DB}
{Abrechnungs-Datenbank}

\begriff
{%Begriff
Abrechnungs-Datenbank
}{%Bedeutung
Die Abrechnungs-Datenbank speichert alle empfangenen Algorithmus-Requests (Anfragen) von Clients und deren Berechnungsergebnisse des Servers, die für die Abrechnung benötigt werden. Außerdem kann der Status einer Berechnung hier gespeichert werden.
}{%Abgrenzung
Die Abrechnungs-Datenbank ist keine eigenständige Datenbank, sondern ist eine Teilmenge an Tabellen einer Datenbank mit weiteren Tabellen.
}{%Gültigkeit
-
}{%Bezeichnung
Die Abrechnungs-Datenbank und die Benutzer-DB befinden sich im selben Datenbanksystem, beides sind aber verschiedene Datenbank-Tabellen und für die Use-Cases sind es konzeptionell verschiedene Akteure. 
}{%Unklarheiten
-
}{%Querverweise
Benutzer-DB
}

\begriff
{%Begriff
Backend-Daten
}{%Bedeutung
Alle Daten, die der Server für seinen Betrieb in Datenbanken oder auf andere Weise speichert, werden als Backend-Daten bezeichnet. Zu den Backend-Daten gehört auch die Graphrepräsentation.
}{%Abgrenzung
Daten auf Client-Systemen, die der Server nicht kennt, gehören nicht zu den Backend-Daten.
}{%Gültigkeit
-
}{%Bezeichnung
-
}{%Unklarheiten
-
}{%Querverweise
Graphrepräsentation
}

\begriff
{%Begriff
Benutzer-DB
}{%Bedeutung
Die Benutzer-DB (Benutzer-Datenbank) speichert alle nötigen Informationen über die Benutzer des Systems, dabei werden sowohl Admins wie auch Kunden in der Benutzer-DB gespeichert.
}{%Abgrenzung
Die Benutzer-DB ist keine eigenständige Datenbank, sondern ist eine Teilmenge an Tabellen einer Datenbank mit weiteren Tabellen.
}{%Gültigkeit
-
}{%Bezeichnung
Die Abrechnungs-Datenbank und die Benutzer-DB befinden sich im selben Datenbanksystem, beides sind aber verschiedene Datenbank-Tabellen und für die Use-Cases sind es konzeptionell verschiedene Akteure.
}{%Unklarheiten
-
}{%Querverweise
Abrechnungs-Datenbank
}

\begriff
{%Begriff
Client
}{%Bedeutung
Eine Anwendung, die auf die Funktionen des ToureNPlaner Servers mittels des ToureNPlaner Protokolls (siehe Protokollspezifikation) zugreift.
}{%Abgrenzung
Innerhalb dieser Spezifikation bezeichnet Client ausschließlich Clients für den ToureNPlaner Server.
}{%Gültigkeit
-
}{%Bezeichnung
-
}{%Unklarheiten
-
}{%Querverweise
Webbasierter Client
}

\begriff
{%Begriff
Graphbasierte Webanwendung
}{%Bedeutung
Eine Webanwendung, die Dienste für bestimmte Graphprobleme anbietet. Ein bekanntes Beispiel ist die Berechnung des kürzesten Weges zwischen zwei Punkten auf einer Karte, auch als Routenplanung bekannt.
}{%Abgrenzung
Webanwendungen, die beliebige Daten speichern, die man als Graph interpretieren kann, sind nicht zwangsläufig als graphbasierte Webanwendungen zu bezeichnen. Die Daten und deren Interpretation als Graph müssen Kernaspekte der angebotenen Dienste sein, damit die Webanwendung als graphbasiert bezeichnet werden kann.
}{%Gültigkeit
-
}{%Bezeichnung
-
}{%Unklarheiten
Es ist unklar, welche Arten von Graphen (z.B. nur Kartengraphen) gemeint sind.
}{%Querverweise
Webanwendung
}

\begriff
{%Begriff
Graphrepräsentation
}{%Bedeutung
Einen Graphen, wie zum Beispiel die Straßennetzkarte von Deutschland, kann man auf verschiedene Weise speichern, es gibt also verschiedene Möglichkeiten der Graphrepräsentation. Eine Graphrepräsentation hat eine bestimmte Art, wie Kanten und Knoten sowie zugehörige Daten eines Graphen gespeichert werden. Auf der Graphrepräsentation werden die von Kunden gewünschten Algorithmen ausgeführt.
}{%Abgrenzung
Die grafische Darstellung eines Graphen hat nichts mit der Graphrepräsentation im Sinne der Server-Spezifikation zu tun.
}{%Gültigkeit
-
}{%Bezeichnung
-
}{%Unklarheiten
-
}{%Querverweise
-
}

\verweis
{Request-Log}
{Abrechnungs-Datenbank}

\verweis
{%Begriff
Rundreise
}{%Verweis
Tour
}

\verweis
{%Begriff
Rundtour
}{%Verweis
Tour
}

\begriff
{%Begriff
Software
}{%Bedeutung
Die Software ToureNPlaner ist die Bezeichnung für die Gesamtheit aller im Rahmen des Projektes entwickelten und verwendeten Systeme, auch die zu entwickelnde Software genannt. Dazu gehören der Server und die Clients sowie ihre verwendeten Komponenten.
}{%Abgrenzung
Es ist nicht das Projekt ToureNPlaner gemeint.
}{%Gültigkeit
Die Software ToureNPlaner hat mit dem Beginn des Studienprojektes angefangen zu existieren, unabhängig von der Funktionalität der Software. Im Rahmen des Studienprojektes entstehende Dokumente und Systeme sind Teil der Software, auch wenn diese letztendlich nicht mit ausgeliefert oder gelöscht werden. Die Software ToureNPlaner wird nach dem Ende des Studienprojektes ToureNPlaner noch weiter bestehen.
}{%Bezeichnung
-
}{%Unklarheiten
-
}{%Querverweise
Studienprojekt ToureNPlaner, ToureNPlaner
}

\begriff
{%Begriff
Studienprojekt
}{%Bedeutung
Das Studienprojekt ToureNPlaner ist ein Studienprojekt im Rahmen des Studienganges Softwaretechnik an der Universität Stuttgart. Im Rahmen von diesem Projekt muss die Software ToureNPlaner entwickelt werden.
}{%Abgrenzung
Es geht ausschließlich um das Studienprojekt ToureNPlaner des Instituts für Formale Methoden der Informatik an der Universität Stuttgart.
}{%Gültigkeit
Das Studienprojekt ToureNPlaner hat am Anfang des Sommersemesters 2011 begonnen und wird voraussichtlich am Anfang des Sommersemesters 2012 beendet sein.
}{%Bezeichnung
-
}{%Unklarheiten
-
}{%Querverweise
-
}

\verweis
{%Begriff
Stupro
}{%Verweis
Studienprojekt
}

\begriff
{%Begriff
textbasiertes Graphformat
}{%Bedeutung
Rerpäsentation eines allgemeinen Graphen mit Knoten, Kanten und Zusatzinformationen (wie z.B. Kantengewichten) als Textdatei (siehe auch Graph-Format-Dokument).
}{%Abgrenzung
Hier ist das von ToureNPlaner verwendete Format und nicht etwa eine XML-Repräsentation gemeint.
}{%Gültigkeit
-
}{%Bezeichnung
-
}{%Unklarheiten
Sind durch das Graph-Format-Dokument zu klären.
}{%Querverweise
-
}

\begriff
{%Begriff
Tour
}{%Bedeutung
Eine Tour beschreibt einen mit bestimmten Fortbewegungsmitteln oder auch zu Fuß begehbaren Weg, wobei dieser Weg einen festgelegten oder bliebigen Startpunkt hat. Die Begehung dieses Weges führt am Ende wieder zum Startpunkt, somit entspricht eine Tour einer Route, bei der End- und Startpunkt identisch sind.
}{%Abgrenzung
Eine Route, bei Start- und Endpunkt unterschiedlich sind, ist keine Tour.
}{%Gültigkeit
-
}{%Bezeichnung
Eine Tour ist durch eine geordnete Liste von Koordinaten definiert.
}{%Unklarheiten
-
}{%Querverweise
-
}

\begriff
{%Begriff
ToureNPlaner
}{%Bedeutung
ToureNPlaner ist die Bezeichnung für das Studienprojekt ToureNPlaner und für die Software ToureNPlaner.
}{%Abgrenzung
Es ist kein allgemeiner Tourenplaner gemeint.
}{%Gültigkeit
-
}{%Bezeichnung
Es muss aus dem Kontext erschlossen werden, ob mit dem Begriff ToureNPlaner das Studienprojekt oder die Software ToureNPlaner gemeint ist. Wenn im Kontext von der Funktionalität vom ToureNPlaner die Rede ist, ist damit zum Beispiel die Software ToureNPlaner gemeint.
}{%Unklarheiten
-
}{%Querverweise
Studienprojekt, Software
}

\begriff
{%Begriff
Webanwendung
}{%Bedeutung
Eine Anwendung, die mit einem Internetbrowser zugänglich ist und mit einem Server kommuniziert.
}{%Abgrenzung
Sonstige Anwendungen, die mittels Internet-Protokollen (z.B. HTTP) mit einem Server kommunizieren, wie zum Beispiel Apps, sind keine Webanwendungen.
}{%Gültigkeit
-
}{%Bezeichnung
-
}{%Unklarheiten
Sind Flash/Java-Applets etc. Webanwendungen?
}{%Querverweise
App
}

\begriff
{%Begriff
Webbasierter Client
}{%Bedeutung
Ein Client im Sinne des ToureNPlaner Projekts, welcher als Webanwendung realisiert ist.
}{%Abgrenzung
Ein Client, der nicht im Browser läuft, ist kein Webclient.
}{%Gültigkeit
-
}{%Bezeichnung
-
}{%Unklarheiten
-
}{%Querverweise
Webanwendung, Client
}



%\begriff
{%Begriff

}{%Bedeutung
%-
}{%Abgrenzung
%-
}{%Gültigkeit
%-
}{%Bezeichnung
%-
}{%Unklarheiten
%-
}{%Querverweise
%-
}

\end{document}
