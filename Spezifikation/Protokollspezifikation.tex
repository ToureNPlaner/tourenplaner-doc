\documentclass[ngerman,titlepage,parskip=true]{scrartcl}
\usepackage[utf8]{inputenc}
\usepackage{array}
\usepackage{babel}
\usepackage{wrapfig}
\usepackage{longtable}
\usepackage[unicode=true,pdfusetitle,bookmarks=true,bookmarksnumbered=false,bookmarksopen=false,breaklinks=false,pdfborder={0 0 0},backref=false,colorlinks=false]{hyperref}
\usepackage{listings}
\usepackage{color}
\usepackage{geometry}
\geometry{a4paper,left=20mm,right=20mm, top=2cm, bottom=4cm} 

\lstset{ %
	language=C,                % the language of the code
	basicstyle=\footnotesize,       % the size of the fonts that are used for the code
	numbers=left,                   % where to put the line-numbers
	numberstyle=\footnotesize,      % the size of the fonts that are used for the line-numbers
	stepnumber=2,                   % the step between two line-numbers. If it's 1, each line 
	                                % will be numbered
	numbersep=5pt,                  % how far the line-numbers are from the code
	backgroundcolor=\color{white},  % choose the background color. You must add \usepackage{color}
	showspaces=false,               % show spaces adding particular underscores
	showstringspaces=false,         % underline spaces within strings
	showtabs=false,                 % show tabs within strings adding particular underscores
	frame=single,                   % adds a frame around the code
	tabsize=4,                      % sets default tabsize to 2 spaces
	captionpos=b,                   % sets the caption-position to bottom
	breaklines=true,                % sets automatic line breaking
	breakatwhitespace=false,        % sets if automatic breaks should only happen at whitespace
	title=\lstname,                 % show the filename of files included with \lstinputlisting;
	                                % also try caption instead of title
	escapeinside={\%*}{*)},         % if you want to add a comment within your code
	morekeywords={*,...}            % if you want to add more keywords to the set
}

% adjust table padding
\setlength{\tabcolsep}{8pt} \renewcommand{\arraystretch}{1.5}

\newcommand{\requestURL}[1]{\textit{#1}}

\newcommand{\request}[9]
{\subsection{#1}
\begin{tabular}{|p{0.2\textwidth}|p{0.7\textwidth}|}
\hline
  URL & \requestURL{#2}\\\hline
    Beschreibung & #3\\\hline
  HTTP-Method & #4\\\hline
  Server-Typ & #5\\\hline
  Authentifizierung & #6\\\hline
  SSL & #7\\\hline
  Parameter & #8\\\hline
  Status-Code Sonderfall & #9\\\hline
 \end{tabular}\vspace*{1em}}
%\request{}%Name
{}%URL
{}%Beschreibung
{}%HTTP-Method
{}%Server-Typ
{}%Auth
{}%SSL
{}%Parameter
{}%Status-Code
 
\title{Protokollspezifikation}
\author{ToureNPlaner Team}

\begin{document}

\maketitle

\tableofcontents

\pagebreak

\section{Einleitung}

	\subsection{Zweck}
	
	Diese Spezifikation dient als Grundlage jeglicher Kommunikation zwischen Client und Server.\\
	Sie enthält alle Funktionen, die der Server bereit stellt, und beschreibt welche Daten diese Funktionen benötigen und welche Sie zurückliefern. 
	Die Spezifikation dient den Teammitgliedern als Grundlage und Richtlinie für die Entwicklung sowohl des Clients als auch des Servers.
	
	\subsection{Leserkreis}
	
	Dieses Dokument ist für den folgenden Leserkreis bestimmt:
	
	\begin{itemize}
		\item Das gesamte Projektteam
		\item Den Kunden
		\item Die künftigen Programmierer bzw. Betreuer dieses Projektes
	\end{itemize}
	
\section{Generelles}

	In diesem Kapitel werden einige der Standards bzw. Protokolle aufgelistet, die zur Implementierung verwendet werden.

	\subsection{HTTP 1.1}

	Die Kommunikation zwischen Client und Server baut auf HTTP (Hyper Text Transport Protocol) in der Version 1.1 auf. Alle Abfragen eines nicht anonymen Servers, bis auf die /info Abfrage gehen über HTTPS.
	Zur Authentifizierung des Nutzers wird das HTTP Basic Authentication Verfahren, ein Feature von HTTP 1.1 verwendet.\\
	Dabei werden Email und Passwort (bzw. in unserem Fall Email und Passwort) mit einer Base64-Kodierung als Teil des HTTP-Headers an den Server gesendet.local
	
	\subsection{JSON}
	 
	JSON, kurz für JavaScript Object Notation, wird zur Repräsentation der Daten verwendet.
	Prinzipiell besteht JSON aus zwei Datentypen, die beliebig ineinander geschachtelt werden können. \\
	Arrays beginnen mit „[“ und enden mit „]“ und enthalten eine durch Komma getrennte Liste von Werten. \\
	Objekte beginnen mit „\{“ und enden mit „\}“ und enthalten eine durch Komma getrennte Liste von Schlüssel-Wert-Paaren, die jeweils durch einen Doppelpunkt getrennt sind.
	Dies ist dabei sowohl analog zu einer klassischen Hashmap als auch zu Javascript Objekten welche als Hashmaps verwaltet werden.
	
	\subsection{World Geodetic System 1984}
	
	Zur Kodierung der Längen- und Breitengrade einer Position auf der Karte wird das WGS 84 verwendet. Alle Koordinaten werden dabei als Integer im Format $x*10^7$ gespeichert.
	
	\subsection*{Serverart}
	Der Server kann in zwei verschiedenen Modi laufen: Privat/Öffentlich. Im öffentlichen Modus kann jeder Probleminstanzen bearbeiten lassen. Im privaten ist ein Benutzerkonto nötig, um Probleminstanzen bearbeiten zu lassen.
	
	\subsection{Standardablauf einer Abfrage}
	Bei erfolgreicher Bearbeitung einer Abfrage antwortet der Server mit dem HTTP-Status-Code 200 OK und gibt eine evtl. Antwort im HTTP-Body als JSON Objekt zurück. Bei einem Fehler gibt der Server einen entsprechenden Status-Code $>$= 400 und weitere Informationen zum Fehler im HTTP-Body als JSON Objekt zurück. siehe Fehlerliste %TODO
	
	\subsection{Format der Date-Time-Strings}
	Die Date-Time-Strings sind nach ISO 8061 formatiert. Somit sind zur Übertragung von Date-Time-Angaben zum Beispiel Strings nach folgendem Muster erlaubt: \\
	2000-12-11T13:55:30.000+0000\\
	2000-12-11T13:55:30Z\\
%nach ISO8061 (Date and time of day, Extended format, complete calendar date, complete UTC local time). 
Siehe auch http://en.wikipedia.org/wiki/ISO{\_}8601. Z bedeutet, dass die Zeit in UTC angegeben ist, +0000 ist der Zeitzonen-Offset zur UTC-Zeit.	

	\subsection{Benutzerobjekt}
	\label{benutzerobjekt}
	Für jeden Benutzer müssen in einer Datenbanktabelle die Informationen gespeichert werden, die zum Login sowie zur Rechnungserstellung benötigt werden. 	
	Das sind:
	
\begin{lstlisting}
{
	"userid": 42
	"email": "max.mustermann@online.de",
	"password": "1234",
	"firstname": "Max",
	"lastname": "Mustermann",
	"address": "Musterstrasse 10, 12345 Musterstadt",
	"admin": false,
	"status": "verified",
	"registrationdate": "2000-12-11T13:55:30.000+0000",
	"verifieddate": null
}
\end{lstlisting}
	
	``userid'' ist die ID eines Users, die bei bestimmten Adminfunktionen benötigt wird.
	
	Das Passwort sollte nur beim Registrieren oder beim Ändern der Benutzerdaten verschickt werden. Der Server wird das Passwort nie mitschicken.
	
	``status'' kann nur die Werte ``needs{\_}verification'' oder ``verified'' annehmen. 
	
	\begin{tabular}{|p{0.2\textwidth}|p{0.7\textwidth}|}
	\hline
	Statusname 	& Beschreibung \\\hline
	needs{\_}verification 	& Neu registrierter Benutzer, der noch nicht aktiviert wurde\\
	verified 				& Durch den Admin aktivierter Benutzer \\
	\hline
	\end{tabular}	
	
	% Kein Status ``Delete'' oder ``VerificationFailed''. Bei ``Delete'' war ursprünglich angedacht, dass der User sich mit diesem Status zum Löschen vormerken lassen kann, diese Funktion ist aber aktuell nicht mehr vorgesehen.
	
	``registrationdate'' und ``verifieddate'' sind Date-Time-Strings oder null. 
	
	\subsection{Requestlogobjekt}	
	Dient der Speicherung alter Abfragen.
	\label{requestlogobjekt}

\begin{lstlisting}
{
	"requestid": 42,
	"userid": 42,
	"algorithm": "sp",
	"cost": 12324,
	"requestdate": "2000-12-11T13:55:30.000+0000",
	"finisheddate": null,
	"duration": 500,
	"status": "ok"
	
}
\end{lstlisting}

	``requestid'' ist die ID eines Requests, die bei bestimmten (Admin-)Funktionen benötigt wird.

	``userid'' ist die ID eines Users, die bei bestimmten (Admin-)Funktionen benötigt wird.
	
	
	``cost'' sind die Kosten als ganzahliger Wert in Cents.

	``requestdate'' und ``finisheddate'' sind Date-Time-Strings oder null.
	
	``duration'' ist die Berechnungsdauer des Requests in Millisekunden.
	
	``status'' kann nur die Werte ``ok'', ``failed'' oder ``pending'' annehmen.
	
	\begin{tabular}{|p{0.2\textwidth}|p{0.7\textwidth}|}
	\hline
	Statusname 	& Beschreibung \\\hline
	ok 			& Request wurde erfolgreich berechnet\\
	failed 		& Request kann nicht berechnet werden \\
	pending 	& Request ist in Berechnung oder wird noch berechnet. Der Status kann auch bei beendeter Berechnung (egal ob erfolgreich oder erfolglos) auftreten, wenn die Datenbank durch einen Verbindungsfehler nicht aktualisiert werden konnte. \\
	\hline
	\end{tabular}	

\subsection{Requestobjekt}	

	Dient dem Transfer von der Algorithmenanfragen vom Server zum Client und zur Speicherung in der Datenbank.
\label{requestobjekt}
\subsubsection{Beispiel}
\begin{lstlisting}
{
    "version": 1,
    "points": [
         { "lt": -95123000, "ln": 81200000, "k": 100 },
         { ... }
    ],
    "constraints": { "t": 100 }
}    	
	\end{lstlisting}
	
		\subsubsection*{Erklärung}
	
	    \noindent \begin{tabular}{|c|p{12cm}|}
	    	\hline
	    	\textbf{Feld} & \textbf{Beschreibung} \\ 
	    	\hline \hline
	    	
	    	version & Die Protokollversion, die der Client verwendet.\\
	    	\hline
	    	
	    	points & Array aller Punkte. Der erste Punkt ist der Startpunkt, der Letzte falls nötig der Zielpunkt. \\ 
	    	\hline
	    	
	    	points.lt & Latitude: Der Breitengrad des Punktes nach WGS 84. Als Integer in 10E7. \\ 
	    	\hline
	    	
	    	points.ln & Longitude: Der Längengrad des Punktes nach WGS 84. Als Integer in 10E7. \\
	    	\hline
	    	
	    	points.k & Ein Beispielconstraint, der für jeden Punkt benötigt wird.\\ 
	    	\hline
	    	
	    	constraints & Array globaler Constraints.\\ 
	    	\hline
	    	
	    	constraints.t & Ein Beispielconstraint, der an keinen Punkt gebunden ist.\\ 
	    	\hline
	    \end{tabular}

\subsection{Responseobjekt}	
Dient dem Transfer der Algorithmenantworten zwischen Server und Client und zum speichern in der Datenbank. 
\label{responseobjekt}
		\begin{lstlisting}
{
	"requestid": 42,
	"constraints": {"maxAltitudeDifference": 30, ... },
	"points": [
		{"lt": -95123000, "ln": 81200000, ... },

		 ...,		
		
		{ ... }
	],
	"way": [
	
		[{"lt": -95123000, "ln": 81200000 }, ... ],
		
		 ...,
		 
		[{...}, ... ]
	
	],
	"misc": {
		"distance": 100,
		"apx": 0.5
	}
}
	\end{lstlisting}
\subsubsection*{Erklärung}
		
		\begin{tabular}{|c|p{12cm}|}
			\hline
			\textbf{Feld} & \textbf{Beschreibung} \\ 
			\hline \hline

			requestid & Die ID des Requests, wodurch die Daten des Requests erneut abgefragt werden können. Wird nur im privaten Modus angegeben.\\
			\hline
			
			constraints & Ein JSON-Objekt mit Constraintdaten für den Algorithmus\\
			\hline
			
			points & Ein Array aller bei der Abfrage übergebender Punkte. Die Reihenfolge wird vom Algorithmus angepasst. In der Abfrage übergebene Metadaten bleiben dabei erhalten. Dient dem Client zum Anzeigen des Ergebnisses z.B. in einer Liste. Dazu kann z.B. bei kürzester Weg Abfragen die Distanz zum vorherigen Knoten mitgegeben werden.\\
			\hline
			
	    	points.lt & Latitude: Der Breitengrad des Punktes nach WGS 84. Als Integer in 10E7.\\ 
	    	\hline
	    	
	    	points.ln & Longitude: Der Längengrad des Punktes nach WGS 84. Als Integer in 10E7.\\
	    	\hline
	    	
	    	way & Ein Array aller Punkte, die auf der Route liegen. Dient dem Client zum Anzeigen des Weges auf der Karte.\\
			\hline
			
	    	way.lt & Latitude: Der Breitengrad des Punktes nach WGS 84. Als Integer in 10E7.\\ 
	    	\hline
	    	
	    	way.ln & Longitude: Der Längengrad des Punktes nach WGS 84. Als Integer in 10E7.\\
	    	\hline
	    	
	    	misc & Ein JSON-Objekt mit algorithmusspezifischen Ergebnissen, welche nicht zum eigentlichen Pfad gehören \\
	    	\hline
	    	
	    	misc.distance & Ein Beispielergebnis, in diesem Fall die Distanz zwischen Start und Ziel.\\
	    	\hline
		\end{tabular}


\subsection{Feste Algorithmen} 
Hier findet sich eine Liste aller Algorithmen, die der Server dem Client unter einem festen Namen zur Verfügung stellt.

\begin{description}
\item[NNS] Nearest Neighbor Search, gibt für eine Liste von Punkten, eine Liste von Punkten zurück, die auf dem jeweils nächsten Knoten im Graph liegen.
\end{description}

\clearpage
\section{Server}

\request{Serverinformation}
{/info}
{Dient zur Abfrage der vorhandenen Algorithmen und anderer Eigenschaften.}
{GET}
{Öffentlich und privat}
{Nie}
{Nie}
{Keine}
{200 OK}
\subsubsection{Beispiel}
	
	\begin{lstlisting}
{
    "version": 1.0,
    "servertype": public/private,
    "sslport": 443,
    "algorithms": [
        {
            "version": 2,
            "name": "Shortest Path",
            "urlsuffix": "sp",
            "details": {
                "hidden" : false,
                "minPoints": 2,
                "sourceIsTarget": true 
            },
            "pointconstraints": [                    
                {
                    "id": "height",
                    "description": "Height",
                    "type": "meter",
                    "min": 0.0,
                    "max": 2000.0
                }
            ],
            "constraints": [
                {
                    "id": "maxAltitudeDifference",
                    "description": "Maximum Altitude Difference",
                    "type": "meter",
                    "min": 0.0,
                    "max": 2000.0
                }
            ]
        }, { ... }
    ],
}
    \end{lstlisting}
    \subsubsection*{Erklärung}
    
    \begin{tabular}{|c|p{14cm}|}
    	\hline
    	\textbf{Feld} & \textbf{Beschreibung} \\ 
    	\hline \hline
    	
    	version & Die Version des Protokolls, die der Server spricht. \\ 
    	\hline
    	
    	servertype & Gibt an, ob der Server im öffentlich oder privaten Modus läuft. \\
    	\hline
    	sslport & Falls angegeben antwortet der Server dort auf SSL-Anfragen.\\
    	\hline
    	
    	algorithms & Ein Array aller Algorithmen, die der Server unterstützt. \\
    	\hline
    \end{tabular}
    
    \subsubsection*{Algorithmen}
    
    \noindent \begin{tabular}{|c|p{11.5cm}|}
    	\hline
    	\textbf{Feld} & \textbf{Beschreibung} \\ 
    	\hline \hline
    	
    	version & Die Version des des Algorithmus. \\ 
    	\hline
    	
    	name & Der Name des Algorithmus. \\
    	\hline
    	
    	hidden & Gibt an, ob der Algorithmus dem Benutzer angezeigt wird. \\
    	\hline
    	
    	urlsuffix & Der Requeststring der Anfrage. \\
    	\hline
    	
    	pointconstraints & Array von Constraints, die für jeden einzelnen Punkt benötigt werden. \\
    	\hline
    	
    	pointconstraints.id & Eindeutiger Name des Constraints. \\
    	\hline
    	
    	pointconstraints.description & Name des Constraints, der einem Clientnutzer angezeigt werden soll. \\
    	\hline
    	
    	pointconstraints.type & Der Typ des Constraints, um dem Nutzer entsprechende Anzeigen/Hilfen geben zu können. Koordinatenconstraints werden immer angenommen. Zur Verfügung stehende Typen können der Tabelle unten entnommen werden.\\ 
    	\hline
    	
    	pointconstraints.min & Minimaler Wert des Constraints. \\
    	\hline
    	
    	pointconstraints.max & Maximaler Wert des Constraints. \\
    	\hline
    	
    	constraints & Constraints, die nur einmal pro Route benötigt werden und nicht Punktspezifisch sind. Enthält Objekte wie in pointconstraints. \\
    	\hline
    	
    	details.minPoints & Die minimale Anzahl an Punkten, die benötigt wird um eine sinnvolle Antwort vom Algorithmus zu erhalten.\\ 
		\hline
		
		details.sourceIsTarget & Gibt an, ob Start und Ziel der gleiche Punkt sind.\\    		\hline    		
    	details.hidden & Soll der Algorithmus dem Benutzer angezeigt werden.\\ 
		\hline

   	\end{tabular}
   	
   	\subsubsection*{Constraint Types}
   	Es stehen für Constraints folgende Constrainttypes zur Verfügung.\\
   	Die Datentypen beziehen sich hierbei auf die JSON-Datentypen.\\
   	\\
   	 \noindent \begin{tabular}{|c|p{3.5cm}|}
    	\hline
    	\textbf{Name} & \textbf{Datentyp} \\ 
    	\hline \hline
    	
    	meter & Fließkommazahl \\ 
    	\hline
    	
    	price & Fließkommazahl \\
    	\hline
    	
    	integer & Ganzzahl \\
    	\hline
    	
    	float & Fließkommazahl \\
    	\hline
    	\end{tabular}
   	\clearpage
\section{Benutzer}

Auf privaten Servern benötigt ein Benutzer ein Konto und muss sich authentifizieren, um einen Algorithmus auf dem Server laufen lassen zu können.
Deshalb werden dem Benutzer Formulare zum Registrieren, Einloggen sowie Bearbeiten seines Accounts gegeben, sowie eine Auflistung der durchgeführten Requests sowie ihrer Kosten.

\request{Registrierung}%Name
{/registeruser}%URL
{Dient zur Erstellung eines Benutzerkontos.\par Dazu werden E-Mail, Passwort, Vor- und Nachname sowie die Rechnungsadresse übermittelt. 
Eine Validierung der Eingaben findet sowohl auf dem Server als auch auf dem Client statt.}%Beschreibung
{POST}%HTTP-Method
{Privat}
{Notwendig für direktes Freischalten (Admin)}%Auth
{Immer}%SSL
{Keine}%Parameter
{401 Unauthorized: Admin Authentifizierung fehlgeschlagen \par 403 Forbidden: Benutzer schon vorhanden}%Status-Code
		
\subsubsection{Request}
Ein \texttt{\$\{BenutzerObjekt\}}, siehe \nameref{benutzerobjekt}.
		
\request{Authentifizierung}%Name
{/authuser}%URL
{Überprüfung von Email und Passwort.}%Beschreibung
{GET}%HTTP-Method
{Privat}%Server-Typ
{Immer}%Auth
{Immer}%SSL
{Keine}%Parameter
{401 Unauthorized: Email/Passwort falsch}%Status-Code
\subsubsection{Request}
Leer.
\subsubsection{Response}
Ein \texttt{\$\{BenutzerObjekt\}}, siehe \nameref{benutzerobjekt}.
\clearpage

\request{Benutzerdaten abfragen}%Name
{/getuser}%URL
{Holt die eigenen Benutzerdaten. Falls der Parameter ``id'' angegeben ist, werden die Daten des Benutzers mit dieser ID geliefert (Admin).}%Beschreibung
{GET}%HTTP-Method
{Privat}%Server-Typ
{Immer}%Auth
{Immer}%SSL
{``id'' (ganzahlige UserID)}%Parameter
{401 Unauthorized: Email/Passwort falsch. Keine Rechte}%Status-Code
\subsubsection{Request}
Leer.
\subsubsection{Response}
Ein \texttt{\$\{BenutzerObjekt\}}, siehe \nameref{benutzerobjekt}.
\clearpage


\request{Benutzerdaten ändern}%Name
{/updateuser}%URL
{Ändert die eigenen Benutzerdaten. Falls der Parameter ``id'' angegeben ist, werden die Daten des Benutzers mit dieser ID geändert (Admin). Wenn der Benutzer kein Admin ist, kann er nur sein Passwort ändern.}%Beschreibung
{POST}%HTTP-Method
{Privat}%Server-Typ
{Immer}%Auth
{Immer}%SSL
{``id'' (ganzahlige UserID)}%Parameter
{401 Unauthorized: Email/Passwort falsch.\par
 403 Forbidden: Keine Admin Rechte}%Status-Code		
\subsubsection{Request}
Ein \texttt{\$\{BenutzerObjekt\}}, siehe \nameref{benutzerobjekt}.
Wenn der Benutzer sein Passwort ändern will setzt er das Passwort Feld im Benutzerobjekt. Wenn er es nicht setzen will darf er das Feld nicht angeben. Im Benutzerobjekt müssen nur die Felder mitgeschickt werden, die der Absender des Requests auch ändern kann, siehe folgende Tabelle. Es müssen immer alle Pflichtfelder im Benutzerobjekt angegeben werden. Das Passwort-Feld ist optional, es soll nur angegeben werden wenn das Passwort geändert werden soll. Ein normaler Benutzer muss also nur sein Passwort mitschicken, kann aber auch ansonsten nichts ändern. 

	\begin{tabular}{|p{0.2\textwidth}|p{0.15\textwidth}|p{0.12\textwidth}|p{0.12\textwidth}|p{0.12\textwidth}|p{0.15\textwidth}|}
	\hline
	Parameter 			& Typ 		& Änderbar durch Admin & Pflichtfeld für Admin & Änderbar durch Benutzer & Pflichtfeld für Benutzer \\\hline
	userid 				& Integer 			& Nein & Nein 	& Nein & Nein\\
	email 				& String 			& Ja   & Ja 	& Nein & Nein\\
	password 			& String 			& Ja   & Nein 	& Ja   & Nein\\
	firstname 			& String 			& Ja   & Ja 	& Nein & Nein\\
	lastname 			& String 			& Ja   & Ja 	& Nein & Nein\\
	address 			& String 			& Ja   & Ja 	& Nein & Nein\\
	admin 				& Boolean 			& Ja   & Ja 	& Nein & Nein\\
	status 				& Enum	 			& Ja   & Ja 	& Nein & Nein\\
	registrationdate 	& Date-Time-String 	& Nein & Nein 	& Nein & Nein\\
	verifieddate 		& Date-Time-String 	& Nein & Nein	& Nein & Nein\\
	\hline
	\end{tabular}

\clearpage

\request{Abfragenverlauf abrufen}%Name
{/listrequests}%URL
{Ruft die eigenen Abfragen ab, falls kein ``id''-Parameter übergeben wurde. Der ``id''-Parameter darf nur von Admins genutzt werden, nähere Informationen dazu stehen in der folgenden Auflistung. Die Liste wird absteigend nach dem Requestdatum (wann der Request beim Server gespeichert wurde) sortiert an den Client geschickt.
\par
Parameter für alle Nutzer:
\begin{itemize}
		\item \mbox{``limit''} limitiert die Anzahl der zurückgelieferten Datensätze. \mbox{``offset''} gibt den Offset zum ersten Datensatz an, gibt also an wie viele Datensätze übersprungen werden. Limit und Offset müssen immer übergeben werden. 
\end{itemize}

Parameter nur für Admins:
	\begin{itemize}
		\item Falls der Parameter ``id'' angegeben ist, werden die Daten des Benutzers mit dieser UserID abgerufen. Falls der Parameter ``id'' den Wert ``all'' (also \mbox{``id=all''}) übergibt, werden die Request-Datensätze aller Benutzer abgerufen (die abgerufenen Datensätze werden aber mit Limit und Offset beschränkt).
\end{itemize}
}%Beschreibung
{GET}%HTTP-Method
{Privat}%Server-Typ
{Immer}%Auth
{Immer}%SSL
{``id'' (ganzahlige UserID oder ``all''), ``limit'', ``offset''}%Parameter
{401 Unauthorized: Email/Passwort falsch.\par
 403 Forbidden: Keine Admin Rechte\par
404 Not Found: Kein Benutzer zur angegebenen ID gefunden}%Status-Code		


	
\subsubsection{Response}		
		\begin{lstlisting}
{
	"number": 100
	"requests": [
		REQUESTLOGOBJEKT,
		...
	]
}
		\end{lstlisting}
		
Siehe \nameref{requestlogobjekt}.

		``number'' gibt die Anzahl aller Einträge an, die zum jeweiligen Benutzer gehören, bzw. bei \mbox{``id=all''} wird die Anzahl aller Request-Einträge angegeben.
		
	

\request{Abfrage abrufen}%Name
{/getrequest}%URL
{Ruft einen bestimmten Request, so wie er vom Client gesendet wurde, ab. Der Parameter ``id'' ist ein Pflichtparameter und gibt die ID des Requests an, dessen Daten abgerufen werden sollen. Wenn der Benutzer kein Admin ist und der Request nicht zum Benutzer gehört, wird mit einer Forbidden-Nachricht geantwortet.
}%Beschreibung
{GET}%HTTP-Method
{Privat}%Server-Typ
{Immer}%Auth
{Immer}%SSL
{``id''}%Parameter
{401 Unauthorized: Email/Passwort falsch.\par
 403 Forbidden: Keine Admin Rechte\par
 403 Forbidden: Benutzer nicht verifiziert\par
404 Not Found: Kein Request zur angegebenen ID gefunden}%Status-Code		

\request{Antwort abrufen}%Name
{/getrequest}%URL
{Ruft eine bestimmte Antwort, so wie sie der Server gesendet hatte, ab. Der Parameter ``id'' ist ein Pflichtparameter und gibt die ID des Responses an, dessen Daten abgerufen werden sollen. Wenn der Benutzer kein Admin ist und der Response nicht zum Benutzer gehört, wird mit einer Forbidden-Nachricht geantwortet.
}%Beschreibung
{GET}%HTTP-Method
{Privat}%Server-Typ
{Immer}%Auth
{Immer}%SSL
{``id''}%Parameter
{401 Unauthorized: Email/Passwort falsch.\par
 403 Forbidden: Keine Admin Rechte\par
 403 Forbidden: Benutzer nicht verifiziert\par
404 Not Found: Kein Response zur angegebenen ID gefunden}%Status-Code



\section{Administration}

Benutzer, bei denen das Adminflag gesetzt ist, können die anderen Benutzer im System verwalten. 
Dazu gibt es zum einen eine Übersicht über alle Benutzer, sowie Einzelansichten des ausgewählten Benutzers, die prinzipiell wie die entsprechende Funktion für normale Benutzer aufgebaut sind.
Es wird nur noch ein weiterer Parameter (id) übergeben, um den Benutzer auszuwählen.

Zusätzlich gibt es für Administratoren die Möglichkeit Benutzer zu löschen und Benutzer ohne Registrierungsformular anzulegen.

\request{Benutzerliste abrufen}%Name
{/listusers}%URL
{``limit' limitiert die Anzahl der zurückgelieferten Datensätze. ``offset'' gibt den Offset zum ersten Datensatz an, gibt also an wie viele Datensätze übersprungen werden. Limit und Offset müssen immer übergeben werden.}%Beschreibung
{GET}%HTTP-Method
{Privat}%Server-Typ
{Immer}%Auth
{Immer}%SSL
{``limit'', ``offset''}%Parameter
{401 Unauthorized: Email/Passwort falsch. Keine Rechte}%Status-Code
\subsubsection{Response}		
		\begin{lstlisting}
{
	"number": 100
	"users": [
		BENUTZEROBJEKT,
		...
	]
}
		\end{lstlisting}
		
Siehe \nameref{benutzerobjekt}.

		``number'' gibt die Anzahl aller Benutzer in der Datenbank an.
		
\clearpage

\request{Benutzer löschen}%Name
{/deleteuser}%URL
{Löscht den Benutzer mit der angegeben ``id''.}%Beschreibung
{GET}%HTTP-Method
{Privat}%Server-Typ
{Immer}%Auth
{Immer}%SSL
{``id''}%Parameter
{401 Unauthorized: Email/Passwort falsch.\par
 403 Forbidden: Keine Admin Rechte}%Status-Code
\clearpage
   	
\section{Algorithmen}

Da die benötigten Daten je nach Algorithmus unterschiedlich sind, wird hier nur ein generelles Format vorgegeben. Die korrekten Parameter können der Dokumentation des jeweiligen Algorithmus bzw. den Serverinformationen entnommen werden.

\request{Abfrage starten}%Name
{/alg\texttt{\$\{shortname\}}}%URL
{Startet einen Berechnungsvorgang für Algorithmus \texttt{\$\{shortname\}} mit der mitgegebenen Probleminstanz.}%Beschreibung
{POST}%HTTP-Method
{Privat und öffentlich}%Server-Typ
{Nötig wenn privat}%Auth
{Nötig wenn privat}%SSL
{Keine}%Parameter
{Gemeinsam:\par
400 Bad Request: Die Probleminstanz ist fehlerhaft\par
503 Service Unavailable: Server ausgelastet\par
Privat:\par
 401 Unauthorized: Email/Passwort falsch\par
 403 Forbidden: Keine Admin Rechte}%Status-Code

\subsubsection{Beispiel}
	siehe \nameref{requestobjekt}
    
    \subsubsection{Antwort}	
	siehe \nameref{responseobjekt}
	
		
\section{Fehlermeldungen}
\subsection{Allgemeine Form}
Am Beispiel eines falsch formatierten JSON Requests.
Der \textit{details} Parameter ist dabei Optional und dient ausschließlich dem leichteren Debuggen von 
Clients.
	\begin{lstlisting}
{
	"errorid": "EBADJSON",
	"message" : "Failed to parse request",
	"details" : "Expected , found . at position 7"
}
	\end{lstlisting}

\subsection{Übersicht}
\begin{tabular}{|c|p{0.35\textwidth}|p{0.35\textwidth}|}
\hline
\textbf{Error Id}	& \textbf{Message}			& \textbf{Beschreibung} \\
\hline
EAUTH		& Wrong email or password	& Authenifizierung fehlgeschlagen \\
\hline
EBADJSON	& There was an error parsing the request & Es trat ein Fehler beim parsen des gelieferten JSON auf \\
\hline
EBUSY		& This server is currently too busy to fullfill the request	& Der Server ist ausgelastet und kann den Request derzeit nicht bearbeiten \\
\hline
EDATABASE	& The server can't contact it's database	& Die Benutzer-DB bzw Abrechnungs-DB ist nicht erreichbar \\
\hline
ELIMIT		& The given limit is invalid			& Das angegebene Limit ist ungültig (z.B. negativ) \\
\hline
ENOUSERID	& The given user id is unknown to this server	& Die Benutzer-ID ist unbekannt \\
\hline
ENOREQUESTID	& The given request id is unknown to this server	& Die Request-ID ist unbekannt \\
\hline
ENOTADMIN	& You are not an admin				& Der Benutzer ist kein Admin \\
\hline
ENOTVERIFIED	& User account is not verified	& Der Benutzer ist noch nicht verifiziert (aktiviert) \\
\hline
EOFFSET		& The given offset is invalid 			& Das angegebene Offset ist ungültig (z.B. negativ) \\
\hline
EREGISTERED	& This email is already registred	& Die Emailadresse ist schon vergeben \\
\hline
EUNKNOWNALG	& An unknown algorithm was requested	& Angeforderter Algorithmus unbekannt \\
\hline
EUNKNOWNURL	& An unknown URL was requested	& Angeforderte URL unbekannt \\
\hline
ECOMPUTE	& (Depends on algorithm)	& (von Algorithmus abhängig) \\
\hline
\end{tabular}


\end{document}
