\documentclass[ngerman,titlepage,parskip=true]{scrartcl}
\usepackage[utf8]{inputenc}
\usepackage{array}
\usepackage{babel}
\usepackage{wrapfig}
\usepackage{longtable}
\usepackage[unicode=true,pdfusetitle,bookmarks=true,bookmarksnumbered=false,bookmarksopen=false,breaklinks=false,pdfborder={0 0 0},backref=false,colorlinks=false]{hyperref}
\usepackage{listings}
\usepackage{color}

\title{Graphfilespecification}
\author{ToureNPlaner Team}

\begin{document}

\maketitle

\tableofcontents

\section{Introduction}
To compute routes and tour on a street network ToureNPlaner needs to have a memory representation of the street network as 
a graph where each edge represents a section of a road (two edges in oppisite direction for a street that can be traveled in both directions).
To load new street networks into the ToureNPlaner server we use a text based representation of the graph.
This representation is described in this document so that graphs in this format can be generated and loaded into ToureNPlaner.

\section{Basics}
The graph file format used by ToureNPlaner is a text based representation of a graph with additional data on the nodes (e.g. coordinates)
and edges (e.g. travel time).
It uses the file ending ``.txt'' and is encoded in ASCII/UTF-8 with line-feed (Unix) line endings (though Windows line endings will work).

\section{Structure}
Each graph file has the following three sections
\begin{itemize}
 \item Header
 \item Nodes
 \item Edges
\end{itemize}

\section{Section Descriptions}
\subsection{Conventions}
To describe the format we use place holders of the form ``<place holder name>'' which need to replaced with
actual values.

\subsection{Header}
The header consits of two lines:
\begin{verbatim*}
<numNodes>
<numEdges>
\end{verbatim*}

\begin{itemize}
 \item 
 \textbf{numNodes} the number of nodes in the graph as a decimal number string e.g. ``14505''
 \item
 \textbf{numEdges} the number of edges in the graph (including shortcuts if Contraction Hierarchies is used) e.g. ``59600''
\end{itemize}


\subsection{Nodes}
The nodes section consists of number of nodes lines with each line describing a node of the graph in the following way.
\begin{verbatim*}
<latitude> <longitude> <height> <nodeRank>
\end{verbatim*}

\begin{itemize}
\item \textbf{latitude} the geographical latitude of the node times $10^7$ e.g. ``484878025'' for ``48.4878025 \textdegree N''
\item \textbf{longitude} the geographical longitude of the node times $10^7$ e.g. ``91901961'' for ``9.1901961 \textdegree E''
\item \textbf{height} the height (geographical elevation above NN) in meters
\item \textbf{nodeRank} the Contraction Hierachy rank of the node (optional if no Contraction Hierachy is used) if the
CH hasn't been computed completely all uncontracted nodes must have their rank set to 32 bit signed int maximum ($2^31 - 1$) 
\end{itemize}



\end{document}