\documentclass[DIV15, fontsize=10pt, ngerman,twocolumn,titlepage,parskip=true]{scrartcl}
\usepackage[utf8]{inputenc}
\usepackage{array}
\usepackage{babel}
\usepackage{wrapfig}
\usepackage{longtable}
\usepackage[unicode=true,pdfusetitle,bookmarks=true,bookmarksnumbered=false,bookmarksopen=false,breaklinks=false,pdfborder={0 0 0},backref=false,colorlinks=false]{hyperref}
\usepackage{listings}
\usepackage{color}

\lstset{ %
	language=C,                % the language of the code
	basicstyle=\footnotesize,       % the size of the fonts that are used for the code
	numbers=left,                   % where to put the line-numbers
	numberstyle=\footnotesize,      % the size of the fonts that are used for the line-numbers
	stepnumber=2,                   % the step between two line-numbers. If it's 1, each line 
	                                % will be numbered
	numbersep=5pt,                  % how far the line-numbers are from the code
	backgroundcolor=\color{white},  % choose the background color. You must add \usepackage{color}
	showspaces=false,               % show spaces adding particular underscores
	showstringspaces=false,         % underline spaces within strings
	showtabs=false,                 % show tabs within strings adding particular underscores
	frame=single,                   % adds a frame around the code
	tabsize=4,                      % sets default tabsize to 2 spaces
	captionpos=b,                   % sets the caption-position to bottom
	breaklines=true,                % sets automatic line breaking
	breakatwhitespace=false,        % sets if automatic breaks should only happen at whitespace
	title=\lstname,                 % show the filename of files included with \lstinputlisting;
	                                % also try caption instead of title
	escapeinside={\%*}{*)},         % if you want to add a comment within your code
	morekeywords={*,...}            % if you want to add more keywords to the set
}

% adjust table padding
\setlength{\tabcolsep}{8pt} \renewcommand{\arraystretch}{1.5}
 
\title{Projektbericht}
\author{NAME}
\date{18.\,April\,2012}

\begin{document}

\maketitle

% Jeweils eine Seite
\section*{Projektübersicht}
\subsection*{Ziel des Projekts}

Ziel des Projektes war die Entwicklung einer Server-Client-Anwendung, die als einfach erweiterbare Plattform dient, um die Eingabe, Verarbeitung und Ausgabe algorithmischer Probleme, die auf Straßendaten und Kartendaten basieren, zu ermöglichen.

\subsection*{Vorgaben}

Vorgaben waren die Verwendung der Programmiersprache Java für den Serverteil und die Erstellung eines Web, sowie eines Android-Clienten und die Implementierung der folgenden Algorithmen: Shortest Path, Constrained Shortest Path und das Traveling Salesman Problem. Außerdem sollten beide Teile möglichst einfach um weitere Algorithmen erweiterbar sein. Die aufbereiteten Kartendaten wurden vom Kunden bereitgestellt.

Neben einer Variante, die für den öffentlichen Betrieb ausgelegt sein soll, wurde außerdem eine Variante mit Zugriffskontrolle gefordert bei der man sich als Benutzer anmelden muss und seine alten Abfragen auch abrufen kann. Diese Daten können dann auch für die Abrechnung der potentiell teuren (im Sinne von Rechner-Zeit) Abfragen dienen.

\subsection*{Organisation}

Das Projekt wurde in zwei Teams aufgeteilt: Das Server-Team bestehend aus Christoph Haag, Sascha Meusel, Niklas Schnelle, Peter Vollmer und das Client-Team, welches später in 2 Teile aufgeteilt wurde: Web-Client (Philipp Gildein, Huy Viet Le, Kevin Wenz) und Android-Client (Vivian Eggert, Steffen Hanikel). Auf Grund der geringen Anzahl an Teilnehmern kam es zu einer Doppelbesetzung von Rollen: Projektleiter: Steffen Hanikel, Stellvertretender Projektleiter: Peter Vollmer, Administrator: Christoph Haag.

\subsection*{Vorgehen}

Nach der ersten Phase, in der mehrere Projektpläne erstellt und anschließend einer ausgewählt wurde, wie es vom Rahmen des Studienprojekts so vorgegeben ist, sind wir auf agiles Vorgehen umgestiegen. So wurden wöchentlich projektinterne Meetings gehalten und monatlich Treffen mit dem Kunden durchgeführt, bei denen der Fortschritt im Projekt und eventuelle Änderungen der Anforderungen besprochenen wurden, damit das System auch nah an den Wünschen des Kunden blieb.

Um eine gute Zusammenarbeit der Komponenten zu gewährleisten wurde auf das Prinzip der kontinuierliche Integration gesetzt, indem alle Komponenten täglich, automatisch neu gebaut und somit auch von allen im Projekt getestet werden konnten.
Als Versionskontrollsystem wurde jeweils für die Teams und für die Dokumentation ein separates Git-Archiv verwendet. Für die Verwaltung von Aufgaben und Protokollen hat sich Redmine als Wiki und Ticket-System im Einsatz bewährt.

\subsection*{Aufbau}
Es wurde ein eigenes Protokoll entwickelt, welches auf \mbox{HTTP/S} mit dem Datenformat JSON wegen der breiten Unterstützung, einfachen Handhabung und guter Erweiterbarkeit, aufbaut. Um zu verhindern, dass bei dem Hinzufügen von neuen Algorithmen alle Klienten angepasst werden müssen, wurde vor allem Wert auf eine flexible Art der Definition von Parametern für Algorithmen gelegt. Um eine gute Skalierbarkeit auf mehrere unabhängige Rechner per Load-Balancing zu ermöglichen

So holen sich die Clients einmal beim Starten eine Liste der verfügbaren Algorithmen und deren Anforderungen vom Server. In einer typischen Anfrage werden dann Punkte, die der Benutzer auf der Karte ausgewählt hat zur Berechnung an den Server geschickt. Zusätzlich können die Clients auch angwiesen werden verschiedene Daten (Zahlen, Zeichketten) für den Algorithmus oder für jeden einzelnen Punkt mitzuliefern. z.\,B. die maximale Höhe bei Constraint Shortest Path. Als Ergebnis liefert der Server dann diese Liste mit zusätzlichen Informationen (Reisezeit, Distanz) und eventuell in einer anderen Reihenfolge (Traveling Salesman Problem), sowie eine Liste von Wegstücken, die auf der Karte eingezeichnet werden sollen, zurück.

Im privaten Modus werden zusätzlich die Abfragen in einer Datenbank gespeichert, so dass sie u.a. für spätere Abrechnungszwecke verfügbar sind. Benutzer müssen sich deshalb einmalig beim Server anmelden, damit sie sich dann bei jder Abfrage authentifizieren können. Sie können die Ergebnisse alter Abfragen auch jederzeit wieder abrufen. Administratoren können Abfragen aller Benutzer abfragen und auch die Daten aller Benutzer ändern oder diese löschen. 	
\clearpage
\section*{Teamübersicht Server}
\subsection*{Spezifische Vorgaben}
Um die leichte Weiterentwicklung der Serverkomponente durch kommende Studentengruppen und Institutsmitarbeiter zu erleichtern, war als Programmiersprache Java zu verwenden. Desweiteren war, wie auch in den anderen Komponenten von ToureNPlaner auf eine leichte Erweiterbarkeit um neue Algorithmen zu achten, was sich in einem relativ generischen Design widerspiegelt.\\
Um möglichst einfach neue Clients entwickeln zu können und vorhandene Technologien wie Load Balancer direkt nutzen zu können wurde früh festgelegt, dass der Server per HTTP und JSON ansprechbar sein soll.\\
Eine weitere Vorgabe ergab sich, durch das beim Institut verfügbare Kartenmaterial, so war Kartenmaterial mit vorberechneten Contraction Hierarchies vorhanden, so dass diese bei den Algorithmen genutzt werden konnten und in die Entwicklung der Graphrepräsentation einflossen. Dabei mussten auch Änderungen im Format der Graphdaten, auch in fortgeschrittenen Entwicklungsphasen, schnell bewältigt werden können.
\subsection*{Organisatorisches}
Mit 4 Personen war das Serverteam das größte Teilteam des Projekts, weshalb eine, wenn auch flache, Hierarchie innerhalb des Teams sinnvoll erschien. So wurde Niklas Schnelle als Teamleiter gewählt, dem eine gewisse Entscheidungshoheit in Architektur- und Entwicklungsfragen zu viel, um diese schnell und unbürokratisch treffen zu können.\\
Im Laufe des Projekts wurden dann alle Aufgabenpakete zu gleichen Teilen über die 4 Teammitglieder verteilt, wobei sich Fachgebiete herausbildeten.
So übernahm Sascha Meusel große Teile der für die Benutzer und Requestverwaltung nötigen Bereiche wie zum Beispiel die Entwicklung der Datenbankanbindung.
Peter Vollmer konzentrierte sich auf die Entwicklung des Constrained Shortest Path Algorithmus und des Systemtests, während Christoph Haag die Qualitätssicherung übernahm und bei der Entwicklung der Graphrepräsentation große Teile übernahm. Der Teamleiter schließlich entwickelte die Grundstruktur des Servers und der Graphrepräsentation, die Integration mit den verwendeten Bibliotheken und Teilmodulen sowie den Shortest Path und Traveling Salesmen Algorithmus.\\
In seiner Struktur lehnten wir uns somit stark an das Chief Programmer Modell an. Der Teamleiter war direkt in die Entwicklung eingebunden und ein Großteil der Entwicklung fand gemeinsam in einem Raum statt, so dass es keinen Bedarf für festgelegte Meetings gab.
\subsection*{Umsetzung}
\subsubsection*{Framework}
Als Grundlage des Servers diente das Netzwerkframework Netty\footnote{\url{http://www.jboss.org/netty}} des JBoss Projekts. Hierbei handelt es sich um ein Java Framework für die Entwicklung Event basierter Serversysteme, welches leicht benutzbare Primitive für die Umsetzung eines HTTP Servers liefert, ohne dabei einen schwergewichtigen Java Application Server oder ähnlichem zu benutzen. Vielmehr wird das HTTP Protokoll direkt in die eigene Software integriert und zur direkten Schnittstelle zur Außenwelt.\\
Zur Behandlung der als JSON kodierten Eingabedaten wurde schließlich die Bibliothek Jackson\footnote{\url{http://wiki.fasterxml.com/JacksonHome}} verwendet, welche auch ein zu JSON kompatibles Binärformat namens \glqq Smile\grqq unterstützt.
\subsubsection*{Graphrepräsentation}
Da Performanz, vor allem bei der Ausführung von Graphalgorithmen, für die Entwicklung des Servers als entscheidende Eigenschaft angesehen wurde, lag besondere Aufmerksamkeit auf der Entwicklung einer leistungsfähigen, schlanken und dennoch leicht zu benutzenden Graphrepräsentation.\\
Aufgrund der Vorraussetzung, den Server in Java zu entwickeln, mussten hierzu einige Kniffe angewendet werden.
So ist es in Java sehr problematisch eine große Anzahl an Objekten anzulegen und zu Verwalten und es werden keine  \glqq struct\grqq Datentypen unterstützt die ohne große Speicherverwaltung und Referenzen auskomme
Daher musste die Struktur des Graphs in einer Menge von Arrays atomar Datentypen wie zum Beispiel \glqq int\grqq abgelegt werden.\\
Um dennoch ein benutzbares Interface zu bieten wurden diese Details, soweit ohne Performance einbußen möglich,  hinter durch den Compiler leicht inlinebarer Methoden versteckt.\\
So können Kanten und Knoten des Graphs über ganzzahlige Ids angesprochen werden, ohne die genaue Repräsentation zu kennen.
\subsubsection*{Datenbankanbindung}
Die Andbindung der für die Verwaltung von Nutzern und gespeicherten Requests benötige Datenbankanbindung wurde mit Hilfe von direktem JDBC programmiert und in einer speziellen Manager Klasse gekapselt.
\subsubsection*{Algorithmen}
Die Entwicklung der eigentlichen Algorithmen erfolgte in Form spezieller Klassen, die sowohl die benötigen Datenstruktuen (z.B. Priority Queue im Dijkstra) als auch die Implementierung des Algorithmus zusammenfassen.\\
Dabei wird für alle Graphalgorithmen in Abstimmung mit den Clients die gleiche Struktur der Eingabedaten verwendet, so dass diese vor Ausführung der Algorithmen bereits vorverarbeitet werden können und diesen direkt als Objekt zur Verfügung stehen. Desweiteren können alle Algorithmen die Daten der Contraction Hierarchie nutzen, aber auch andere Algorithmen. So nutzt zum Beispiel der Traveling Salesmen Algorithmus, zur Berechnung einer kürzeste Wege Distanzmatrix, die Implementierung des Shortest Path Algorithmus.\\

\clearpage
\section*{Teamübersicht Webclient}
\subsection*{Spezifische Vorgaben}

Die Vorgaben für den Webclienten waren, dass er auf aktuellen Versionen der Browser Mozilla Firefox und Google Chrome lauffähig sein soll.
Desweiteren galt es eine gut und intuitiv bedienbare Oberfläche zu gestalten.

Der Webclient sollte außerdem zur Administration eingesetzt werden können.
Darunter fällt die Verwaltung von Nutzern, deren Daten sowie deren getätigten Anfragen.

\subsection*{Organisatorisches}

Da der Webclient neben der Tourenberechnung auch die Funktion der Nutzerverwaltung und Administration bereitstellen sollte, besteht dieses Team  aus drei Mitgliedern. 
Aufgrund der meisten Erfahrung wurde Philipp Gildein zum Teamleiter bestimmt.

Es wurde versucht die Aufgaben aufzuteilen, was ein unabhängiges Arbeiten möglich machen sollte.
Die grobe Einteilung sah wie folgt aus: Philipp Gildein widmete sich hauptsächlich der groben und grundliegenden Struktur des Clienten, ebenso hat er das Layout und die meisten Designaspekte bearbeitet.
Huy Viet Le sorgte dafür, dass die Karte und Route angezeigt werden konnten.
Kevin Wenz kümmerte sich um die Schnittstelle zwischen Client und Server, sowie die Tests.
Aufgrund der vielen übergreifenden Funktionen konnte diese Einteilung nicht immer eingehalten werden.

\subsection*{Umsetzung}

Aufgrund des Einsatzes im Webbrowser wurden Javascript (für den logischen Teil), HTML (für das Layout) und CSS (für die Gestaltung) als Implementierungssprachen verwendet.
Der Client wurde nach dem Model-View-Controller Muster strukturiert.
Durch einen modularen Aufbau ist das einfache Austauschen einzelner (Teil-)Komponenten möglich.

Zur Unterstützung der Mehrsprachigkeit und besseren Übersichtlichkeit im Javascript-Code wurde die gesamte HTML-Struktur in Templates ausgelagert.
Benötigte Komponenten werden daher bei jeden Laden neu zusammengebaut und an die Landessprache angepasst.

\subsection*{Externe Bibliotheken}

Javascript unterstützt nicht alle modernen Programmierkonzepte von Haus aus.
Deshalb setzten wir Backbone.js\footnote{\url{http://documentcloud.github.com/backbone/}} und dessen einzigste Anforderung underscore.js\footnote{\url{http://documentcloud.github.com/underscore/}} ein, um über eine simple MVC-Implementierung zu verfügen.

Zur Manipulierung des HTML-Layouts und CSS-Designs sowie zum Aufruf der Serverfunktionen über HTTP/S wurde jQuery\footnote{\url{http://jquery.com/}} verwendet.
Dessen Funktionalität wurde durch den Einsatz kleinerer Plugins noch erweitert.

Um ein einheitliches Erscheinungsbild der Anwendung gewährleisten zu können, setzten wir auf Bootstrap\footnote{\url{http://twitter.github.com/bootstrap/}}, eine CSS-Biliothek, die von Twitter entwickelt wurde.

Zu Beginn der Entwicklung wurde OpenLayers verwendet um die Karte darzustellen. 
Da sich dies im Verlauf des Projektes als unzureichend bzw. fehlerbehaftet erwies, wurde sie durch Leaflet\footnote{\url{http://leaflet.cloudmade.com/}} ersetzt.

Tests wurden mit QUnit$^3$ (Unittest) und SeleniumHQ\footnote{\url{http://seleniumhq.org/}} (Systemtest) durchgeführt.

\subsection*{Herausforderungen}

Schwierigkeiten stellten sich in der Webclient-Entwicklung in der Entwicklung für verschiedene Browser und deren Versionen dar.
Das Problem hierbei ist, dass verschiedene Browser Programmieranweisungen verschieden interpretieren oder auch gar nicht unterstützen. 
Grundsätzlich wurde diese Schwierigkeit schon eingeschränkt, indem der Client nur in aktueller Version des Firefox und Chrome lauffähig sein muss. Außerdem hat die Verwendung externer Bibliotheken die Problematik ebenfalls entschärft.

Eine weitere Herausforderung war die Verbindung zum Server per HTTPS.
Um eine sichere Kommunikation zu gewährleisten, muss sich der Server durch ein Zertifikat ausweisen.
Wenn dieses Zertifikat nicht offiziell verifiziert wurde, muss es in jeden Browser manuell importiert werden, was Chrome in aktuellen Versionen nur über Umwege unterstützt.
Unsere Lösung besteht nun darin, dass entweder durch den Serverbetreiber ein Zertifikat gekauft werden muss oder der Server und der Webclient auf derselben Domain laufen müssen.
Das Problem besteht zudem nur auf privaten Servern.

\clearpage
\section*{Teamübersicht Androidclient}
\subsection*{Spezifische Vorgaben}
Vorgegeben wurde die Unterstützung der vom Kunden bereitgestellten Handys. Da auf diesen Android in der Version 2.1 lief setzen wir uns als Ziel alle Versionen ab 2.1 zu untersützen. Zur Darstellung der Karte sollte die Biliothek Mapsforge\footnote{\url{http://code.google.com/p/mapsforge/}} verwendet werden.

Auf Verwaltungsfunktionen für Benutzer, sowie Administratorfunktionen wurde verzicht, da Handys auf Grund des kleinen Bildschirms im Vergleich zu normalen PCs oder Laptops eine im allgemeinen eher schlechtere Übersichtlichkeit haben.

\subsection*{Organisatorisches}
Das Android-Team bestand wegen des geringeren Funktionsaufwands nur aus zwei Mitgliedern: Steffen Hanikel und Vivian Eggert. Deshalb wurden die meisten Besprechungen bei Problemen eher spontan gehalten und keine festen Treffen außer den projektweiten vereinbart. Trotzdem wurde auch das Ticket-System intensiv zur Verwaltung von Aufgaben verwendet.

\subsection*{Externe Bibliotheken}
Wie eingangs erwähnt wurde für die Darstellung der Karten Mapsforge eingesetzt, da damit im Gegensatz zur Goolgle Maps API die Darstellung von frei vergfübaren Karten, insbesondere OpenStreetMap Karten, auch offline möglich ist.

Da der JSON-Parser im Android SDK zu langsam war und immer ganze Antworten in den Speicher lud wurde der sehr schnelle und flexible JSON-Parser Jackson\footnote{\url{http://wiki.fasterxml.com/JacksonHome}} verwendet. Auch wenn der große Platzbedarf von $\approx$~1,2~MB ein Nachteil ist.

Leider erst spät im Projekt hat sich die Verwendung der Blibliothek ActionBarSherlock\footnote{\url{http://actionbarsherlock.com/}} als sinnvoll erwiesen. Durch diese können Funktionen der neusten Android Version auch auf älteren Geräten eingesetzt werden. Vor allem die Verwendung der Action Bar ermöglichte eine einfachere, schönere und flexiblere Gestaltung der Oberfläche.

\subsection*{Umsetzung}
\clearpage
\section*{Persönliche Übersicht}
 


\end{document}
