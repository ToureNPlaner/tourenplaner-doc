\subsection*{Spezifische Vorgaben}

Die Vorgaben für den Webclienten waren, dass er auf aktuellen Versionen der Browser Firefox () und Chromium () lauffähig sein soll. Desweiteren galt es eine gut und intuitiv bedienbare Oberfläche zu Gestalten.

Der Webclient sollte außerdem zur Verwaltung von Nutzern und deren Daten dienen.

\subsection*{Organisation}

Die Leitung des Webclienten-Teams übernahm Philipp Gildein. Die einzelnen Komponenten wurde auf einzelne Teammitglieder verteilt, was ein unabhängiges Arbeiten ermöglicht hat. 

\subsection*{Umsetzung}

Aufgrund ihrer weiten Verbreitung, guten Unterstützung und felxiblen Einsetzbarkeit kamen Javascript (für den logischen Teil) und CSS (für die Gestaltung) als Implementierungssprachen zum Einsatz. Der Client wurde nach dem Model-View-Controller Muster strukturiert. Durch einen modularen Aufbau ist das einfache Austauschen einzelner (Teil-)Komponenten möglich.

Die Programmierung wurde hauptsächlich durch die Bibliotheken Backbone.js (MVC-Implementierung für Javascript), JQuery (http/s Anfragen), underscore.js (Erweiterung der Javascript Basisfunktionen) und Twitter's Bootstrap (ermöglicht ein vordefiniertes, einheitliches Erscheinungsbild) unterstützt.