\subsection*{Spezifische Vorgaben}

Die Vorgaben für den Webclienten waren, dass er auf aktuellen Versionen der Browser Mozilla Firefox und Google Chrome lauffähig sein soll.
Desweiteren galt es eine gut und intuitiv bedienbare Oberfläche zu gestalten.

Der Webclient sollte außerdem zur Administration eingesetzt werden können.
Darunter fällt die Verwaltung von Nutzern, deren Daten sowie deren getätigten Anfragen.

\subsection*{Organisation}

Die Leitung des Webclienten-Teams übernahm Philipp Gildein. Die einzelnen Komponenten wurde auf einzelne Teammitglieder verteilt, was ein unabhängiges Arbeiten ermöglicht hat. 

\subsection*{Umsetzung}

Aufgrund des Einsatzes im Webbrowser wurden Javascript (für den logischen Teil), HTML (für das Layout) und CSS (für die Gestaltung) als Implementierungssprachen verwendet.
Der Client wurde nach dem Model-View-Controller Muster strukturiert. 
Durch einen modularen Aufbau ist das einfache Austauschen einzelner (Teil-)Komponenten möglich.

Die Programmierung wurde hauptsächlich durch die Bibliotheken Backbone.js (MVC-Implementierung für Javascript), JQuery (http/s Anfragen), underscore.js (Erweiterung der Javascript Basisfunktionen) und Twitter's Bootstrap (ermöglicht ein vordefiniertes, einheitliches Erscheinungsbild) unterstützt.