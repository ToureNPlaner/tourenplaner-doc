\subsection*{Spezifische Vorgaben}
Vorgegeben wurde die Unterstützung der vom Kunden bereitgestellten Handys. Da auf diesen Android in der Version 2.1 lief setzen wir uns als Ziel alle Versionen ab 2.1 zu untersützen. Zur Darstellung der Karte sollte die Biliothek Mapsforge\footnote{\url{http://code.google.com/p/mapsforge/}} verwendet werden.

Auf Verwaltungsfunktionen für Benutzer, sowie Administratorfunktionen wurde verzicht, da Handys auf Grund des kleinen Bildschirms im Vergleich zu normalen PCs oder Laptops eine im allgemeinen eher schlechtere Übersichtlichkeit haben.

\subsection*{Organisatorisches}
Das Android-Team bestand wegen des geringeren Funktionsaufwands nur aus zwei Mitgliedern: Steffen Hanikel und Vivian Eggert. Deshalb wurden die meisten Besprechungen bei Problemen eher spontan gehalten und keine festen Treffen außer den projektweiten vereinbart. Trotzdem wurde auch das Ticket-System intensiv zur Verwaltung von Aufgaben verwendet.

\subsection*{Externe Bibliotheken}
Wie eingangs erwähnt wurde für die Darstellung der Karten Mapsforge eingesetzt, da damit im Gegensatz zur Goolgle Maps API die Darstellung von frei vergfübaren Karten, insbesondere OpenStreetMap Karten, auch offline möglich ist.

Da der JSON-Parser im Android SDK zu langsam war und immer ganze Antworten in den Speicher lud wurde der sehr schnelle und flexible JSON-Parser Jackson\footnote{\url{http://wiki.fasterxml.com/JacksonHome}} verwendet. Auch wenn der große Platzbedarf von $\approx$~1,2~MB ein Nachteil ist.

Leider erst spät im Projekt hat sich die Verwendung der Blibliothek ActionBarSherlock\footnote{\url{http://actionbarsherlock.com/}} als sinnvoll erwiesen. Durch diese können Funktionen der neusten Android Version auch auf älteren Geräten eingesetzt werden. Vor allem die Verwendung der Action Bar ermöglichte eine einfachere, schönere und flexiblere Gestaltung der Oberfläche.
	
\subsection*{Umsetzung}
Die Android Platform und die mobile Umgebung stellen im Vergleich zu einer normalen Systemumgebung einige besondere Anforderungen an die Eigenschaften einer Anwendung. So kann eine Anwendung nach vorheriger Ankündigung vom System praktisch jederzeit vom System „getötet“ werden. D.h. dass die Anwendung durch Speichern ihres aktuellen Zustand jederzeit wiederhergestellt können muss. Dabei passieren häufig Fehler, so dass dafür viel Zeit beim Testen verwendet werden muss.

Auch andere Eigenschaften, wie der sehr langsame Flash-Speicher erfordern besondere Vorgehensweisen, damit der Anwender keine Nachteile bei der Bedienung hat. Deshalb werden die relevanten Teile des Datenmodells direkt nach einer Änderung asynchron auf den Speicher geschrieben, da es ansonsten zu bemerkbaren Verzögerungen beim Anwendungswechsel kommen würde, wenn man diese erst zum Zeitpunkt des Wechsels speichert.

Jede Android Anwendung ist in Activities gegliedert, die jeweils einer Benutzermaske entsprechen. Beim Wechsel von Activity zu Activity können jedoch keine Java-Objekte direkt weitergeben werden, sondern nur über Java-Serialisierung. Da die Serialisierung des kompletten Datenmodells ($\approx$~2~MB) zu lange dauern würde und statische Klassenvariablen eine unsaubere Lösung wären, wurde ein System entwickelt, das jeweils nur einen eindeutigen Bezeichner an die nächste Activity weitergibt und im Hintergrund die Daten im Speicher vorhält um die Ladezeiten gering zu halten.

Probleme machte auch die langsame Geschwindigkeit beim Parsen der Antworten des Servers. Gerade, wenn ein Wegstück über ganz Deutschland führt kann es ohne Probleme bis zu 8000 Punkte enthalten und mehrere Megabyte groß sein. Nach einem Umstieg vom mitgeliefertem JSON-Parser auf Jackson, war die Verarbeitungszeit allerdings immer noch zu lange, so dass nach vorherigen Benchmarks eine Umstellung des Formats für Koordinaten von Float auf Integer veranlasst wurde. Mit Hilfe der transparenten GZIP-Kompression von HTTP konnte die Geschwindigkeit ebenfalls noch erhöht werden. Da der Server auch die Jackson Bibliothek zum erstellen der Antworten in JSON verwendete, lag es Nahe die binäre Kodierung Smile von Jackson zu verwenden. Mit Hilfe all dieser Optimierungen konnte schließlich eine Geschwindigkeit erreicht werden, die eine interaktive Verwendung für z.B. Drag und Drop von Markierungen auf der Karte ermöglicht.