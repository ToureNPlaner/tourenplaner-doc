\subsection*{Ziel des Projekts}

Ziel des Projektes war die Entwicklung einer Server-Client-Anwendung, die als einfach erweiterbare Plattform dient, um die Eingabe, Verarbeitung und Ausgabe algorithmischer Probleme, die auf Straßendaten und Kartendaten basieren, zu ermöglichen.

\subsection*{Vorgaben}

Vorgaben waren die Verwendung der Programmiersprache Java für den Serverteil und die Erstellung eines Web, sowie eines Android-Clienten und die Implementierung der folgenden Algorithmen: Shortest Path, Constrained Shortest Path, Traveling Salesman Problem. Außerdem sollten beide Teile möglichst einfach um weitere Algorithmen erweiterbar sein. Die aufbereiteten Kartendaten wurden vom Kunden bereitgestellt.

Neben einer Variante, die für den öffentlichen Betrieb ausgelegt sein soll, wurde außerdem eine Variante mit Zugriffskontrolle gefordert bei der man sich als Benutzer anmelden muss und seine alten Abfragen sehen kann.

\subsection*{Organisation}

Das Projekt wurde in zwei Teams aufgeteilt: Das Server-Team bestehend aus Christoph Haag, Sascha Meusel, Niklas Schnelle, Peter Vollmer und das Client-Team, welches später in 2 Teile aufgeteilt wurde: Web-Client (Philipp Gildein, Huy Viet Le, Kevin Wenz) und Android-Client (Vivian Eggert, Steffen Hanikel). Auf Grund der geringen Anzahl an Teilnehmern kam es zu einer Doppelbesetzung von Rollen: Projektleiter: Steffen Hanikel, Stellvertretender Projektleiter: Peter Vollmer, Administrator: Christoph Haag.

\subsection*{Vorgehen}

Nach der ersten Phase, in der mehrere Projektpläne erstellt und anschließend einer ausgewählt wurde, wie es vom Rahmen des Studienprojekts so vorgegeben ist, sind wir auf agiles Vorgehen umgestiegen. So wurden wöchentlich projektinterne Meetings gehalten und monatlich Treffen mit dem Kunden durchgeführt, bei denen der Fortschritt im Projekt und eventuelle Änderungen der Anforderungen besprochenen wurden.

Um eine gute Zusammenarbeit der Komponenten zu gewährleisten wurde auf das Prinzip der kontinuierliche Integration gesetzt, indem alle Komponenten täglich neu gebaut und somit auch von allen getestet werden konnten.
Als Versionskontrollsystem wurde für alle Teams und für die Dokumentation ein separates Git-Archiv verwendet.    Für die Verwaltung von Aufgaben und Protokollen hat sich Redmine als Ticket-System und Wiki im Einsatz bewährt.

\subsection*{Aufbau}
Als Protokoll zwischen den Komponenten wurde HTTP/S mit dem Datenformat JSON wegen der breiten Unterstützung, einfachen Handhabung und guter Erweiterbarkeit verwendet. Um zu verhindern, dass bei dem Hinzufügen von neuen Algorithmen alle Klienten angepasst werden müssen, wurde vor allem Wert auf eine flexible Art der Definition von Parametern für Algorithmen gelegt.

\clearpage