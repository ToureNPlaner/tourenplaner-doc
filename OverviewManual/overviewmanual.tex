\documentclass[titlepage,parskip=true]{scrartcl}
\usepackage[utf8]{inputenc}
\usepackage{array}
\usepackage{wrapfig}
\usepackage{longtable}
\usepackage[unicode=true,pdfusetitle,bookmarks=true,bookmarksnumbered=false,bookmarksopen=false,breaklinks=false,pdfborder={0 0 0},backref=false,colorlinks=false]{hyperref}
\usepackage{listings}
\usepackage{color}
\usepackage{framed}
\definecolor{shadecolor}{gray}{.85}
\usepackage{tabularx}
\usepackage[margin=2.0cm]{geometry}
\usepackage{hyperref}

\lstset{ %
	language=java,                % the language of the code
	basicstyle=\footnotesize,       % the size of the fonts that are used for the code
	numbers=left,                   % where to put the line-numbers
	numberstyle=\footnotesize,      % the size of the fonts that are used for the line-numbers
	stepnumber=1,                   % the step between two line-numbers. If it's 1, each line 
	                                % will be numbered
	numbersep=5pt,                  % how far the line-numbers are from the code
	backgroundcolor=\color{white},  % choose the background color. You must add \usepackage{color}
	showspaces=false,               % show spaces adding particular underscores
	showstringspaces=false,         % underline spaces within strings
	showtabs=false,                 % show tabs within strings adding particular underscores
	frame=single,                   % adds a frame around the code
	tabsize=4,                      % sets default tabsize to 2 spaces
	captionpos=b,                   % sets the caption-position to bottom
	breaklines=true,                % sets automatic line breaking
	breakatwhitespace=true,        % sets if automatic breaks should only happen at whitespace
	title=\lstname,                 % show the filename of files included with \lstinputlisting;
	                                % also try caption instead of title
	escapeinside={\%*}{*)},         % if you want to add a comment within your code
	morekeywords={*,...}            % if you want to add more keywords to the set
}

% adjust table padding
\setlength{\tabcolsep}{8pt} \renewcommand{\arraystretch}{1.5}
 
\title{Overview Manual}
\author{Niklas Schnelle}

%\begin{lstlisting}
%put your code here
%\end{lstlisting}

% Repository Links should always be formatted the same, built a command
\newcommand{\repobox}[2]{
	\begin{shaded}
		\textbf{TheoGit}\\\hspace{5mm}\url{#1}\\
		\textbf{GitHub}\\\hspace{5mm}\url{#2}
	\end{shaded}
}

\begin{document}

\maketitle

\tableofcontents

\pagebreak

\section{Introduction}
The ToureNPlaner projects from the Algorithms Department at the
University of Stuttgart provide a complete infrastructure for researching, developing,
testing and employing routing and mapping algorithms based on OpenStreetMap data.
\section{System Components}
In the following we will describe the original ToureNPlaner subprojects as well
as related projects, which together form a comprehensive testbed for
experimenting and working with OpenStreetMap data.  Each section will describe
the general purpose of its particular subproject and provide links to its source
repository. For more detailed information
the individual projects should provide instructions and documentation
preferably within their README. Additionally \autoref{sec:setup} will provide
a guided tour for setting up a simple ToureNPlaner system.

\subsection{ToureNPlaner Server}
This is the backend server component of the ToureNPlaner system, it provides
in-memory data structures for handling large graphs such as the European road
network and offers an easily extended infrastructure for developing algorithms
using these graphs.
\repobox{https://theogit.fmi.uni-stuttgart.de/schnelns/tourenplaner-server}{https://github.com/ToureNPlaner/tourenplaner-server}
\subsection{ToureNPlaner Web Client}
This subproject provides a web interface for street network routing. It is
statically hosted and talks to the ToureNPlaner Server directly using
JavaScript initiated requests.
\repobox{https://theogit.fmi.uni-stuttgart.de/schnelns/tourenplaner-web}{https://github.com/ToureNPlaner/tourenplaner-web}
\subsection{ToureNPlaner Android Client}
The ToureNPlaner Android client provides mobile access to the ToureNPlaner
servers algorithms and also implements the DORC algorithm for running parts of
the routing query on the mobile client itself.
\repobox{https://theogit.fmi.uni-stuttgart.de/schnelns/tourenplaner-android}{https://github.com/ToureNPlaner/tourenplaner-android}
\subsection{Java CHConstructor}
This subproject provides a simple CH preprocessing tool written in Java. Its
performance while in the same ballpark may be slower than the C++ based version
which is also due to the fact that it skips some special preprocessing steps in
favor of a simpler approach with fewer special cases. On the other hand this
makes the resulting CHs better suited for rendering with the URAR algorithms
although as of now no special provisions for rendering quality have been
implemented.
\repobox{https://theogit.fmi.uni-stuttgart.de/schnelns/java-chconstructor}{https://github.com/ToureNPlaner/CHConstructor}
\subsection{C++ CHConstructor}
Playing the same role as the Java based CHConstructor this version is written
in modern C++1x and aims for maximum performance while using very flexible
template based graph data structures and algorithms. In turn its codebase may
be harder to understand especially if you aren't well versed in modern C++
techniques.
\repobox{https://theogit.fmi.uni-stuttgart.de/nusserae/chconstructor}{https://github.com/chaot4/ch\_constructor}
\subsection{Java Rendering Client}
The Java based rendering client implements the URAR algorithm for CH based map
rendering using a Mercator projection and also supports routing. It is designed
mainly as a testbed with easy to change algorithm details and simple drawing
routines. Due to the use of Java AWT based drawing without hardware
acceleration drawing performance is limited and can be slow at times.
\repobox{https://theogit.fmi.uni-stuttgart.de/schnelns/chrenderclient}{https://github.com/ToureNPlaner/chrenderclient}
\subsection{C++ Rendering Client}
The C++ and OpenGL based rendering client has much better rendering performance
employing hardware acceleration and drawing on a 3D globe. However at the
moment there is no support for routing and due to the unavailability of a SMILE
(binary JSON) library for C++ it uses simple gzipped JSON for its requests,
resulting in slightly larger bundles and more data usage.
\repobox{https://theogit.fmi.uni-stuttgart.de/schnelns/chrenderopengl}{https://github.com/ToureNPlaner/chrenderopengl}
\subsection{OsmGraphCreator}
This subproject provides a tool for converting OpenStreetMap data in the
protocol buffer based format to the text formats understood by most other
subprojects.
\repobox{https://theogit.fmi.uni-stuttgart.de/bahrdtdl/osmgraphcreator}{https://github.com/dbahrdt/OsmGraphCreator}
\subsection{SCCExtractor}
This subproject provides a tool for extracting the largest strongly connected
component from a graph. This is useful for creating useful road network graphs
that still present a single strongly connected component. Many tools also
accept graphs composed of several components though.
\repobox{https://theogit.fmi.uni-stuttgart.de/nusserae/scc\_extractor}{https://github.com/chaot4/scc\_extractor}
\section{Guided Example Setup}
In this section we will setup a minimal ToureNPlaner system. It is assumed that
you are running a Unix-like operating system though similar steps should work
on Windows too, note however that most of these projects have only be tested on
Linux, though *BSD or MacOS X should work too.
\label{sec:setup}
\subsection{Initial Setup}
For this minimal local setup we will install everything into a single working
directory, for production setups we recommend installing to meaningful
locations and running the server as a service e.g. by using systemd. As an
example the ToureNPlaner Server includes a PKGBUILD for building an Arch Linux
package.
\subsection{Getting Graphs}
In this section we will describe how to get from openly available and updated
OpenStreetMap data to graphs in a format useable by the ToureNPlaner system.
Alternatively if you are working at the FMI graphs in the "fmitext" format may
already be available.
\subsubsection{Downloading Current OpenStreetMap Graphs}
In this example we will be using the road network of Baden-Württemberg in the
Protocol Buffer format (.pbf), available here:
\url{http://download.geofabrik.de/europe/germany.html}
\begin{lstlisting}[language=bash]
mkdir graphs
wget http://download.geofabrik.de/europe/germany/baden-wuerttemberg-latest.osm.pbf -P graphs
\end{lstlisting}
This should give us the latest OpenStreetMap data for Baden-Württemberg in
"graphs/baden-wuerttemberg-latest.osm.pbf".
\subsubsection{Converting to the FMI Graph Format}
To convert the graph to the format used by our tools we will first need to get
the OsmGraphCreator. This needs as dependencies development files for cmake,
zlib and Google's protobuf which hopefully should be available as packages in
most Linux distributions.
\begin{lstlisting}[language=bash]
git clone --recursive git@github.com:dbahrdt/OsmGraphCreator.git
cd OsmGraphCreator
mkdir build
cd build
cmake ../
make -j
cd creator
\end{lstlisting}
If everything went well you should find the OsmGraphCreator exectuable in
"OsmGraphCreator/build/creator" if errors occured you might be missing
dependencies especially the Google protobuf development files.

Now lets convert the previously downloaded Baden-Württemberg graph to fmitext format.
We also add the -s option here to sort by source-/target node ids.
\begin{lstlisting}[language=bash]
./creator -s -g fmimaxspeedtext -t distance -c ../../data/configs/car.cfg -o ../../../graphs/baden-wuerttemberg.txt ../../../graphs/baden-wuerttemberg-latest.osm.pbf
\end{lstlisting}
If sucessful this will create a graph file in the fmitext format in "graphs/baden-wuerttemberg.txt"
\subsubsection{Reading Graph Metadata}
One interesting feature of our graph format we can now try out is the use of a metadata header.
This header is easily readable as text at the start of both binary and textual graphs in the
FMI standard format using 
\begin{lstlisting}[language=bash]
head -n 50 <graphfile>
\end{lstlisting}
and has a format similar to the following example, which comes from a more
complex CH graph.  This graph has been preprocessed by several tools each
adding its own metadata headers while preserving all other headers by
prepending "Origin".
\begin{verbatim}
# Type : chgraph
# Id : 6e4e9a5ea33d8d67c4652bc14d6c8850
# Revision : 1
# Timestamp : 1453237855
# Origin : ch_constructor
# OriginId : a173bf3f27b0a874b1543f136ceadcba
# OriginOrigin : scc_extractor
# OriginOriginId : 0
# OriginOriginRevision : 1
# OriginOriginTimestamp : 1453058797
# OriginOriginType : maxspeed
# OriginRevision : 1
# OriginTimestamp : 1453058920
# OriginType : graph

3434345
12751981
0 163354 48.6674214 9.2445576 0 2
1 163355 48.6694744 9.2432712 0 0
2 163358 48.6661923 9.2515362 0 0
\end{verbatim}
The timestamps are in seconds since the Unix epoch and can be converted using
the following date command.
\begin{lstlisting}[language=bash]
date -d @<numeric_value>
\end{lstlisting}
for example the following will output "Tue Jan 19 22:10:55 CET 2016"
\begin{lstlisting}[language=bash]
date -d @1453237855
\end{lstlisting}
\subsubsection{Extracting the Largest Strongly Connected Component (Optional)}
Now we will extract the largest strongly connected component of the graph so
that all points are connected. Technically the ToureNPlaner Server works with
more than one strongly connected component too.
\begin{lstlisting}[language=bash]
git clone git@github.com:chaot4/scc_extractor.git
cd scc_extractor
./create-build.sh
cd build
\end{lstlisting}
This should compile and test the scc\_extractor, if everything went well we can
use it like so
\begin{lstlisting}[language=bash]
./scc_extractor -i ../../graphs/baden-wuerttemberg.txt -f FMI_DIST -o ../../graphs/baden-wuerttemberg-scc.txt
\end{lstlisting}
\subsection{Computing a Contraction Hierarchy}
Until now we have dealt with a direct graph representation of the road network.
We could now simply run Dijkstra's algorithm on these graphs and compute
shortest paths with it, however every computation would take on the order of
seconds. So as to make things scale to a point where a single server can handle
many shortest path computations we need a speed-up scheme.  For this we will
use Contraction Hierarchies to augment our graphs with shortcuts which allow us
to speed-up shortest path computations by several orders of magnitude. This
preprocessing is what tbe CHConstructor subprojects implement.
\subsubsection{Using the Java based CH Constructor}
As the first of the two CH Constructors compatible with the ToureNPlaner framework
we will look at the Java based variant. Building it is easy though maven needs and the
Java JDK need to be installed.
\begin{lstlisting}[language=bash]
git clone https://github.com/ToureNPlaner/CHConstructor.git
cd CHConstructor
./build.sh
\end{lstlisting}
Now we should have built the CH constructor and should be able to use it to
preprocess the graph created during the previous sections.
\begin{lstlisting}[language=bash]
java -jar target/chconstructor-0.1-SNAPSHOT-jar-with-dependencies.jar -if standard -i ../graphs/baden-wuerttemberg-scc.txt -of standard -o ../graphs/baden-wuerttemberg-ch.txt
\end{lstlisting}
\subsubsection{Using the C++ based CH Constructor}
Alternatively to the previous section you may also use the other CH Constructor written in C++.
\begin{lstlisting}[language=bash]
git clone https://github.com/chaot4/ch_constructor.git
cd ch_constructor
./create-build.sh
cd build
\end{lstlisting}
This should give us an executable called "ch\_constructor" we can use to compute
a Contraction Hierarchy.
\begin{lstlisting}[language=bash]
./ch_constructor -i ../../graphs/baden-wuerttemberg-scc.txt -f FMI_EUCL -o ../../graphs/baden-wuerttemberg-ch2.txt -g FMI_EUCL_CH
\end{lstlisting}
\subsection{Running the ToureNPlaner Server}
Now that we have preprocessed our graph we can build and run the ToureNPlaner Server
\begin{lstlisting}[language=bash]
git clone https://github.com/ToureNPlaner/tourenplaner-server.git
cd tourenplaner-server 
./build.sh jar
\end{lstlisting}
This will create a jar file with dependencies included in the target directory. To run it
wir the previously created graph we need to insert its path in the configuration file by changing
the graphfile line to
\begin{verbatim}
"graphfilepath" : "../graphs/baden-wuerttemberg-ch.txt",
\end{verbatim}
Optionally we may also choose a higher count of threads used for computations
in the "threads" field.  Finally we can run the ToureNPlaner Server
\begin{lstlisting}[language=bash]
java -jar target/tourenplaner-server-1.0-SNAPSHOT-jar-with-dependencies.jar -c tourenplaner.conf
\end{lstlisting}
\end{document}
