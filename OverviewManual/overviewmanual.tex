\documentclass[titlepage,parskip=true]{scrartcl}
\usepackage[utf8]{inputenc}
\usepackage{array}
\usepackage{wrapfig}
\usepackage{longtable}
\usepackage[unicode=true,pdfusetitle,bookmarks=true,bookmarksnumbered=false,bookmarksopen=false,breaklinks=false,pdfborder={0 0 0},backref=false,colorlinks=false]{hyperref}
\usepackage{listings}
\usepackage{color}
\usepackage{framed}
\definecolor{shadecolor}{gray}{.85}
\usepackage{tabularx}
\usepackage{hyperref}

\lstset{ %
	language=java,                % the language of the code
	basicstyle=\footnotesize,       % the size of the fonts that are used for the code
	numbers=left,                   % where to put the line-numbers
	numberstyle=\footnotesize,      % the size of the fonts that are used for the line-numbers
	stepnumber=2,                   % the step between two line-numbers. If it's 1, each line 
	                                % will be numbered
	numbersep=5pt,                  % how far the line-numbers are from the code
	backgroundcolor=\color{white},  % choose the background color. You must add \usepackage{color}
	showspaces=false,               % show spaces adding particular underscores
	showstringspaces=false,         % underline spaces within strings
	showtabs=false,                 % show tabs within strings adding particular underscores
	frame=single,                   % adds a frame around the code
	tabsize=4,                      % sets default tabsize to 2 spaces
	captionpos=b,                   % sets the caption-position to bottom
	breaklines=true,                % sets automatic line breaking
	breakatwhitespace=false,        % sets if automatic breaks should only happen at whitespace
	title=\lstname,                 % show the filename of files included with \lstinputlisting;
	                                % also try caption instead of title
	escapeinside={\%*}{*)},         % if you want to add a comment within your code
	morekeywords={*,...}            % if you want to add more keywords to the set
}

% adjust table padding
\setlength{\tabcolsep}{8pt} \renewcommand{\arraystretch}{1.5}
 
\title{Overview Manual}
\author{Niklas Schnelle}

%\begin{lstlisting}
%put your code here
%\end{lstlisting}

% Repository Links should always be formatted the same, built a command
\newcommand{\repobox}[2]{
	\begin{shaded}
		\textbf{TheoGit}\\\hspace{5mm}\url{#1}\\
		\textbf{GitHub}\\\hspace{5mm}\url{#2}
	\end{shaded}
}

\begin{document}

\maketitle

\tableofcontents

\pagebreak

\section{Introduction}
The ToureNPlaner projects from the Algorithms Department at the
University of Stuttgart provide a complete infrastructure for researching, developing,
testing and employing routing and mapping algorithms based on OpenStreetMap data.
\section{System Components}
In the following we will describe the original ToureNPlaner subprojects as well
as related projects, which together form a comprehensive testbed for
experimenting and working with OpenStreetMap data.  Each section will describe
the general purpose of its particular subproject and provide links to its source
repository. For more detailed information
the individual projects should provide instructions and documentation
preferably within their README. Additionally \autoref{sec:setup} will provide
a guided tour for setting up a simple ToureNPlaner system.

\subsection{ToureNPlaner Server}
This is the backend server component of the ToureNPlaner system, it provides
in-memory data structures for handling large graphs such as the European road
network and offers an easily extended infrastructure for developing algorithms
using these graphs.
\repobox{https://theogit.fmi.uni-stuttgart.de/schnelns/tourenplaner-server}{https://github.com/ToureNPlaner/tourenplaner-server}
\subsection{ToureNPlaner Web Client}
This subproject provides a web interface for street network routing. It is
statically hosted and talks to the ToureNPlaner Server directly using
JavaScript initiated requests.
\repobox{https://theogit.fmi.uni-stuttgart.de/schnelns/tourenplaner-web}{https://github.com/ToureNPlaner/tourenplaner-web}
\subsection{ToureNPlaner Android Client}
The ToureNPlaner Android client provides mobile access to the ToureNPlaner
servers algorithms and also implements the DORC algorithm for running parts of
the routing query on the mobile client itself.
\repobox{https://theogit.fmi.uni-stuttgart.de/schnelns/tourenplaner-android}{https://github.com/ToureNPlaner/tourenplaner-android}
\subsection{Java CHConstructor}
This subproject provides a simple CH preprocessing tool written in Java. Its
performance while in the same ballpark may be slower than the C++ based version
which is also due to the fact that it skips some special preprocessing steps in
favor of a simpler approach with fewer special cases. On the other hand this
makes the resulting CHs better suited for rendering with the URAR algorithms
although as of now no special provisions for rendering quality have been
implemented.
\repobox{https://theogit.fmi.uni-stuttgart.de/schnelns/java-chconstructor}{https://github.com/ToureNPlaner/CHConstructor}
\subsection{C++ CHConstructor}
Playing the same role as the Java based CHConstructor this version is written
in modern C++1x and aims for maximum performance while using very flexible
template based graph data structures and algorithms. In turn its codebase may
be harder to understand especially if you aren't well versed in modern C++
techniques.
\repobox{https://theogit.fmi.uni-stuttgart.de/nusserae/chconstructor}{https://github.com/chaot4/ch\_constructor}
\subsection{Java Rendering Client}
The Java based rendering client implements the URAR algorithm for CH based map
rendering using a Mercator projection and also supports routing. It is designed
mainly as a testbed with easy to change algorithm details and simple drawing
routines. Due to the use of Java AWT based drawing without hardware
acceleration drawing performance is limited and can be slow at times.
\repobox{https://theogit.fmi.uni-stuttgart.de/schnelns/chrenderclient}{https://github.com/ToureNPlaner/chrenderclient}
\subsection{C++ Rendering Client}
The C++ and OpenGL based rendering client has much better rendering performance
employing hardware acceleration and drawing on a 3D globe. However at the
moment there is no support for routing and due to the unavailability of a SMILE
(binary JSON) library for C++ it uses simple gzipped JSON for its requests,
resulting in slightly larger bundles and more data usage.
\repobox{https://theogit.fmi.uni-stuttgart.de/schnelns/chrenderopengl}{https://github.com/ToureNPlaner/chrenderopengl}
\subsection{OsmGraphCreator}
This subproject provides a tool for converting OpenStreetMap data in the
protocol buffer based format to the text formats understood by most other
subprojects.
\repobox{https://theogit.fmi.uni-stuttgart.de/bahrdtdl/osmgraphcreator}{https://github.com/dbahrdt/OsmGraphCreator}
\subsection{SCCExtractor}
This subproject provides a tool for extracting the largest strongly connected
component from a graph. This is useful for creating useful road network graphs
that still present a single strongly connected component. Many tools also
accept graphs composed of several components though.
\repobox{https://theogit.fmi.uni-stuttgart.de/nusserae/scc\_extractor}{https://github.com/chaot4/scc\_extractor}
\section{Guided Example Setup}
\label{sec:setup}
\end{document}
