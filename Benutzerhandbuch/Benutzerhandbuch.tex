\documentclass[ngerman,titlepage,parskip=true]{scrartcl}
\usepackage[utf8]{inputenc}
\usepackage{array}
\usepackage{babel}
\usepackage{wrapfig}
\usepackage{longtable}
\usepackage[unicode=true,pdfusetitle,bookmarks=true,bookmarksnumbered=false,bookmarksopen=false,breaklinks=false,pdfborder={0 0 0},backref=false,colorlinks=false]{hyperref}
\usepackage{listings}
\usepackage{color}
\usepackage{tabularx}

\lstset{ %
	language=bash,                % the language of the code
	basicstyle=\footnotesize,       % the size of the fonts that are used for the code
	numbers=left,                   % where to put the line-numbers
	numberstyle=\footnotesize,      % the size of the fonts that are used for the line-numbers
	stepnumber=2,                   % the step between two line-numbers. If it's 1, each line 
	                                % will be numbered
	numbersep=5pt,                  % how far the line-numbers are from the code
	backgroundcolor=\color{white},  % choose the background color. You must add \usepackage{color}
	showspaces=false,               % show spaces adding particular underscores
	showstringspaces=false,         % underline spaces within strings
	showtabs=false,                 % show tabs within strings adding particular underscores
	frame=single,                   % adds a frame around the code
	tabsize=4,                      % sets default tabsize to 2 spaces
	captionpos=b,                   % sets the caption-position to bottom
	breaklines=true,                % sets automatic line breaking
	breakatwhitespace=false,        % sets if automatic breaks should only happen at whitespace
	title=\lstname,                 % show the filename of files included with \lstinputlisting;
	                                % also try caption instead of title
	escapeinside={\%*}{*)},         % if you want to add a comment within your code
	morekeywords={*,...}            % if you want to add more keywords to the set
}

% adjust table padding
\setlength{\tabcolsep}{8pt} \renewcommand{\arraystretch}{1.5}
 
\title{Benutzerhandbuch}
\author{ToureNPlaner Team}


\newenvironment{owncompactitem}{%
\compactitem
}{%
\@finalstrut\@arstrutbox
\@nameuse{endcompactitem}%
\aftergroup\let\aftergroup\@finalstrut\aftergroup\@gobble
}
\newenvironment{owncompactenum}{%
\compactenum
}{%
\@finalstrut\@arstrutbox
\@nameuse{endcompactenum}%
\aftergroup\let\aftergroup\@finalstrut\aftergroup\@gobble
}
\makeatother

\newcommand{\configoption}[4]
{%\paragraph{#1}
\setlength{\extrarowheight}{2pt}
\begin{tabular}{|p{0.2\textwidth}|p{0.9\textwidth}|}
\hline
  \textbf{Name} & \textbf{#1}\\\hline
  Beispiel & #2\\\hline
  Typ & #3\\\hline
  Beschreibung & #4\\\hline
\end{tabular}
}

\begin{document}

\maketitle

\tableofcontents

\pagebreak

\section{Einleitung}

	\subsection{\"Uber ToureNPlaner}
   ToureNPlaner ist eine Server-Anwendung, die folgende Algorithmen implementiert:
   \begin{itemize}
     \item Shortest Path (mit Contraction Hierarchy)
     \item Constrained Shortest Path
     \item Travelling Sales Person
   \end{itemize}
   ToureNPlaner hat zwei Betriebsmodi:
   \begin{itemize}
     \item Public - Es sind keine Anmeldedaten n\"otig, um auf dem Server Berechnungen durchzuf\"uhren.
     \item Private - Es ist eine Registrierung erforderlich, um auf dem Server Berechnungen durchzuf\"uhren. Die Verbindung wird mit SSL verschl\"usselt. Berechnungen und die Ergebnisse werden in einer Datenbank gespeichert.
   \end{itemize}


\section{Generelles}
	\subsection{Systemvoraussetzungen}
	
	\begin{itemize}
		\item JVM in Version 1.6 (oder h\"oher)
		\item 3 Gigabyte Arbeitsspeicher f\"ur den Graph + TODO Gigabyte pro Thread
		\item optional eine MySQL-Datenbank f\"ur den Betrieb im Private-Modus
	\end{itemize}

	\label{usedfiles}
  \subsection{Von ToureNPlaner benutzte Dateien}
	 \begin{itemize}
	   \item Der ToureNPlaner-Server selbst besteht aus dem ausf\"uhrbahren .jar-Archiv tourenplaner-server.jar.
	   \item Konfigurationsdatei tourenplaner.conf im json-Format, deren Pfad mit ``-c /Pfad/zur/Konfigurationsdatei'' angegeben werden kann
	   \item Stra\ss{}engraph in einem Speziellen Textformat (siehe Graphspezifikation), dessen Pfad in der Konfigurationsdatei angegeben wird
		\begin{itemize}
		  \item optional schneller einlesbarer Bin\"ardump des Stra\ss{}engraphen, der von ToureNPlaner erstellt und benutzt wird, wenn ``-f dump'' angegeben wird
		\end{itemize}
	   \item optional ein Java Keystore, der ein Zertifikat f\"ur den SSL-verschl\"usselten Betrieb enth\"alt
	 \end{itemize}

\section{Installation}
  Pakete f\"ur Folgende Distributionen werden angeboten:
  \begin{itemize}
    \item Archlinux
  \end{itemize}
  \subsection{Dateien im Archlinux-Paket}
	\begin{itemize}
	  \item /etc/rc.d/tourenplanerd : Script zum Starten und Stoppen des ToureNPlaner Servers als Daemon
	  \item /etc/tourenplaner.conf : Konfigurationsdatei
	  \item /usr/bin/tourenplaner-server : minimales Script, zum direkten Starten des ToureNPlaner Servers
	  \item /usr/share/java/tourenplaner/tourenplaner-server.jar : der eigentliche Server
	  \item Script zum Einrichten der Datenbank f\"ur den Betrieb im Private-Modus
		\begin{itemize}
		  \item /usr/share/tourenplaner/db\_init\_script.sql
		  \item /usr/share/tourenplaner/db\_init.sh
		\end{itemize}
	\end{itemize}

	\subsection{weitere Installations-Schritte (Public-Modus und Private-Modus)}
	\subsubsection{Stra\ss{}engraph}
	 In der Standardkonfiguration muss der Stra\ss{}engraph nach /var/lib/tourenplaner/germany.txt.
	\subsection{weitere Installations-Schritte (nur Private-Modus)}
	 \subsubsection{Java Keystore}
	 F\"ur den Private-Betrieb muss der Java Keystore nach /var/lib/tourenplaner/keystore.jks kopiert werden.
	 \paragraph{Erstellen eines Java Keystore}
	%TODO erstellen des java keystore
	\subsubsection{Einrichten der Datenbank}
	und die Datenbank mit dem Initialisierungs-Script eingerichtet werden (daf\"ur wird das MySQL root-Passwort ben\"otigt):
   \begin{lstlisting}
	 cd /usr/share/tourenplaner
	 ./db_init.sh
	\end{lstlisting}
	
  \subsection{Manuelle Installation}
	Es werden die im Abschnitt \hyperref[usedfiles]{Von ToureNPlaner verwendeten Dateien} ben\"otigt.\\
	In der Datei \textit{tourenplaner.conf} werden die entsprechenden Pfade zu den Dateien eingestellt. Das Verzeichnis, in dem der Stra\ss{}engraph liegt, muss f\"ur den Benutzer, der ToureNPlaner ausf\"uhrt schreibbar sein, wenn der ToureNPlaner Server einen Bin\"ardump erstellen soll
	\subsubsection{Manuelles Starten}
  Gestartet wird der Server folgenderma\ss{}en:
	\begin{lstlisting}
  "${JAVA_HOME}/bin/java" -Xmx12g -Xincgc -jar /pfad/zum/server/tourenplaner-server.jar -c /pfad/zur/config/tourenplaner.conf -f dump
	\end{lstlisting}
	Hierbei gibt \textit{-Xmx12g} die maximale RAM-Nutzung des Servers an. Beendet sich der Server mit einer OutOfMemoryException, sollte dieser Wert erh\"oht werden.
	Um den Server unter einem anderen Benutzername zu starten, kann \textit{sudo} verwendet werden:
	\begin{lstlisting}
	   sudo -E -u http "${JAVA_HOME}/bin/java" -Xmx12g -Xincgc -jar /pfad/zum/server/tourenplaner-server.jar -c /pfad/zur/config/tourenplaner.conf -f dump
	\end{lstlisting}
  Hierbei steht \textit{-E} f\"ur das beibehalten der Environment Variablen des Ausf\"uhrenden, also z.B. \$JAVA\_HOME


	\subsubsection{Command Line Switches}
	Der ToureNPlaner Server unterst\"utzt die folgenden Kommandozeilenoptionen:
	\begin{itemize}
	  \item -c \textit{Pfad} : Pfad zur Konfigurationsdatei (wenn nicht angegeben, wird im Server eingebaute Standardkonfiguration wird verwendet)
	  \item -f text / -f dump : Liest den Graph aus einer Textdatei oder aus einem Bin\"ardump mit dem selben Namen des Textgraphen mit der zus\"atzlichen Erweiterung \textit{.dat} (Ist mit -f dump kein Bin\"ardump vorhanden, wird \textit{-f text} angenommen und ein Bin\"ardump erstellt) (Wenn nicht angegeben, wird \textit{-f text} angenommen)
	 \item dumpgraph : Erstellt nur den Bin\"ardump des Graphen, ohne den Server zu starten
	\end{itemize}

\section{Konfiguration}
Die Konfigurationsdatei (/etc/tourenplaner.conf) enth\"alt folgende Optionen im JSON-Format nach folgendem Schema:
\begin{lstlisting}[caption=Konfigurations-Schema,float]
  {
	 "A_String" = "text",
	 // A Comment
	 "A_Boolean" = true, // another comment
	 "An_Integer" = 3
  }
\end{lstlisting}

Fehlende Optionen werden durch im Server eingebaute Standardoptionen ersetzt.

%\configoption
%{}
%{}
%{}
%{}

\subsection{Empfohlene Optionen}

\subsubsection{Public und Privater Server}
\configoption
{threads}
{2}
{integer}
{Anzahl der gleichzeitig ausf\"uhrbahren Algorithmen. Ein guter Wert ist die Anzahl der zu verwendenden CPU-Kerne - 1 (Reserve f\"ur den Garbage-Collector)}

\configoption
{queuelength}
{20}
{integer}
{Die Anzahl der Requests, die warten sollen, sollten alle Berechnungsthreads belegt sein. Requests, die auch nicht mehr in die Warteschlange passen, werden direkt abgewiesen}

\configoption
{graphfilepath}
{``/var/tourenplaner/germany.txt''}
{string}
{Pfad zum Stra\ss{}engraphen}

\configoption
{private}
{true}
{boolean}
{F\"ur Berechnungen auf einem Public Server ist keine Authentifizierung notwendig. Ein Private-Server hat eine Benutzerverwaltung und bietet seinen Service \"uber HTTPS an}

\configoption
{httppport}
{8080}
{integer}
{Port, auf dem der Server via HTTP lauscht}

\configoption
{costpertimeunit}
{10}
{integer}
{Gibt die Kosten (ohne spezifizierte W\"ahrung) f\"ur eine Zeiteinheit an, deren L\"ange durch die \textit{timeunitsize} angegen wird}

\configoption
{timeunitsize}
{1000}
{integer}
{Gibt an, wie lange eine Zeiteinheit f\"ur die Abrechnung sein soll. Die Basiseinheit sind Millisekunden. Eine Angabe von 1000 bedeutet, dass der Preis bei \textit{costpertimeunit} f\"ur eine Sekunde gilt}

\subsubsection{Nur Privater Server}

\configoption
{dburi}
{``jdbc:mysql://localhost:3306/tourenplaner?autoReconnect=true''}
{string}
{Die URI zur MySQL Datenbank in Form einer von JAVA verarbeitbaren URI mit der Datenbankangabe (z.B ``/tourenplaner'')}

\configoption
{dbuser}
{``tourenplaner''}
{string}
{MySQL Benutzername}

\configoption
{dbpw}
{toureNPlaner}
{string}
{MySQL Passwort f\"ur den angegebenen Benutzer}

\configoption
{sslport}
{8081}
{integer}
{Port, auf dem der Server via HTTPS lauscht}

\configoption
{sslcert}
{``/var/tourenplaner/keystore.jks''}
{string}
{Pfad zum Java Keystore, der das SSL-Zertifikat enth\"alt}

\configoption
{sslalias}
{tourenplaner}
{string}
{Bezeichnung des SSL-Zertifikates im Java Keystore}

\configoption
{sslpw}
{``tour3nplan3er''}
{string}
{Passwort f\"ur den Java Keystore}

\subsection{Optionale Optionen}

\configoption
{serverinfosslport}
{443}
{integer}
{Der bei serverinfosslport angegebene Port steht zwar in der Serverinfo, die der Server auf /info antwortet, in Wirklichkeit lauscht der Server aber auf dem Port, der mit sslport angegeben wurde}

\configoption
{maxdbtries}
{3}
{integer}
{Gibt an, wie oft der Server versucht, nach einem Verbindungsabbruch, die Verbindung zur Datenbank wiederherzustellen}

\subsection{nginx als Proxy f\"ur ToureNPlaner}

Prinzipiell kann jeder Webserver, der die ``Reverse-Proxy''-Technik unterst\"utzt, verwendet werden, um Anfragen zu ToureNPlaner weiterzuleiten. Praktisch ist nur nginx getestet.\\

Folgende ``locations'' werden dazu in die nginx-Konfiguration eingetragen:
\lstinputlisting[caption=nginx location Konfiguration]{nginx-proxy-location.txt}
Im Falle eines ``public''-Servers sind diese Locations f\"ur den http-Port einzutragen, im Falle eines ``private''-Servers sind die Locations f\"ur den HTTPS-Port einzutragen sowie nur die ``/info''-Location f\"ur den HTTP Port einzutragen.

F\"ur den Android Client muss noch das ChunkinModule f\"ur nginx installiert werden und die entsprechende Konfiguration vorgenommen werden:
\url{http://wiki.nginx.org/HttpChunkinModule}

\end{document}