\documentclass[ngerman,titlepage,parskip=true]{scrartcl}
\usepackage[utf8]{inputenc}
\usepackage{array}
\usepackage{babel}
\usepackage{wrapfig}
\usepackage{longtable}
\usepackage[unicode=true,pdfusetitle,bookmarks=true,bookmarksnumbered=false,bookmarksopen=false,breaklinks=false,pdfborder={0 0 0},backref=false,colorlinks=false]{hyperref}
\usepackage{listings}
\usepackage{color}

\lstset{ %
	language=bash,                % the language of the code
	basicstyle=\footnotesize,       % the size of the fonts that are used for the code
	numbers=left,                   % where to put the line-numbers
	numberstyle=\footnotesize,      % the size of the fonts that are used for the line-numbers
	stepnumber=2,                   % the step between two line-numbers. If it's 1, each line 
	                                % will be numbered
	numbersep=5pt,                  % how far the line-numbers are from the code
	backgroundcolor=\color{white},  % choose the background color. You must add \usepackage{color}
	showspaces=false,               % show spaces adding particular underscores
	showstringspaces=false,         % underline spaces within strings
	showtabs=false,                 % show tabs within strings adding particular underscores
	frame=single,                   % adds a frame around the code
	tabsize=4,                      % sets default tabsize to 2 spaces
	captionpos=b,                   % sets the caption-position to bottom
	breaklines=true,                % sets automatic line breaking
	breakatwhitespace=false,        % sets if automatic breaks should only happen at whitespace
	title=\lstname,                 % show the filename of files included with \lstinputlisting;
	                                % also try caption instead of title
	escapeinside={\%*}{*)},         % if you want to add a comment within your code
	morekeywords={*,...}            % if you want to add more keywords to the set
}

% adjust table padding
\setlength{\tabcolsep}{8pt} \renewcommand{\arraystretch}{1.5}
 
\title{Benutzerhandbuch}
\author{ToureNPlaner Team}

\begin{document}

\maketitle

\tableofcontents

\pagebreak

\section{Einleitung}

	\subsection{\"Uber ToureNPlaner}
   ToureNPlaner ist eine Server-Anwendung, die folgende Algorithmen implementiert:\\
   \begin{itemize}
     \item Shortest Path (mit Contraction Hierarchy)
     \item Constrained Shortest Path
     \item Travelling Sales Person
   \end{itemize}
   ToureNPlaner hat zwei Betriebsmodi:
   \begin{itemize}
     \item Public - Es sind keine Anmeldedaten n\"otig, um auf dem Server Berechnungen durchzuf\"uhren.
     \item Private - Es ist eine Registrierung erforderlich, um auf dem Server Berechnungen durchzuf\"uhren. Die Verbindung wird mit SSL verschl\"usselt.
   \end{itemize}


\section{Generelles}
	\subsection{Systemvoraussetzungen}
	
	\begin{itemize}
		\item JVM in Version 1.6 oder h\"oher
		\item 3 Gigabyte Arbeitsspeicher f\"ur den Graph + TODO Gigabyte pro Thread
		\item optional eine MySQL-Datenbank f\"ur den Betrieb im Private-Modus
	\end{itemize}

  \subsection{Von ToureNPlaner benutzte Dateien}
	 \begin{itemize}
	   \item Der ToureNPlaner-Server selbst besteht aus dem ausf\"uhrbahren .jar-Archiv tourenplaner-server.jar.
	   \item Konfigurationsdatei tourenplaner.conf im json-Format, deren Pfad mit ``-c /Pfad/zur/Konfigurationsdatei'' angegeben werden kann
	   \item Stra\ss{}engraph in einem Speziellen Textformat (siehe Graphspezifikation), dessen Pfad in der Konfigurationsdatei angegeben wird
	   \subitem optional schneller einlesbarer Bin\"ardump des Stra\ss{}engraphen, der von ToureNPlaner erstellt und benutzt wird, wenn ``-f dump'' angegeben wird
	   \item optional ein Java Keystore, der ein Zertifikat f\"ur den SSL-verschl\"usselten Betrieb enth\"alt
	 \end{itemize}

\section{Installation}
  Pakete f\"ur Folgende Distributionen werden angeboten:
  \begin{itemize}
    \item Archlinux
  \end{itemize}
  \subsection{Dateien im Archlinux-Paket}
	\begin{itemize}
	  \item /etc/rc.d/tourenplanerd - Script zum Starten und Stoppen des ToureNPlaner Servers als Daemon
	  \item /etc/tourenplaner.conf - Konfigurationsdatei
	  \item /usr/bin/tourenplaner-server minimales Script, zum direkten Starten des ToureNPlaner Servers
	  \item /usr/share/java/tourenplaner/tourenplaner-server.jar - der eigentliche Server
	  \item Script zum Einrichten der Datenbank f\"ur den Betrieb im Private-Modus
	  \subitem /usr/share/tourenplaner/db_init_script
	  \subitem /usr/share/tourenplaner/db_init.sh
	\end{itemize}

	\subsection{weitere Installations-Schritte}
	 In der Standardkonfiguration muss der Stra\ss{}engraph nach /var/tourenplaner/germany.txt und der Java Keystore nach /var/tourenplaner/keystore.jks kopiert werden.\\
	 F\"ur den Private-Betrieb muss die Datenbank mit dem Initialisierungs-Script eingerichtet werden (daf\"ur wird das MySQL root-Passwort ben\"otigt):
   \begin{lstlisting}
	 cd /usr/share/tourenplaner
	 ./db_init.sh
	\end{lstlisting}

\section{Konfiguration}

\section{Inbetriebnahme}

\end{document}