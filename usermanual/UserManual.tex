\documentclass[ngerman,titlepage,parskip=true]{scrartcl}
\usepackage[utf8]{inputenc}
\usepackage{array}
\usepackage{babel}
\usepackage{wrapfig}
\usepackage{longtable}
\usepackage[unicode=true,pdfusetitle,bookmarks=true,bookmarksnumbered=false,bookmarksopen=false,breaklinks=false,pdfborder={0 0 0},backref=false,colorlinks=false]{hyperref}
\usepackage{listings}
\usepackage{color}
\usepackage{tabularx}

\lstset{ %
	language=bash,                % the language of the code
	basicstyle=\footnotesize,       % the size of the fonts that are used for the code
	numbers=left,                   % where to put the line-numbers
	numberstyle=\footnotesize,      % the size of the fonts that are used for the line-numbers
	stepnumber=2,                   % the step between two line-numbers. If it's 1, each line 
	                                % will be numbered
	numbersep=5pt,                  % how far the line-numbers are from the code
	backgroundcolor=\color{white},  % choose the background color. You must add \usepackage{color}
	showspaces=false,               % show spaces adding particular underscores
	showstringspaces=false,         % underline spaces within strings
	showtabs=false,                 % show tabs within strings adding particular underscores
	frame=single,                   % adds a frame around the code
	tabsize=4,                      % sets default tabsize to 2 spaces
	captionpos=b,                   % sets the caption-position to bottom
	breaklines=true,                % sets automatic line breaking
	breakatwhitespace=false,        % sets if automatic breaks should only happen at whitespace
	title=\lstname,                 % show the filename of files included with \lstinputlisting;
	                                % also try caption instead of title
	escapeinside={\%*}{*)},         % if you want to add a comment within your code
	morekeywords={*,...}            % if you want to add more keywords to the set
}

% adjust table padding
\setlength{\tabcolsep}{8pt} \renewcommand{\arraystretch}{1.5}
 
\title{User Manual}
\author{ToureNPlaner Team}


\newenvironment{owncompactitem}{%
\compactitem
}{%
\@finalstrut\@arstrutbox
\@nameuse{endcompactitem}%
\aftergroup\let\aftergroup\@finalstrut\aftergroup\@gobble
}
\newenvironment{owncompactenum}{%
\compactenum
}{%
\@finalstrut\@arstrutbox
\@nameuse{endcompactenum}%
\aftergroup\let\aftergroup\@finalstrut\aftergroup\@gobble
}
\makeatother

\newcommand{\configoption}[4]
{%\paragraph{#1}
\setlength{\extrarowheight}{2pt}
\begin{tabular}{|p{0.2\textwidth}|p{0.9\textwidth}|}
\hline
  \textbf{Name} & \textbf{#1}\\\hline
  Example & #2\\\hline
  Type & #3\\\hline
  Description & #4\\\hline
\end{tabular}
}

\begin{document}

\maketitle

\tableofcontents

\pagebreak

\section{Introduction}

	\subsection{About ToureNPlaner}
   ToureNPlaner is a server application that implements the following algorithms:
   \begin{itemize}
     \item Shortest Path (with Contraction Hierarchy)
     \item Constrained Shortest Path
     \item Travelling Sales Person
   \end{itemize}
   ToureNPlaner can run in two modes:
   \begin{itemize}
     \item Public - No authentication is required to send requests to the server.
     \item Private - Users that send requests to the server need to be registered and verified. The connection to the server is secured with SSL. Requests and their responses are saved in a MySQL database.
   \end{itemize}


\section{Generelles}
	\subsection{System Requirements}
	
	\begin{itemize}
		\item JVM in Version 1.6 (or higher)
		\item 3 Gigabytes of memory for the graph + TODO Gigabytes per Thread
		\item optionally a MySQL-database for running the server in the private modus
	\end{itemize}

	\label{usedfiles}
  \subsection{Files used by ToureNPlaner}
	 \begin{itemize}
	   \item The ToureNPlaner server itself: A runnable jar ``tourenplaner-server.jar''
	   \item Configuration file: tourenplaner.conf in json format. Its path is given to the server via ``-c /path/to/config/file''
	   \item Street graph in the text format specified in the graph file specification. Its path is given to the server in the configuration file
		\begin{itemize}
		  \item optional: Binary dump of the street graph that is read faster than the text graph. The server uses the dump (and creates it from the textgraph if it is not present) when given ``-f dump''
		\end{itemize}
	   \item optional: a Java Keystore that contains an SSL certificate used for a ``private'' server
	 \end{itemize}

\section{Installation}
  Packages for the following Linux distributions are provided:
  \begin{itemize}
    \item Archlinux
  \end{itemize}
  \subsection{Archlinux Installation}
  Navigate to the archlinuxpackage folder with the PKGBUILD file. Run
  \begin{lstlisting}
makepkg -si
  \end{lstlisting}
  to install the dependencies, build the package and install it.
  \subsection{Debian/Ubuntu Installation}
  The debian package needs \textit{jarwrapper} to be installed.
  Navigate to the debianpackage folder so that you see \textit{debian} as a subfolder. Run
  \begin{lstlisting}
debuild -uc -us
  \end{lstlisting}
  to create a .deb package.
  Run
  \begin{lstlisting}
dpkg -i ../tourenplaner_*.deb
  \end{lstlisting}
  to install the package.
  
  \subsection{Files in the packages}
	\begin{itemize}
	  \item /etc/rc.d/tourenplanerd (Archlinux) or /etc/init.d/tourenplaner (Debian/Ubuntu) : a script for starting and stopping the tourenplaner server as an daemon
	  \item /etc/tourenplaner.conf : the configuration file
	  \item /usr/bin/tourenplaner-server : a minimal startscript for the tourenplaner server (Archlinux only)
	  \item /usr/share/java/tourenplaner/tourenplaner-server.jar (Archlinux) or /usr/share/tourenplaner/tourenplaner-server.jar (Debian/Ubuntu) : the server itself
	  \item scripts to set up the database for a private server
		\begin{itemize}
		  \item /usr/share/tourenplaner/db\_init\_script.sql
		  \item /usr/share/tourenplaner/db\_init.sh
		\end{itemize}
	\end{itemize}

	\subsection{Further Installation Steps  (public and private server)}
	\subsubsection{Street graph}
	 Per default the street graph needs to be copied to /var/lib/tourenplaner/germany.txt.
	\subsection{Further Installation Steps (private server only)}
	 \subsubsection{Java Keystore}
	 Per default a java keystore needs to be copied to /var/lib/tourenplaner/keystore.jks
	 \paragraph{How to create a Java Keystore}
	 A java keystore can be generated with the \textit{keytool} utility:
	 	\begin{lstlisting}
keytool -genkey -keyalg RSA -alias selfsigned -keystore keystore.jks -storepass testpasswort -validity 360 -keysize 2048
		\end{lstlisting}
		If the resulting keystore is placed in \textit{/var/lib/tourenplaner/keystore.jks} the resulting configuration looks like this:
	 	\begin{lstlisting}
"sslcert" : "/var/lib/tourenplaner/keystore.jks",
"sslalias" : "selfsigned",
"sslpw" : "testpasswort",
 		\end{lstlisting}
	 \subsubsection{Setting up the Database (private server only)}
	The database is set up by the \textit{db\_init.sh} script (the MySQL root password is needed)
   \begin{lstlisting}
	 cd /usr/share/tourenplaner
	 ./db_init.sh
	\end{lstlisting}
	
  \subsection{Manual Installation}
	%TODO: Link einf\"ugen
	For the manual installation the files described in \hyperref[usedfiles]{Files used by ToureNPlaner} are required.\\
	Firstly, in the file \textit{tourenplaner.conf} the paths to all further files are configured: The directory containing the street graph needs to be writeable by the user that runs ToureNPlaner to write a binary graph.
	\subsubsection{Manual starting}
  The ToureNPlaner server is started using the following command:
	\begin{lstlisting}
  "${JAVA_HOME}/bin/java" -Xmx12g -Xincgc -jar /pfad/zum/server/tourenplaner-server.jar -c /pfad/zur/config/tourenplaner.conf -f dump
	\end{lstlisting}
	Here \textit{-Xmx12g} tells the JVM how much memory ToureNPlaner is allowed to allocate on the heap. In case the ToureNPlaner dies with an OutOfMemoryException this value should be increased.
	The server can be run as a different user by using \textit{sudo}
	\begin{lstlisting}
	   sudo -E -u http "${JAVA_HOME}/bin/java" -Xmx12g -Xincgc -jar /pfad/zum/server/tourenplaner-server.jar -c /pfad/zur/config/tourenplaner.conf -f dump
	\end{lstlisting}
  The sudo parameter \textit{-E} tells sudo to preserve the environment variables of the executing user, for example \$JAVA\_HOME

	\subsubsection{Command Line Switches}
	ToureNPlaner supports the following command line switches:
	\begin{itemize}
	  \item -c \textit{Pfad} : Path to the configuration file. If it is missing ToureNPlaner will use an inbuilt default configuration.
	  \item -f text / -f dump : Read the graph either from a text file described in the graph format specification or from a binary dump with the same name of the text graph and the added file extension of \textit{.dat} (If -f dump is given and the binary dump cannot be opened ToureNPlaner assumes \textit{-f text} and creates a binary dump) (This parameter is optional and if omitted, the server assumes \textit{-f text})
	 \item dumpgraph : Just creates a binary dump without actually starting the ToureNPlaner server
	\end{itemize}

\section{Configuration}
\subsection{Database Configuration}
\subsubsection{/etc/mysql/my.cnf}
	 \paragraph{\#skip-networking}
	 The ToureNPlaner server connects to the database server only through TCP. MySQL distributions might be set to per default only accept connections through sockets.
	 To enable TCP connections, find \textit{skip-networking} and put the comment sign \textit{\#} in front of it.
	 \paragraph{max\_allowed\_packet}
	 The responses to large request may be larger than the default size of packets that MySQL accepts. Reasonable requests should not generate responses bigger than about 16M.

\subsection{The config file}
The configuration file (/etc/tourenplaner.conf) contains the following options in the JSON format following this pattern:
\begin{lstlisting}[caption=Sample configuration]
  {
	 "A_String" = "text",
	 // A Comment
	 "A_Boolean" = true, // another comment
	 "An_Integer" = 3
  }
\end{lstlisting}

Missing Options will be replaced by inbuilt defaults.

%\configoption
%{}
%{}
%{}
%{}

\subsection{Recommended Options}

\subsubsection{public and private server}

\configoption
{threads}
{2}
{integer}
{Number of Computecores that can calculate requests at the same time. A good value is CPU-cores - 1 (It\'s a good idea to let a spare core that the garbage collector can run on)}

\configoption
{queuelength}
{20}
{integer}
{The number of requests being queued in case all computethreads are active. Requests that are sent to the server when all cores are active and the queue is full will be rejected}

\configoption
{graphfilepath}
{``/var/tourenplaner/germany.txt''}
{string}
{Path to the street graph}

\configoption
{private}
{true}
{boolean}
{A private server has user management, logs all requests and provides its service through HTTPS. A server that is not private is public and answers requests through HTTP without authentication and without saving them}

\configoption
{httppport}
{8080}
{integer}
{listen port for HTTP}

\configoption
{costpertimeunit}
{10}
{integer}
{The cost for a time unit that is set in \textit{timeunitsize}}

\configoption
{timeunitsize}
{1000}
{integer}
{The length of one time unit that has the costs set in \textit{costpertimeunit}. The basis unit is milliseconds. A timeunitsize of 1000 means that the cost is per second}

\configoption
{logfilepath}
{``/var/log/tourenplaner/tourenplaner.log''}
{string}
{The location of the logfile. Should be writeable by the tourenplaner user.}

\configoption
{loglevel}
{``info''}
{string}
{A java loglevel. Possible values: ALL, SEVERE (highest value), WARNING, INFO, CONFIG, FINE, FINER, FINEST (lowest value), OFF. Not case sensitive. FINE generates anonymous requestlogs (Hash of IP and current day)}

\subsubsection{private server only}

\configoption
{store-full-response}
{true}
{boolean}
{If set, the full result of every query is saved in the database. If not set, it only stores the points of the request connected with a straight line}

\configoption
{dburi}
{``jdbc:mysql://localhost:3306/tourenplaner?autoReconnect=true''}
{string}
{The URI to the MySQL database. It must be a JAVA JDBC URI. The database name is directly set with this option (e.g. ``/tourenplaner'')}

\configoption
{dbdriverclass}
{"com.mysql.jdbc.Driver"}
{string}
{}

\configoption
{dbuser}
{``tourenplaner''}
{string}
{MySQL user name}

\configoption
{dbpw}
{toureNPlaner}
{string}
{MySQL password}

\configoption
{sslport}
{8081}
{integer}
{listen port for HTTPS}

\configoption
{sslcert}
{``/var/tourenplaner/keystore.jks''}
{string}
{Path to the Java Keystore that contains the certificate named in sslalias}

\configoption
{sslalias}
{tourenplaner}
{string}
{Identifier of the SSL certificate in the Java Keystore}

\configoption
{sslpw}
{``tour3nplan3er''}
{string}
{Passwort for the Java Keystore}

\subsection{Optional Options}

\configoption
{serverinfosslport}
{443}
{integer}
{The port that is displayed in the serverinfo the ToureNPlaner server answers by a HTTP(s) request on /info. In reality the server listens for HTTPS connections on the port given by sslport}

\configoption
{maxdbtries}
{3}
{integer}
{How often the server tries to reconnect to the database after the connection has been lost}

\subsection{nginx as Proxy for ToureNPlaner}

Theoretically every Webserver that supports ``reverse proxying'' can be used to relay requests to ToureNPlaner. But only nginx has been tested.\\

For the Android Client the ChunkinModule for nginx needs to be installed and set up according to the documentation:
\url{http://wiki.nginx.org/HttpChunkinModule}

The following ``locations'' need to be set in the nginx config file:

\lstinputlisting[caption=nginx location Configuration]{nginx-proxy-location.txt}

\subsubsection{}
For a ``public'' server these locations should be set for the HTTP port.\\
For a ``private'' server these locations should be set for the HTTPS port. If needed, an ``/info'' location can be set for the HTTP port. (Clients can connect to the HTTP port to get the server info and then reconnect to the HTTPS port in case of a private server).

\subsubsection{Public server with SSL with nginx}

It may be desired to run a server in public mode and still use SSL for all requests. This can be achieved with a similar configuration than above.\\
Run the server with the configuration option \textit{``private'' = false} and the above location rule but modify the port so that it points at the http port of the ToureNPlaner server instead of the ssl port.
In this configuration nginx will encrypt all traffic between the client and nginx, but the traffic between nginx and the ToureNPlaner server will be unencrypted.\\
Instead of full logging into a database we provide a lightweight alternative for public servers: On loglevel FINE we log full requests (but no responses) together with an SHA1 hash of the current day concatenated with the IP of the user. In this mode it is very hard to find which IP has issued which requests while still being useful for statistics.\\
The whole effort to retain the privacy of the user can be undermined if there is a Reverse Proxy that logs IP addresses and together with the exact time of an access. In nginx this is how to disable IP logging.\\
First, we define a new logformat named ``no\_ip''. The only difference to the default format is that instead of the IP 0.0.0.0 will be logged.
\begin{lstlisting}
log_format no_ip '0.0.0.0 - $remote_user [$time_local] '
'"$request" $status $body_bytes_sent '
'"$http_referer" "$http_user_agent"';
\end{lstlisting}
Now we find the access\_log directive and add our ``no\_ip'' format after the path. This directive can be server-wide so that any access will be anonymizied or just applied to the tourenplaner location(s).
\begin{lstlisting}
access_log /var/log/nginx/access.log no_ip;
\end{lstlisting}
\end{document}