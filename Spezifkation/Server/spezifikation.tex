%server specification
\documentclass[a4paper,10pt,titlepage,parskip=true]{article}

\usepackage[utf8]{inputenc}
\usepackage[ngerman]{babel}
\usepackage{graphicx}
\usepackage{a4wide}
\usepackage{url}
\usepackage{booktabs}
\usepackage{float}
\usepackage{paralist}
\usepackage{array}
\usepackage{ifthen}

\usepackage[pdfborder={0 0 0}]{hyperref}

\graphicspath{{./grafiken/}}

\author{Christoph Haag, Sascha Meusel, Niklas Schnell, Peter Vollmer}
\title{Spezifikation ToureNPlaner Server\\ Stupro 2011/2012\\ Version }

\newcommand{\shortcut}[1]
{\texttt{#1}}

\makeatletter
\newcommand\novspace{\@minipagetrue}

\newenvironment{owncompactitem}{%
\compactitem
}{%
\@finalstrut\@arstrutbox
\@nameuse{endcompactitem}%
\aftergroup\let\aftergroup\@finalstrut\aftergroup\@gobble
}
\newenvironment{owncompactenum}{%
\compactenum
}{%
\@finalstrut\@arstrutbox
\@nameuse{endcompactenum}%
\aftergroup\let\aftergroup\@finalstrut\aftergroup\@gobble
}
\makeatother

\newcommand{\usecase}[7]
{\subsection{#1}
\setlength{\extrarowheight}{2pt}
\begin{tabular}{|p{0.2\textwidth}|p{0.9\textwidth}|}
\hline
  Akteure & #2\\\hline
  Ziel & #3\\\hline
  Vorbedingungen & \novspace
  	\begin{owncompactitem}[-] #4 \end{owncompactitem} \\\hline
  Normalablauf & \vspace{-7pt}
  	\begin{owncompactenum}[1.] #6 \end{owncompactenum} \\\hline
  Nachbedingungen & \novspace
  	\begin{owncompactitem}[-] #5 \end{owncompactitem} \\\hline
  #7
\end{tabular}
}

\newcommand{\sonderfall}[4][\empty]
{
Sonderfall #2 & \vspace{-10pt}
	\textit{#3}
	\begin{owncompactenum}[{#2}.1] {#4} \end{owncompactenum}
  	\ifthenelse{\equal{#1}{\empty}}
    	{\\\hline} %no opt
    	{\ensuremath{\rightarrow} #1 \\ [+1pt] \hline} %opt

}

\newcommand{\kurzersonderfall}[3][\empty]
{
Sonderfall #2 & \vspace{-10pt}
	\textit{#3}
  	\ifthenelse{\equal{#1}{\empty}}
    	{\\\hline} %no opt
    	{\\&\ensuremath{\rightarrow} #1 \\ [+1pt] \hline} %opt

}

\newcommand{\sondernachbedingung}[1]
{
Nachbedingungen im Sonderfall& \novspace
	\begin{owncompactitem}[-]
		#1
	\end{owncompactitem} \\\hline
}


\newcommand{\begriff}[7]
{\subsection{#1}
\begin{tabular}{|p{0.2\textwidth}|p{0.7\textwidth}|}
\hline
  Bedeutung & #2\\\hline
  Abgrenzung & #3\\\hline
  Gültigkeit & #4\\\hline
  Bezeichnung & #5\\\hline
  Unklarheiten & #6\\\hline
  Querverweise & #7\\\hline
 \end{tabular}}

\newcommand{\doaction}[1]
{Der Benutzer führt die Aktion „\nameref{#1}“ (\ref{#1}) aus.}

\begin{document}

\pagenumbering{alph}
\maketitle
\pagenumbering{roman}
\setcounter{page}{1}
\tableofcontents
\clearpage
\pagenumbering{arabic}
\setcounter{page}{1}

\section{Einleitung}
\subsection{Zweck der Spezifikation}
Die Spezifikation beschreibt die Anforderungen und Funktionalität vom ToureNPlaner Server. Sie ist die Grundlage für alle weiteren Dokumente; insbesondere muss sie fortlaufend mit dem Entwurf verglichen und bei Bedarf angepasst werden.\\
Bei der Erstellung der Spezifikation wurde bewusst auf die Festlegung von Implementierungsdetails verzichtet, um gewisse Freiheiten in der Umsetzung der Anforderungen zu erhalten.

\subsection{Leserkreis}
Der Leserkreis dieses Dokuments besteht aus folgenden Personengruppen:
\begin{itemize}
\item Die Entwickler der ToureNPlaner Serverkomponente
\item Die Beteiligten des Spezifikationsreviews
\item Der Kunde
\item Die Betreuer des Studienprojekts
\end{itemize}

\subsection{Motivation}
Betrachtet man das Angebot an Karten- oder Graphbasierten Webanwendungen wie OpenStreetMap und Google Maps, so fällt auf, dass
es zahlreiche Angebote zum berechnen eines kürzesten Weges von A nach B gibt, jedoch kaum allgemeinere Funktionen zur
Berechnung von (Rund-) Touren oder kürzesten Wegen unter bestimmten Bedingungen.\\
Da viele der Problemstellungen im Bereich der Graphbasierten Rundtouren NP-schwer sind, braucht es hier desweiteren Approximationsalgorithmen und
nicht zu vernachlässigenden Rechenaufwand.\\
Auch fällt auf, dass viele dieser Algorithmen ähnliche Backend-Daten wie zum Beispiel eine Graphrepräsentation benötigen. Es ist also sinnvoll, ein flexibles Framework zu schaffen, um solcherlei Probleme zu lösen.\\
Dieses sollte dabei sowohl eine Umgebung bieten, in der auch aufwendige Berechnungen durchgeführt werden können, als auch die den Algorithmen gemeinsamen Daten und Interfaces
bereitstellen und bündeln. Hierbei scheint es uns sinnvoll, eine Client-Server Struktur einzuführen, bei der aufwendige Berechnungen auf einem potentiell leistungsstarken Server erfolgen, während die Darstellung und Eingabe der Daten auf Webbasierten und/oder mobilen Clients erfolgt.
%\subsection{Konventionen}

\section{Funktionale Anforderungen}
\subsection{Grundfunktionen}
Der Server stellt den implementierten Algorithmen folgende Supportfunktionen bereit:
\begin{itemize}
 \item Geteilte Graphrepräsentation
  \begin{itemize}
   \item 
    Basiert auf OpenStreetMap und einem textbasierten Graphformat, welches 
    bereits erfolgreich in der Abteilung Algorithmik der Universität Stuttgart eingesetzt wird
   \item
    Threadsicheres abstrahierendes Interface zum Zugriff auf die Graphdaten
  \end{itemize}
  \item Bereitstellung eines allgemeinen Interfaces für Algorithmen welches diese implementieren
  \begin{itemize}
   \item Ermöglicht es mit dem in der Protokollspezifikation definierten Protokoll, Proleminstanzen für die implementierten Algorithmen auf dem Server berechnen zu lassen.
   \item Mehrere Probleminstanzen auch verschiedener Algorithmen können gleichzeitig bearbeitet werden (Multithreading)

  \end{itemize}
\end{itemize}
\subsection{Algorithmen}
Der Server unterstützt in erster Ausbaustufe mindestens folgende Algorithmen:
\begin{itemize}
 \item Shortest Path (z.B. Dijkstra oder Dijkstra+Contraction Hierachies)
 \item Traveling-Salesmen
 \item Constrained Shortest Path
\end{itemize}

Weitergehende Informationen siehe Algorithmen-Dokumentation 

\section{Nichtfunktionale Anforderungen}
\subsection{Effizienz}
Die Implementierung der Graphrepräsentation sowie der Algorithmen soll auf Effizienz ausgelegt sein.
Insbesondere bedeutet dies auch, dass höhere Abstraktion in den Interfaces der nötigen Performanz
untergeordnet werden kann (wenn dies Aufgrund der eingesetzten Technologien nötig ist). Da Zugriffe auf den Graph in den meisten Algorithmen einen beträchtlichen Teil des Aufwands ausmachen, hat hier
Geschwindigkeit besondere Priorität.\\
Auch in den Algorithmen selbst, sowie in den von Ihnen verwendeten Datenstrukturen (z.B. Priority Queues) ist auf höchst mögliche Effizienz zu achten.
\subsection{Erweiterbarkeit}
Eine weitere wichtige Anforderung ist die gute Erweiterbarkeit des zu entwickelnden Systems um neue Algorithmen, Graphdaten, aber auch um neue Klienten.
Bei den Algorithmen soll es hierbei möglichst einfach zu implementierende Interfaces geben, die neuen Algorithmen die gesamte Funktionalität des Servers,
also sowohl die Erreichbarkeit über das Netzwerk, als auch den Zugriff auf die Graphdaten zugänglich machen.\\
Der Server fungiert hierbei als generisches Framework zur Implementierung verschiedener Graph- und kartenbasierter Algorithmen.
\subsection{Mengengerüst}
\label{Mengengeruest}

\begin{itemize}
 \item Die Algorithmen sollen mindestens auf dem gesamten Deutschlandgraph operieren können
 \item Anzahl gleichzeitig laufender Berechnungen sowie die Warteschlangenlänge ist frei konfigurierbar
\end{itemize}

\subsection{Entwurfseinschränkungen}
\subsubsection{Systemumgebung}
\label{Systemumgebung}
Das System wird für Rechner entwickelt mit mindestens 4 GB RAM und 4 Kernen. Jedoch wird darauf geachtet, dass das System gut skaliert und auch auf größeren Rechnern die zusätzliche Leistung nutzen kann.
Als Referenz kann das vom Kunden genannte System „Alge“ dienen, das 24 Kerne sowie 96 GB RAM besitzt.

\begin {itemize}
 \item Nutzt Java SE 6 und höher
 \item Soll auf allen von Java SE unterstützten Systemen laufen, primär auf Unix-basierten Systemen.
\end {itemize}
\subsubsection{Datenhaltung}
\label{datenhaltung}
Benutzerkonten sowie Request-Log Daten werden in einer MySQL-Datenbank gehalten.\\
Die Graphdaten werden zur Laufzeit im RAM gespeichert. Sie werden aus einer Graphdatei geladen siehe Dokument Graphformat.

\subsection{Robustheit}
Durch softwaretechnische Maßnahmen, wie Unit-Tests und einem ausgiebigen Systemtest wird die Robustheit der Serverkomponente sichergestellt.
\subsection{Portabilität}
siehe \ref{Systemumgebung}
\subsection{Distributionsform und Installation}
Die Serverapplikation wird als ausführbares „Jar“ ausgeliefert, außerdem werden für ausgewählte Linux Distributionen Pakete bereitgestellt. Zur Installation wird ein MySQL-Server benötigt die Einrichtung der Datenbank wird über mitgelieferte Skripte erleichtert.
\subsection{Sprachunterstützung}
Als Sprache wird Englisch verwendet. Weiter Sprachen werden nicht unterstützt.

\clearpage
\section{Anwendungsfälle (Use-Cases)}
%\begin{figure}[H]
%  \centering
  %TODO
  %\includegraphics[width=\linewidth]{usecases.png}
%  \caption{Use Cases}
%\end{figure}

%\usecase{Name}{Akteur}{Ziel}{Vorbedingungen}{Nachbedingungen}{Normalablauf}{Sonderfälle und dessen Nachbedingungen}

%\sonderfall[optionale letzte Zeile für "-> Schritt ..."]{Nummer}{Bezeichnung}{Ablauf}

%\sondernachbedingung{Bedingungen}

%\usecase{}{}%Name und Akteure
{%Ziel
}{%Vorbedingungen
}{%Nachbedingungen
}{%Normalablauf
}{%Sondernachbedingungen und Sonderfälle
}

%\sonderfall[]{}%Letzte Zeile und Nummer
{%Bezeichnung
}{%Sonderfallablauf
}

%\kurzersonderfall[]{}%Letzte Zeile und Nummer
{%Bezeichnung
}

\usecase{Server-Informationen abfragen}{Client}%Name und Akteure
{%Ziel
Der Client möchte die Server-Informationen über die Protokollversion und die verfügbaren Algorithmen erhalten
}{%Vorbedingungen
\item Der Server ist in Betrieb
\item Der Server hat einen Request über \textit{/info} vom Client erhalten
}{%Nachbedingungen
\item Die Server-Informationen wurden dem Client gesendet
}{%Normalablauf
\item Der Server sendet die aktuellsten Server-Informationen als Response
}{%Sondernachbedingungen und Sonderfälle
\sondernachbedingung{
	\item Der Server führt seine Arbeit wie gewohnt fort
	\item Die Fehlermeldungen wurden ins Log geschrieben}

\sonderfall[Weiter mit normalem Betrieb]{1a}%Letzte Zeile und Nummer
	{%Bezeichnung
	TCP/IP Verbindung abgebrochen
	}{%Sonderfallablauf
	\item Fehlermeldung „TCP/IP Verbindung abgebrochen“ ins Log schreiben}
}



\usecase{Probleminstanz verarbeiten}{Client, Abrechnungs-DB, Benutzer-DB}%Name und Akteure
{Der Client möchte eine Probleminstanz berechnet haben}%Ziel
{%Vorbedingungen
  \item Der Server ist in Betrieb
  \item Der Server hat einen Request über \textit{/algsuffix} vom Client erhalten
}
{%Nachbedingungen
  \item Der Client hat das Berechnungsergebnis erhalten
  \item Der Server hat die Berechnung in der Abrechnungs-DB gelogged
}
{%Normalablauf
  \item Der Server prüft Benutzername und Passwort anhand der Benutzer-DB
  \item Der Server schreibt den zu berechnenden Request als \textit{pending} in die Abrechnungs-DB
  \item Der Server wählt die passende Algorithmusinstanz aus den vorhandenen Algorithmen
  \item Der Server berechnet mit der Algorithmusinstanz die Lösung der Probleminstanz
  \item Der Server trägt die Lösung in die Abrechnungs-DB ein und entfernt die \textit{pending}-Markierung
  \item Der Server sendet die Lösung an den Client
  }
{%Sondernachbedingungen und Sonderfälle
\sondernachbedingung{
	\item Der Server führt seine Arbeit wie gewohnt fort, solange kein kritischer Fehler aufgetreten ist
	\item Die Fehlermeldungen wurden ins Log geschrieben
	}
\sonderfall[Weiter mit normalem Betrieb]{1a}
    {Ungültige Authentifizerung}
    {
    \item Fehler wird ins Log geschrieben
  	\item Der Client erhält eine entsprechende Fehlermeldung mit HTTP-Status 403 Access Denied
    }
    
\sonderfall[Weiter mit normalem Betrieb]{3a}%Letzte Zeile und Nummer
	{Keine Passende Algorithmusinstanz}%Bezeichnung
  	{
	\item Der Fehler wird ins Log geschrieben
	\item Der Client erhält eine entsprechende Fehlermeldung
  	}

\sonderfall[Weiter mit normalem Betrieb]{4a}%Letzte Zeile und Nummer
	{Die Berechnung bricht mit einem Fehler ab}%Bezeichnung
  	{
	\item Der Fehler wird ins Log geschrieben
	\item Der Request wird in der Abrechnungs-DB auf \textit{failed} gesetzt und der genaue Fehler eingetragen
	\item Der Client erhält eine entsprechende Fehlermeldung
  	}

\sonderfall[Kritischer Fehler, Server ist beendet]{*}%Letzte Zeile und Nummer
	{Verbindung zu Abrechnungs-DB oder Benutzer-DB verloren}%Bezeichnung
  	{
	\item Der Fehler wird ins Log geschrieben (als schwerwiegender Fehler)
	\item Der Client erhält eine entsprechende Fehlermeldung
	\item Der Server wird beendet
  	}

\sonderfall[Weiter mit normalem Betrieb]{**}%Letzte Zeile und Nummer
	{TCP/IP Verbindung abgebrochen}%Bezeichnung
	{
	\item Fehlermeldung wird ins Log geschrieben
	}

}

\usecase{Benutzer registrieren}{Client, Benutzer-DB}%Name und Akteure
{Ein Benutzer soll in der Benutzer-DB registriert werden}%Ziel
{%Vorbedingungen
  \item Der Server ist in Betrieb.
  \item Der Server hat einen Request über \textit{/register} vom Client erhalten
}
{%Nachbedingungen
  \item Benutzer ist in Datenbank eingetragen, aber noch nicht als aktiv gekennzeichnet.
  \item Client wurde benachrichtigt, dass der Benutzer aufgenommen wurde.
}
{%Normalablauf
  \item  Die Werte werden in die Benutzer-DB eingetragen.
  \item  Der Server sendet dem Client die Antwort 200 OK.
}
{%Sondernachbedingungen und Sonderfälle
  \sondernachbedingung{
	\item Benutzer wurde nicht in die Datenbank aufgenommen.
	\item Fehler wurde ins Log geschrieben.
	}
  
  \sonderfall[Fehler]{1a}%Letzte Zeile und Nummer
	  {Wert existiert schon (Benutzername)}%Bezeichnung
	  {
	  \item Der Fehler wird ins Log geschrieben.
	  \item Dem Client wird ein entsprechender Fehler übermittelt.
	  } 

  \sonderfall[Fehler]{1b}%Letzte Zeile und Nummer
	  {Wert ist leer}%Bezeichnung
	  {
	  \item Der Fehler wird ins Log geschrieben.
	  \item Dem Client wird ein entsprechender Fehler übermittelt.
	  }

	  \sonderfall[Kritischer Fehler, Server ist beendet]{*}%Letzte Zeile und Nummer
	{Verbindung zu Benutzer-DB verloren}%Bezeichnung
  	{
	\item Der Fehler wird ins Log geschrieben (als schwerwiegender Fehler)
	\item Der Client erhält eine entsprechende Fehlermeldung
	\item Der Server wird beendet
  	}

\sonderfall[Weiter mit normalem Betrieb]{**}%Letzte Zeile und Nummer
	{TCP/IP Verbindung abgebrochen}%Bezeichnung
	{
	\item Fehlermeldung wird ins Log geschrieben
	}
}

\usecase{Benutzer anmelden}{Client, Benutzer-DB}%Name und Akteure
{Benutzer wurde authentifiziert}%Ziel
{%Vorbedingungen
  \item Der Server ist in Betrieb.
  \item Der Server hat einen Request über \textit{/login} vom Client
}
{%Nachbedingungen
  \item Der Client hat ein 200 OK erhalten.
}
{%Normalablauf
  \item Der Server prüft Benutzername und Passwort anhand der Benutzer-DB
  \item Schickt an Client einen 200 OK.
}
{%Sondernachbedingungen und Sonderfälle
  \sondernachbedingung{
	\item Client hat 403 Access Denied erhalten.
	\item Fehler steht im Log.
	}
	\sonderfall[Weiter mit normalem Betrieb]{1a}
    {Ungültige Authentifizerung}
    {
    \item Fehler wird ins Log geschrieben
  	\item Der Client erhält eine entsprechende Fehlermeldung mit HTTP-Status 403 Access Denied
    }
	
  
	\sonderfall[Kritischer Fehler, Server ist beendet]{*}%Letzte Zeile und Nummer
	{Verbindung zu Benutzer-DB verloren}%Bezeichnung
  	{
	\item Der Fehler wird ins Log geschrieben (als schwerwiegender Fehler)
	\item Der Client erhält eine entsprechende Fehlermeldung
	\item Der Server wird beendet
  	}

\sonderfall[Weiter mit normalem Betrieb]{**}%Letzte Zeile und Nummer
	{TCP/IP Verbindung abgebrochen}%Bezeichnung
	{
	\item Fehlermeldung wird ins Log geschrieben
	}
}

%TODO: ggfs mit überarbeiteter Protokollspezi abgleichen
\usecase{Benutzerdaten abfragen}{Client, Benutzer-DB}%Name und Akteure
{Client erhält seine Benutzerdaten}%Ziel
{%Vorbedingungen
  \item Der Server ist in Betrieb.
  \item Der Server hat einen Request über \textit{/retrieve} vom Client empfangen
}
{%Nachbedingungen
  \item Client hat die Benutzerdaten erhalten.
}
{%Normalablauf
  \item Der Server prüft Benutzername und Passwort anhand der Benutzer-DB
  \item Benutzerdaten werden an Client geschickt.
}
{%Sondernachbedingungen und Sonderfälle
  \sondernachbedingung{
	\item Client hat 403 Access Denied erhalten.
	\item Fehler steht im Log.
	}
	
	\sonderfall[Weiter mit normalem Betrieb]{1a}
    {Ungültige Authentifizerung}
    {
    \item Fehler wird ins Log geschrieben
  	\item Der Client erhält eine entsprechende Fehlermeldung mit HTTP-Status 403 Access Denied
    }

	 \sonderfall[Kritischer Fehler, Server ist beendet]{*}%Letzte Zeile und Nummer
	{Verbindung zu Benutzer-DB verloren}%Bezeichnung
  	{
	\item Der Fehler wird ins Log geschrieben (als schwerwiegender Fehler)
	\item Der Client erhält eine entsprechende Fehlermeldung
	\item Der Server wird beendet
  	}

\sonderfall[Weiter mit normalem Betrieb]{**}%Letzte Zeile und Nummer
	{TCP/IP Verbindung abgebrochen}%Bezeichnung
	{
	\item Fehlermeldung wird ins Log geschrieben
	}
}

\usecase{Benutzerdaten verändern}{Client, Benutzer-DB}%Name und Akteure
{Client verändert seine Benutzerdaten}%Ziel
{%Vorbedingungen
  \item Der Server ist in Betrieb.
  \item Der Server hat einen POST-Request mit optionalem id-Parameter über \textit{/update} vom Client empfangen
}
{%Nachbedingungen
  \item Benutzerdaten wurden geändert.
}
{%Normalablauf
  \item Der Server prüft Benutzername und Passwort anhand der Benutzer-DB
  \item Benutzerdaten werden anhand des id-Parameters bzw. aus Basic-Auth-Informationen aus der Datenbank gelesen.
  \item Geänderte Benutzerdaten werden in die Datenbank geschrieben
}
{%Sondernachbedingungen und Sonderfälle
  \sondernachbedingung{
	\item Client hat 403 Access Denied erhalten.
	\item Fehler steht im Log.
	}
	
	\sonderfall[Weiter mit normalem Betrieb]{1a}
    {Ungültige Authentifizerung}
    {
    \item Fehler wird ins Log geschrieben
  	\item Der Client erhält eine entsprechende Fehlermeldung mit HTTP-Status 403 Access Denied
    }
    
   	\sonderfall[Weiter mit normalem Betrieb]{2a}
    {Ungültige id}
    {
    \item Fehler wird ins Log geschrieben
  	\item Der Client erhält eine entsprechende Fehlermeldung
    }

	 \sonderfall[Kritischer Fehler, Server ist beendet]{*}%Letzte Zeile und Nummer
	{Verbindung zu Benutzer-DB verloren}%Bezeichnung
  	{
	\item Der Fehler wird ins Log geschrieben (als schwerwiegender Fehler)
	\item Der Client erhält eine entsprechende Fehlermeldung
	\item Der Server wird beendet
  	}

\sonderfall[Weiter mit normalem Betrieb]{**}%Letzte Zeile und Nummer
	{TCP/IP Verbindung abgebrochen}%Bezeichnung
	{
	\item Fehlermeldung wird ins Log geschrieben
	}
}

%TODO: Billing vs History
\usecase{Billinginformationen abfragen}{Client, Abrechnungs-DB, Benutzer-DB}%Name und Akteure
{Die in der Benutzer-DB gespeicherten Requests des Benutzers sowie deren Ergebnisse (bzw. Status) und Kosten sollen angezeigt werden}%Ziel
{%Vorbedingungen
  \item Der Server hat einen Get-Request an /billing erhalten
}
{%Nachbedingungen
  \item Die entsprechenden Informationen wurden gesendet
}
{%Normalablauf
  \item Der Server prüft Benutzername und Passwort anhand der Benutzer-DB
  \item Der Server liest aus der Abrechnungs-DB alle vom Benutzer berechneten Requests sowie Zahlungsinformationen zu diesen Requests aus
  \item Der Server schickt dem Client eine Liste seiner bisher berechneten und \textit{pending} Requests sowie Zahlungsinformationen
}
{%Sondernachbedingungen und Sonderfälle
  \sondernachbedingung{
	\item Client hat 403 Access Denied erhalten.
	\item Fehler steht im Log.
	}
		\sonderfall[Weiter mit normalem Betrieb]{1a}
    {Ungültige Authentifizerung}
    {
    \item Fehler wird ins Log geschrieben
  	\item Der Client erhält eine entsprechende Fehlermeldung mit HTTP-Status 403 Access Denied
    }
	\sonderfall[Kritischer Fehler, Server ist beendet]{*}%Letzte Zeile und Nummer
	{Verbindung zu Benutzer-DB verloren}%Bezeichnung
  	{
	\item Der Fehler wird ins Log geschrieben (als schwerwiegender Fehler)
	\item Der Client erhält eine entsprechende Fehlermeldung
	\item Der Server wird beendet
  	}

\sonderfall[Weiter mit normalem Betrieb]{**}%Letzte Zeile und Nummer
	{TCP/IP Verbindung abgebrochen}%Bezeichnung
	{
	\item Fehlermeldung wird ins Log geschrieben
	}
}

%TODO: Protokollspezifikation sagt hier noch gar nichts, also gibt's auch keinen Use Case
%\usecase{Billinginformationen Abfragen}{Client, Abrechnungs DB}%Name und Akteure
%{Die bisherigen und ausstehenden Zahlungen für die bisherigen Berechnungen sollen angezeigt werden, Administratorübersicht}%Ziel
%{%Vorbedingungen
%  \item Der Server hat einen Get-Request von einem als Administrator angemeldeten Benutzer an /billing erhalten
%}
%{%Nachbedingungen
%  \item
%}
%{%Normalablauf
%  \item
%}
%{%Sondernachbedingungen und Sonderfälle
%  \sondernachbedingung{
%	\item
%	}
%  \sonderfall[Panic]{2a}%Letzte Zeile und Nummer
%	  {}%Bezeichnung
%	  {
%	  \item .
%	  }
%}

%TODO Protokollspezifikation??
\usecase{Benutzerliste abfragen}{Client, Benutzer-DB}%Name und Akteure
{Eine Liste der Benutzer wird abgefragt}%Ziel
{%Vorbedingungen
  \item Der Server hat einen Get-Request an /users erhalten
}
{%Nachbedingungen
  \item Die entsprechenden Informationen wurden gesendet
}
{%Normalablauf
  \item Der Server prüft Benutzername und Passwort anhand der Benutzer-DB
  \item Der Server liest aus der Benutzer-DB alle durch die offset- und limit-Parameter ausgewählten Benutzer-Informationen aus
  \item Der Server schickt dem Client die ausgelesenen Informationen
}
{%Sondernachbedingungen und Sonderfälle
  \sondernachbedingung{
	\item Client hat 403 Access Denied erhalten.
	\item Fehler steht im Log.
	}
		\sonderfall[Weiter mit normalem Betrieb]{1a}
    {Ungültige Authentifizerung}
    {
    \item Fehler wird ins Log geschrieben
  	\item Der Client erhält eine entsprechende Fehlermeldung mit HTTP-Status 403 Access Denied
    }

	\sonderfall[Weiter mit normalem Betrieb]{2a}%Letzte Zeile und Nummer
	{Ungültiges Offset/Limit}%Bezeichnung
  	{
	\item Der Fehler wird ins Log geschrieben
	\item Der Client erhält eine entsprechende Fehlermeldung
  	}
  	    
	\sonderfall[Kritischer Fehler, Server ist beendet]{*}%Letzte Zeile und Nummer
	{Verbindung zu Benutzer-DB verloren}%Bezeichnung
  	{
	\item Der Fehler wird ins Log geschrieben (als schwerwiegender Fehler)
	\item Der Client erhält eine entsprechende Fehlermeldung
	\item Der Server wird beendet
  	}

\sonderfall[Weiter mit normalem Betrieb]{**}%Letzte Zeile und Nummer
	{TCP/IP Verbindung abgebrochen}%Bezeichnung
	{
	\item Fehlermeldung wird ins Log geschrieben
	}
}



\usecase{Benutzer löschen}{Client, Benutzer-DB}%Name und Akteure
{Das Benutzerkonto eines Kunden soll aus der Benutzer-DB gelöscht werden}%Ziel
{%Vorbedingungen
  \item Der Server hat einen Get-Request an /delete erhalten
}
{%Nachbedingungen
  \item Der gelöschte Benutzer ist tatsächlich gelöscht
}
{%Normalablauf
  \item Der Server prüft Benutzername und Passwort anhand der Benutzer-DB
  \item Der Server, der den Request erhält, löscht den Benutzer aus der Benutzer-DB.
  %TODO: rechtliche implikationen, nachvollziehbarkeit?
  %\item Die laufenden Jobs des gelöschten Benutzers werden abgebrochen/doch noch ausgeführt
}
{%Sondernachbedingungen und Sonderfälle
  \sondernachbedingung{
	\item Client hat 403 Access Denied erhalten.
	\item Fehler steht im Log.
	}
  \sonderfall[Weiter mit normalem Betrieb]{1a}%Letzte Zeile und Nummer
  	%TODO: Vielleicht doch nicht
	  {Zu löschender Benutzer existiert nicht in der Datenbank}%Bezeichnung
	  {
	  \item Der Server schickt eine entsprechende Fehlermeldung an den Client, der die Lösch-Anfrage verursacht hat
	  }
	  		\sonderfall[Weiter mit normalem Betrieb]{1a}
    {Ungültige Authentifizerung}
    {
    \item Fehler wird ins Log geschrieben
  	\item Der Client erhält eine entsprechende Fehlermeldung mit HTTP-Status 403 Access Denied
    }
	\sonderfall[Kritischer Fehler, Server ist beendet]{*}%Letzte Zeile und Nummer
	{Verbindung zu Benutzer-DB verloren}%Bezeichnung
  	{
	\item Der Fehler wird ins Log geschrieben (als schwerwiegender Fehler)
	\item Der Client erhält eine entsprechende Fehlermeldung
	\item Der Server wird beendet
  	}

\sonderfall[Weiter mit normalem Betrieb]{**}%Letzte Zeile und Nummer
	{TCP/IP Verbindung abgebrochen}%Bezeichnung
	{
	\item Fehlermeldung wird ins Log geschrieben
	}
}
\clearpage
\appendix
\section{Begriffslexikon}

%\begriff{Begriff}{Bedeutung}{Abgrenzung}{Gültigkeit}{Bezeichnung}{Unklarheiten}{Querverweise}

%\begriff
{%Begriff
}{%Bedeutung
}{%Abgrenzung
}{%Gültigkeit
}{%Bezeichnung
}{%Unklarheiten
}{%Querverweise
}

\begriff
{%Begriff
ToureNPlaner
}{%Bedeutung
ToureNPlaner ist die Bezeichnung für das Studienprojekt ToureNPlaner und für die Software ToureNPlaner.
}{%Abgrenzung
Es ist kein allgemeiner Tourenplaner gemeint.
}{%Gültigkeit
-
}{%Bezeichnung
Es muss aus dem Kontext erschlossen werden, ob mit dem Begriff ToureNPlaner das Studienprojekt oder die Software ToureNPlaner gemeint ist. Wenn im Kontext von der Funktionalität vom ToureNPlaner die Rede ist, ist damit zum Beispiel die Software ToureNPlaner gemeint.
}{%Unklarheiten
-
}{%Querverweise
Software ToureNPlaner, Studienprojekt ToureNPlaner
}

\begriff
{%Begriff
Studienprojekt ToureNPlaner, synonym Stupro
}{%Bedeutung
Das Studienprojekt ToureNPlaner ist ein Studienprojekt im Rahmen des Studienganges Softwaretechnik an der Universität Stuttgart. Im Rahmen von diesem Projekt muss die Software ToureNPlaner entwickelt werden.
}{%Abgrenzung
Es ist nicht die Software ToureNPlaner gemeint.
}{%Gültigkeit
Das Studienprojekt ToureNPlaner hat am Anfang des Sommersemesters 2011 begonnen und wird voraussichtlich am Anfang des Sommersemesters 2012 beendet sein.
}{%Bezeichnung
-
}{%Unklarheiten
-
}{%Querverweise
Software ToureNPlaner, ToureNPlaner
}

\begriff
{%Begriff
Software ToureNPlaner
}{%Bedeutung
Die Software ToureNPlaner ist die Bezeichnung für die Gesamtheit aller im Rahmen des Projektes entwickelten und verwendeten Systeme, auch die zu entwickelnde Software genannt. Dazu gehören der Server und die Clients sowie ihre verwendeten Komponenten.
}{%Abgrenzung
Es ist nicht das Projekt ToureNPlaner gemeint.
}{%Gültigkeit
Die Software ToureNPlaner hat mit dem Beginn des Studienprojektes angefangen zu existieren, unabhängig von der Funktionalität der Software. Im Rahmen des Studienprojektes entstehende Dokumente und Systeme sind Teil der Software, auch wenn diese letztendlich nicht mit ausgeliefert oder gelöscht werden. Die Software ToureNPlaner wird nach dem Ende des Studienprojektes ToureNPlaner noch weiter bestehen.
}{%Bezeichnung
-
}{%Unklarheiten
-
}{%Querverweise
Studienprojekt ToureNPlaner, ToureNPlaner
}

\begriff
{%Begriff
graphbasierte Webanwendung
}{%Bedeutung
Eine Webanwendung, die Dienste für bestimmte Graphprobleme anbietet. Ein bekanntes Beispiel ist die Berechnung des kürzesten Weges zwischen zwei Punkten auf einer Karte, auch als Routenplanung bekannt.
}{%Abgrenzung
Webanwendungen, die beliebige Daten speichern, die man als Graph interpretieren kann, sind nicht zwangsläufig als graphbasierte Webanwendungen zu bezeichnen. Die Daten und deren Interpretation als Graph müssen Kernaspekte der angebotenen Dienste sein, damit die Webanwendung als graphbasiert bezeichnet werden kann.
}{%Gültigkeit
-
}{%Bezeichnung
-
}{%Unklarheiten
Es ist unklar, welche Arten von Graphen (z.B. nur Kartengraphen) gemeint sind.
}{%Querverweise
Webanwendung
}

\begriff
{%Begriff
Webanwendung
}{%Bedeutung
Eine Anwendung, die mit einem Internetbrowser zugänglich ist und mit einem Server kommuniziert.
}{%Abgrenzung
Sonstige Anwendungen, die mittels Internet-Protokollen (z.B. HTTP) mit einem Server kommunizieren, wie zum Beispiel Apps, sind keine Webanwendungen.
}{%Gültigkeit
-
}{%Bezeichnung
-
}{%Unklarheiten
Sind Flash/Java-Applets etc. Webanwendungen?
}{%Querverweise
App
}


\begriff
{%Begriff
Tour, synonym Rundtour, Rundreise
}{%Bedeutung
Eine Tour beschreibt einen mit bestimmten Fortbewegungsmitteln oder auch zu Fuß begehbaren Weg, wobei dieser Weg einen festgelegten oder bliebigen Startpunkt hat. Die Begehung dieses Weges führt am Ende wieder zum Startpunkt, somit entspricht eine Tour einer Route, bei der End- und Startpunkt identisch sind.
}{%Abgrenzung
Eine Route, bei Start- und Endpunkt unterschiedlich sind, ist keine Tour.
}{%Gültigkeit
-
}{%Bezeichnung
Eine Tour ist durch eine geordnete Liste von Koordinaten definiert.
}{%Unklarheiten
-
}{%Querverweise
-
}

\begriff
{%Begriff
Backend-Daten
}{%Bedeutung
Alle Daten, die der Server für seinen Betrieb in Datenbanken oder auf andere Weise speichert, werden als Backend-Daten bezeichnet. Zu den Backend-Daten gehört auch die Graphrepräsentation.
}{%Abgrenzung
Daten auf Client-Systemen, die der Server nicht kennt, gehören nicht zu den Backend-Daten.
}{%Gültigkeit
-
}{%Bezeichnung
-
}{%Unklarheiten
-
}{%Querverweise
Graphrepräsentation
}

\begriff
{%Begriff
Graphrepräsentation
}{%Bedeutung
Einen Graphen, wie zum Beispiel die Straßennetzkarte von Deutschland, kann man auf verschiedene Weise speichern, es gibt also verschiedene Möglichkeiten der Graphrepräsentation. Eine Graphrepräsentation hat eine bestimmte Art, wie Kanten und Knoten sowie zugehörige Daten eines Graphen gespeichert werden. Auf der Graphrepräsentation werden die von Kunden gewünschten Algorithmen ausgeführt.
}{%Abgrenzung
Die grafische Darstellung eines Graphen hat nichts mit der Graphrepräsentation im Sinne der Server-Spezifikation zu tun.
}{%Gültigkeit
-
}{%Bezeichnung
-
}{%Unklarheiten
-
}{%Querverweise
-
}

\begriff
{%Begriff
Webbasierter Client
}{%Bedeutung
Ein Client im Sinne des ToureNPlaner Projekts, welcher als Webanwendung realisiert ist.
}{%Abgrenzung
Ein Client, der nicht im Browser läuft, ist kein Webclient.
}{%Gültigkeit
-
}{%Bezeichnung
-
}{%Unklarheiten
-
}{%Querverweise
Webanwendung, Client
}

\begriff
{%Begriff
textbasiertes Graphformat
}{%Bedeutung
Rerpäsentation eines allgemeinen Graphen mit Knoten, Kanten und Zusatzinformationen (wie z.B. Kantengewichten) als Textdatei (siehe auch Graph-Format-Dokument).
}{%Abgrenzung
Hier ist das von ToureNPlaner verwendete Format und nicht etwa eine XML-Repräsentation gemeint.
}{%Gültigkeit
-
}{%Bezeichnung
-
}{%Unklarheiten
Sind durch das Graph-Format-Dokument zu klären.
}{%Querverweise
-
}

\begriff
{%Begriff
Client
}{%Bedeutung
Eine Anwendung, die auf die Funktionen des ToureNPlaner Servers mittels des ToureNPlaner Protokolls (siehe Protokollspezifikation) zugreift.
}{%Abgrenzung
Innerhalb dieser Spezifikation bezeichnet Client ausschließlich Clients für den ToureNPlaner Server.
}{%Gültigkeit
-
}{%Bezeichnung
-
}{%Unklarheiten
-
}{%Querverweise
Webbasierter Client
}

\begriff
{%Begriff
Abrechnungs-DB, synonym Request-Log
}{%Bedeutung
Die Abrechnungs-DB (Abrechnungs-Datenbank) speichert alle empfangenen Algorithmus-Requests (Anfragen) von Clients und deren Berechnungsergebnisse des Servers, die für die Abrechnung benötigt werden. Außerdem kann der Status einer Berechnung hier gespeichert werden.
}{%Abgrenzung
Die Abrechnungs-DB ist keine eigenständige Datenbank, sondern ist eine Teilmenge an Tabellen einer Datenbank mit weiteren Tabellen.
}{%Gültigkeit
-
}{%Bezeichnung
Die Abrechnungs-DB und die Benutzer-DB befinden sich im selben Datenbanksystem, beides sind aber verschiedene Datenbank-Tabellen und für die Use-Cases sind es konzeptionell verschiedene Akteure. 
}{%Unklarheiten
-
}{%Querverweise
Benutzer-DB
}

\begriff
{%Begriff
Benutzer-DB
}{%Bedeutung
Die Benutzer-DB (Benutzer-Datenbank) speichert alle nötigen Informationen über die Benutzer des Systems, dabei werden sowohl Admins wie auch Kunden in der Benutzer-DB gespeichert.
}{%Abgrenzung
Die Benutzer-DB ist keine eigenständige Datenbank, sondern ist eine Teilmenge an Tabellen einer Datenbank mit weiteren Tabellen.
}{%Gültigkeit
-
}{%Bezeichnung
Die Abrechnungs-DB und die Benutzer-DB befinden sich im selben Datenbanksystem, beides sind aber verschiedene Datenbank-Tabellen und für die Use-Cases sind es konzeptionell verschiedene Akteure.
}{%Unklarheiten
-
}{%Querverweise
Abrechnungs-DB
}


%\begriff
{%Begriff

}{%Bedeutung
%-
}{%Abgrenzung
%-
}{%Gültigkeit
%-
}{%Bezeichnung
%-
}{%Unklarheiten
%-
}{%Querverweise
%-
}

\end{document}
