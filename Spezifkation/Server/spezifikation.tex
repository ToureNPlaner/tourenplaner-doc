%server specification
\documentclass[a4paper,10pt,titlepage,parskip=true]{article}

\usepackage[utf8]{inputenc}
\usepackage[ngerman]{babel}
\usepackage{graphicx}
\usepackage{a4wide}
\usepackage{url}
\usepackage{booktabs}
\usepackage{float}
\usepackage{paralist}
\usepackage{array}
\usepackage{ifthen}

\usepackage[pdfborder={0 0 0}]{hyperref}

\graphicspath{{./grafiken/}}

\author{Christoph Haag, Sascha Meusel, Niklas Schnell, Peter Vollmer}
\title{Spezifikation ToureNPlaner Server\\ Stupro 2011/2012\\ Version }

\newcommand{\shortcut}[1]
{\texttt{#1}}

\makeatletter
\newcommand\novspace{\@minipagetrue}

\newenvironment{owncompactitem}{%
\compactitem
}{%
\@finalstrut\@arstrutbox
\@nameuse{endcompactitem}%
\aftergroup\let\aftergroup\@finalstrut\aftergroup\@gobble
}
\newenvironment{owncompactenum}{%
\compactenum
}{%
\@finalstrut\@arstrutbox
\@nameuse{endcompactenum}%
\aftergroup\let\aftergroup\@finalstrut\aftergroup\@gobble
}
\makeatother

\newcommand{\usecase}[7]
{\subsection{#1}
\setlength{\extrarowheight}{2pt}
\begin{tabular}{|p{0.2\textwidth}|p{0.9\textwidth}|}
\hline
  Akteure & #2\\\hline
  Ziel & #3\\\hline
  Vorbedingungen & \novspace
  	\begin{owncompactitem}[-] #4 \end{owncompactitem} \\\hline
  Normalablauf & \vspace{-7pt}
  	\begin{owncompactenum}[1.] #6 \end{owncompactenum} \\\hline
  Nachbedingungen & \novspace
  	\begin{owncompactitem}[-] #5 \end{owncompactitem} \\\hline
  #7
\end{tabular}
}

\newcommand{\sonderfall}[4][\empty]
{
Sonderfall #2 & \vspace{-10pt}
	\textit{#3}
	\begin{owncompactenum}[{#2}.1] {#4} \end{owncompactenum}
  	\ifthenelse{\equal{#1}{\empty}}
    	{\\\hline} %no opt
    	{\ensuremath{\rightarrow} #1 \\ [+1pt] \hline} %opt

}

\newcommand{\kurzersonderfall}[3][\empty]
{
Sonderfall #2 & \vspace{-10pt}
	\textit{#3}
  	\ifthenelse{\equal{#1}{\empty}}
    	{\\\hline} %no opt
    	{\\&\ensuremath{\rightarrow} #1 \\ [+1pt] \hline} %opt

}

\newcommand{\sondernachbedingung}[1]
{
Nachbedingungen im Sonderfall& \novspace
	\begin{owncompactitem}[-]
		#1
	\end{owncompactitem} \\\hline
}


\newcommand{\begriff}[7]
{\subsection{#1}
\begin{tabular}{|p{0.2\textwidth}|p{0.7\textwidth}|}
\hline
  Bedeutung & #2\\\hline
  Abgrenzung & #3\\\hline
  Gültigkeit & #4\\\hline
  Bezeichnung & #5\\\hline
  Unklarheiten & #6\\\hline
  Querverweise & #7\\\hline
 \end{tabular}}

\newcommand{\doaction}[1]
{Der Benutzer führt die Aktion „\nameref{#1}“ (\ref{#1}) aus.}

\begin{document}

\pagenumbering{alph}
\maketitle
\pagenumbering{roman}
\setcounter{page}{1}
\tableofcontents
\clearpage
\pagenumbering{arabic}
\setcounter{page}{1}

\section{Einleitung}
\subsection{Zweck der Spezifikation}
Die Spezifikation beschreibt die Anforderungen und Funktionalität vom ToureNPlaner Server. Sie ist die Grundlage für alle weiteren Dokumente; insbesondere muss sie fortlaufend mit dem Entwurf verglichen und bei Bedarf angepasst werden.\\
Bei der Erstellung der Spezifikation wurde bewusst auf die Festlegung von Implementierungsdetails verzichtet, um gewisse Freiheiten in der Umsetzung der Anforderungen zu erhalten.

\subsection{Leserkreis}
Der Leserkreis dieses Dokuments besteht aus folgenden Personengruppen:
\begin{itemize}
\item Die Entwickler der ToureNPlaner Serverkomponente
\item Die Beteiligten des Spezifikationsreviews
\item Der Kunde
\item Die Betreuer des Studienprojekts
\end{itemize}

\subsection{Motivation}
Betrachtet man das Angebot an Karten- oder Graphbasierten Webanwendungen wie OpenStreetMap und Google Maps, so fällt auf, dass
es zahlreiche Angebote zum berechnen eines kürzesten Weges von A nach B gibt, jedoch kaum allgemeinere Funktionen zur
Berechnung von (Rund-) Touren oder kürzesten Wegen unter bestimmten Bedingungen.\\
Da viele der Problemstellungen im Bereich der Graphbasierten Rundtouren NP-schwer sind, braucht es hier desweiteren Approximationsalgorithmen und
nicht zu vernachlässigenden Rechenaufwand.\\
Auch fällt auf, dass viele dieser Algorithmen ähnliche Backend-Daten wie zum Beispiel eine Graphrepräsentation benötigen. Es ist also sinnvoll, ein flexibles Framework zu schaffen, um solcherlei Probleme zu lösen.\\
Dieses sollte dabei sowohl eine Umgebung bieten, in der auch aufwendige Berechnungen durchgeführt werden können, als auch die den Algorithmen gemeinsamen Daten und Interfaces
bereitstellen und bündeln. Hierbei scheint es uns sinnvoll, eine Client-Server Struktur einzuführen, bei der aufwendige Berechnungen auf einem potentiell leistungsstarken Server erfolgen, während die Darstellung und Eingabe der Daten auf Webbasierten und/oder mobilen Clients erfolgt.
%\subsection{Konventionen}

\section{Funktionale Anforderungen}
\subsection{Grundfunktionen}
Der Server stellt den implementierten Algorithmen folgende Supportfunktionen bereit:
\begin{itemize}

	\item Implementierung des ToureNPlaner Protokolls mit HTTP und HTTPS. Siehe Protokollspezifikation.

	\item Geteilte Graphrepräsentation
	\begin{itemize}
		\item Basiert auf OpenStreetMap und einem textbasierten Graphformat, welches bereits erfolgreich in der Abteilung Algorithmik der Universität Stuttgart eingesetzt wird
		\item Threadsicheres abstrahierendes Interface zum Zugriff auf die Graphdaten
	\end{itemize}

  \item Implementierung eines Frameworks für Graphalgorithmen
  \begin{itemize}
	\item Ermöglicht es mit dem in der Protokollspezifikation definierten Protokoll, Probleminstanzen für die implementierten Algorithmen auf dem Server berechnen zu lassen.
	\item Mehrere Probleminstanzen auch verschiedener Algorithmen können gleichzeitig bearbeitet werden (Multithreading)
  \end{itemize}

 \item Breitstellen eines Fehlerlogs

 \item Implementieren einer optionalen Benutzerverwaltung dabei unterscheiden wir öffentliche Server ohne Benutzerverwaltung und private Server mit Benutzerverwaltung.
 \begin{itemize}
	\item Der Server verwaltet eine Verlauf von bearbeiteten Berechnungen, ihre Rechenzeit und Kosten. Für weitere Informationen siehe Protokollspezifikation.
	\item Die Verwaltung von Benutzern erfolgt wie in der Protokollspezifikation beschrieben. Der Server bietet Funktionen zum Anlegen, Ändern, Auflisten und Löschen von Benutzern.
 \end{itemize}
\end{itemize}
\subsection{Algorithmen}
Der Server unterstützt in erster Ausbaustufe mindestens folgende Algorithmen:
\begin{itemize}
 \item Shortest Path (z.B. Dijkstra oder Dijkstra+Contraction Hierachies)
 \item Traveling-Salesmen
 \item Constrained Shortest Path
\end{itemize}

Weitergehende Informationen siehe Algorithmen-Dokumentation 

\section{Nichtfunktionale Anforderungen}
\subsection{Effizienz}
Die Implementierung der Graphrepräsentation sowie der Algorithmen soll auf Effizienz ausgelegt sein.
Insbesondere bedeutet dies auch, dass höhere Abstraktion in den Interfaces der nötigen Performanz
untergeordnet werden kann (wenn dies Aufgrund der eingesetzten Technologien nötig ist). Da Zugriffe auf den Graph in den meisten Algorithmen einen beträchtlichen Teil des Aufwands ausmachen, hat hier
Geschwindigkeit besondere Priorität.\\
Auch in den Algorithmen selbst, sowie in den von Ihnen verwendeten Datenstrukturen (z.B. Priority Queues) ist auf höchst mögliche Effizienz zu achten.
\subsection{Erweiterbarkeit}
Eine weitere wichtige Anforderung ist die gute Erweiterbarkeit des zu entwickelnden Systems um neue Algorithmen, Graphdaten, aber auch um neue Klienten.
Bei den Algorithmen soll es hierbei möglichst einfach zu implementierende Interfaces geben, die neuen Algorithmen die gesamte Funktionalität des Servers,
also sowohl die Erreichbarkeit über das Netzwerk, als auch den Zugriff auf die Graphdaten zugänglich machen.\\
Der Server fungiert hierbei als generisches Framework zur Implementierung verschiedener Graph- und kartenbasierter Algorithmen.
\subsection{Mengengerüst}
\label{Mengengeruest}

\begin{itemize}
 \item Die Algorithmen sollen mindestens auf dem gesamten Deutschlandgraph operieren können (23 Mio Knoten und 48 Mio Kanten)
 \item Anzahl gleichzeitig laufender Berechnungen (Threads)sowie die Warteschlangenlänge ist frei konfigurierbar (mindestens 20)
 \item Unterstützte Benutzerzahl ist mindestens 100 
\end{itemize}

\subsection{Entwurfseinschränkungen}
\subsubsection{Systemumgebung}
\label{Systemumgebung}
Das System wird für Rechner entwickelt mit mindestens 4 GB RAM und 4 Kernen. Jedoch wird darauf geachtet, dass das System gut skaliert und auch auf größeren Rechnern die zusätzliche Leistung nutzen kann.
Als Referenz kann das vom Kunden genannte System „Alge“ dienen, das 24 Kerne sowie 96 GB RAM besitzt.

\begin {itemize}
 \item Nutzt Java SE 6 und höher
 \item Soll auf allen von Java SE unterstützten Systemen laufen, primär auf Unix-basierten Systemen.
\end {itemize}
\subsubsection{Datenhaltung}
\label{datenhaltung}
Benutzerkonten sowie Request-Log Daten werden in einer MySQL-Datenbank gehalten.\\
Die Graphdaten werden zur Laufzeit im RAM gespeichert. Sie werden aus einer Graphdatei geladen siehe Dokument Graphformat.

\subsection{Robustheit}
Durch softwaretechnische Maßnahmen, wie Unit-Tests und einem ausgiebigen Systemtest wird die Robustheit (IEEE Std 610.12(1990)) der Serverkomponente sichergestellt. Der Server soll auch bei falschen Eingaben korrekt funktionieren und passende Fehler ausgeben.
\subsection{Portabilität}
siehe \ref{Systemumgebung}
\subsection{Distributionsform und Installation}
Die Serverapplikation wird als ausführbares „Jar“ ausgeliefert, außerdem werden für ausgewählte Linux Distributionen Pakete bereitgestellt. Zur Installation wird ein MySQL-Server benötigt die Einrichtung der Datenbank wird über mitgelieferte Skripte erleichtert.
\subsection{Sprachunterstützung}
Als Sprache wird Englisch verwendet. Weiter Sprachen werden nicht unterstützt.

\clearpage
\section{Anwendungsfälle (Use-Cases)}
Genauere Informationen zu Requests und ausgeschriebene Fehler finden sich in der Protokollspezifikation. Alle Anwendungsfälle beginnen im normalen Betrieb, dass heißt der Server nimmt Anfragen über das Netz entgegen.
%\begin{figure}[H]
%  \centering
  %TODO
  %\includegraphics[width=\linewidth]{usecases.png}
%  \caption{Use Cases}
%\end{figure}

%\usecase{Name}{Akteur}{Ziel}{Vorbedingungen}{Nachbedingungen}{Normalablauf}{Sonderfälle und dessen Nachbedingungen}

%\sonderfall[optionale letzte Zeile für "-> Schritt ..."]{Nummer}{Bezeichnung}{Ablauf}

%\sondernachbedingung{Bedingungen}

%\usecase{}{}%Name und Akteure
{%Ziel
}{%Vorbedingungen
}{%Nachbedingungen
}{%Normalablauf
}{%Sondernachbedingungen und Sonderfälle
}

%\sonderfall[]{}%Letzte Zeile und Nummer
{%Bezeichnung
}{%Sonderfallablauf
}

%\kurzersonderfall[]{}%Letzte Zeile und Nummer
{%Bezeichnung
}

\usecase{Server-Informationen abfragen}{Client}%Name und Akteure
{%Ziel
Der Client möchte die Server-Informationen über die Protokollversion und die verfügbaren Algorithmen erhalten
}{%Vorbedingungen
\item Der Server ist in Betrieb
\item Der Server hat einen Request über \textit{/info} vom Client erhalten
}{%Nachbedingungen
\item Die Server-Informationen wurden dem Client gesendet
}{%Normalablauf
\item Der Server sendet die aktuellsten Server-Informationen als Response
}{%Sondernachbedingungen und Sonderfälle
\sondernachbedingung{
	\item Der Server führt seine Arbeit wie gewohnt fort
	\item Die Fehlermeldungen wurden ins Log geschrieben}

\sonderfall[Weiter mit normalem Betrieb]{1a}%Letzte Zeile und Nummer
	{%Bezeichnung
	TCP/IP Verbindung abgebrochen
	}{%Sonderfallablauf
	\item Fehlermeldung „TCP/IP Verbindung abgebrochen“ ins Log schreiben}
}



\usecase{Probleminstanz verarbeiten}{Client, Abrechnungs-DB, Benutzer-DB}%Name und Akteure
{Der Client möchte eine Probleminstanz berechnet haben}%Ziel
{%Vorbedingungen
  \item Der Server ist in Betrieb
  \item Der Server hat einen Request über \textit{/alg\$\{suffix\}} vom Client erhalten
}
{%Nachbedingungen
  \item Der Client hat das Berechnungsergebnis erhalten
  \item Der Server hat die Berechnung in der Abrechnungs-DB gelogged
}
{%Normalablauf
  \item Der Server prüft Benutzername und Passwort anhand der Benutzer-DB
  \item Der Server schreibt den zu berechnenden Request als \textit{pending} in die Abrechnungs-DB
  \item Der Server wählt die passende Algorithmusinstanz aus den vorhandenen Algorithmen
  \item Der Server berechnet mit der Algorithmusinstanz die Lösung der Probleminstanz
  \item Der Server trägt die Lösung in die Abrechnungs-DB ein und entfernt die \textit{pending}-Markierung
  \item Der Server sendet die Lösung an den Client
  }
{%Sondernachbedingungen und Sonderfälle
\sondernachbedingung{
	\item Die Fehlermeldungen wurden ins Log geschrieben
	}
	\sonderfall[Weiter mit normalem Betrieb]{1a}
    {Request nicht parsebar}
    {
    \item Fehler wird ins Log geschrieben
  	\item Der Client erhält einen EBADJSON Fehler mit HTTP-Status 400 Bad Request
    }
\sonderfall[Weiter mit normalem Betrieb]{1b}
    {Ungültige Authentifizerung}
    {
    \item Fehler wird ins Log geschrieben
  	\item Der Client erhält einen EAUTH Fehler mit HTTP-Status 401 Unauthorized
    }
\sonderfall[Weiter mit normalem Betrieb]{1c}
    {Server ausgelastet}
    {
    \item Fehler wird ins Log geschrieben
  	\item Der Client erhält einen EBUSY Fehler mit HTTP-Status 503 Service Unavailable
    }
    
\sonderfall[Weiter mit normalem Betrieb]{3a}%Letzte Zeile und Nummer
	{Keine Passende Algorithmusinstanz}%Bezeichnung
  	{
	\item Der Fehler wird ins Log geschrieben
	\item Der Client erhält einen EUNKNOWNALG Fehler
  	}

\sonderfall[Weiter mit normalem Betrieb]{4a}%Letzte Zeile und Nummer
	{Die Berechnung bricht mit einem Fehler ab}%Bezeichnung
  	{
	\item Der Fehler wird ins Log geschrieben
	\item Der Request wird in der Abrechnungs-DB auf \textit{failed} gesetzt und der genaue Fehler eingetragen
	\item Der Client erhält einen ECOMPUTE Fehler
  	}

\sonderfall[Der Server nimmt keine weiteren Anfragen an bis Verbindung zur Datenbank wieder hergestellt.]{*}%Letzte Zeile und Nummer
	{Verbindung zu Abrechnungs-DB oder Benutzer-DB verloren}%Bezeichnung
  	{
	\item Der Fehler wird ins Log geschrieben
	\item Der Client erhält einen EDATABASE Fehler mit HTTP-Status 503 Service Unavailable
  	}

\sonderfall[Weiter mit normalem Betrieb]{**}%Letzte Zeile und Nummer
	{TCP/IP Verbindung abgebrochen}%Bezeichnung
	{
	\item Fehlermeldung wird ins Log geschrieben
	}

}

\usecase{Benutzer registrieren}{Client, Benutzer-DB}%Name und Akteure
{Ein Benutzer soll in der Benutzer-DB registriert werden}%Ziel
{%Vorbedingungen
  \item Der Server ist in Betrieb.
  \item Der Server hat einen Request über \textit{/register} vom Client erhalten
}
{%Nachbedingungen
  \item Benutzer ist in Datenbank eingetragen, aber noch nicht als aktiv gekennzeichnet.
  \item Client wurde benachrichtigt, dass der Benutzer aufgenommen wurde.
}
{%Normalablauf
  \item  Die Werte werden in die Benutzer-DB eingetragen.
  \item  Der Server sendet dem Client die Antwort 200 OK.
}
{%Sondernachbedingungen und Sonderfälle
  \sondernachbedingung{
	\item Benutzer wurde nicht in die Datenbank aufgenommen.
	\item Fehler wurde ins Log geschrieben.
	}
  
	\sonderfall[Weiter mit normalem Betrieb]{1a}
    {Request nicht parsebar}
    {
    \item Fehler wird ins Log geschrieben
  	\item Der Client erhält einen EBADJSON Fehler mit HTTP-Status 400 Bad Request
    }  
  
  \sonderfall[Weiter mit normalen Betrieb]{1b}%Letzte Zeile und Nummer
	  {Benutzer existiert schon}%Bezeichnung
	  {
	  \item Der Fehler wird ins Log geschrieben.
	  \item Dem Client erhält einen EREGISTERED Fehler HTTP-Status 403 forbidden.
	  } 

  \sonderfall[Fehler]{1c}%Letzte Zeile und Nummer
	  {Ungültige Nutzerdaten}%Bezeichnung
	  {
	  \item Der Fehler wird ins Log geschrieben.
	  \item Dem Client wird ein entsprechender Fehler übermittelt.
	  }

  \sonderfall[Der Server nimmt keine weiteren Anfragen an bis Verbindung zur Datenbank wieder hergestellt.]{*}%Letzte Zeile und Nummer
	{Verbindung zu Abrechnungs-DB oder Benutzer-DB verloren}%Bezeichnung
  	{
	\item Der Fehler wird ins Log geschrieben
	\item Der Client erhält einen EDATABASE Fehler mit HTTP-Status 503 Service Unavailable
  	}

\sonderfall[Weiter mit normalem Betrieb]{**}%Letzte Zeile und Nummer
	{TCP/IP Verbindung abgebrochen}%Bezeichnung
	{
	\item Fehlermeldung wird ins Log geschrieben
	}
}

\usecase{Benutzer authentifizieren}{Client, Benutzer-DB}%Name und Akteure
{Benutzer wurde authentifiziert}%Ziel
{%Vorbedingungen
  \item Der Server ist in Betrieb.
  \item Der Server hat einen Request über \textit{/authuser} vom Client
}
{%Nachbedingungen
  \item Der Client hat ein 200 OK erhalten.
}
{%Normalablauf
  \item Der Server prüft Benutzername und Passwort anhand der Benutzer-DB
  \item Schickt an Client einen 200 OK.
}
{%Sondernachbedingungen und Sonderfälle
  \sondernachbedingung{
	\item Client hat den entsprechenden Fehler erhalten.
	\item Fehler steht im Log.
	}
	\sonderfall[Weiter mit normalem Betrieb]{1a}
    {Ungültige Authentifizierung}
    {
    \item Fehler wird ins Log geschrieben
  	\item Der Client erhält einen EAUTH Fehler mit HTTP-Status 401 Unauthorized
    }
	
  
	 \sonderfall[Der Server nimmt keine weiteren Anfragen an bis Verbindung zur Datenbank wieder hergestellt.]{*}%Letzte Zeile und Nummer
	{Verbindung zu Abrechnungs-DB oder Benutzer-DB verloren}%Bezeichnung
  	{
	\item Der Fehler wird ins Log geschrieben
	\item Der Client erhält einen EDATABASE Fehler mit HTTP-Status 503 Service Unavailable
  	}

\sonderfall[Weiter mit normalem Betrieb]{**}%Letzte Zeile und Nummer
	{TCP/IP Verbindung abgebrochen}%Bezeichnung
	{
	\item Fehlermeldung wird ins Log geschrieben
	}
}

\usecase{Benutzerdaten abfragen}{Client, Benutzer-DB}%Name und Akteure
{Client erhält seine Benutzerdaten}%Ziel
{%Vorbedingungen
  \item Der Server ist in Betrieb.
  \item Der Server hat einen Request über \textit{/getuser} vom Client empfangen
}
{%Nachbedingungen
  \item Client hat die Benutzerdaten erhalten.
}
{%Normalablauf
  \item Der Server prüft Benutzername und Passwort anhand der Benutzer-DB
  \item Der Server ruft Benutzerdaten aus der Benutzer-DB ab
  \item Benutzerdaten werden an Client geschickt.
}
{%Sondernachbedingungen und Sonderfälle
  \sondernachbedingung{
	\item Client hat den entsprechenden Fehler erhalten.
	\item Fehler steht im Log.
	}
	
	\sonderfall[Weiter mit normalem Betrieb]{1a}
    {Ungültige Authentifizerung}
    {
    \item Fehler wird ins Log geschrieben
  	\item Der Client erhält einen EAUTH Fehler mit HTTP-Status 401 Unauthorized
    }
    	\sonderfall[Weiter mit normalem Betrieb]{1b}
    {Keine Adminrechte aber ID angegeben}
    {
    \item Fehler wird ins Log geschrieben
  	\item Der Client erhält einen ENOTADMIN Fehler mit HTTP-Status 403 Forbidden
    }
    	\sonderfall[Weiter mit normalem Betrieb]{2a}
    {Benutzer ID nicht vorhanden}
    {
    \item Fehler wird ins Log geschrieben
  	\item Der Client erhält einen ENOID Fehler mit HTTP-Status 404 Not Found
    }
  \sonderfall[Der Server nimmt keine weiteren Anfragen an bis Verbindung zur Datenbank wieder hergestellt.]{*}%Letzte Zeile und Nummer
	{Verbindung zu Abrechnungs-DB oder Benutzer-DB verloren}%Bezeichnung
  	{
	\item Der Fehler wird ins Log geschrieben
	\item Der Client erhält einen EDATABASE Fehler mit HTTP-Status 503 Service Unavailable
  	}

\sonderfall[Weiter mit normalem Betrieb]{**}%Letzte Zeile und Nummer
	{TCP/IP Verbindung abgebrochen}%Bezeichnung
	{
	\item Fehlermeldung wird ins Log geschrieben
	}
}

\usecase{Benutzerdaten verändern}{Client, Benutzer-DB}%Name und Akteure
{Client verändert seine Benutzerdaten}%Ziel
{%Vorbedingungen
  \item Der Server ist in Betrieb.
  \item Der Server hat einen Request mit optionalem id-Parameter über \textit{/updateuser} vom Client empfangen
}
{%Nachbedingungen
  \item Benutzerdaten wurden geändert.
}
{%Normalablauf
  \item Der Server prüft Benutzername und Passwort anhand der Benutzer-DB
  \item Geänderte Benutzerdaten werden in der Datenbank verändert
}
{%Sondernachbedingungen und Sonderfälle
  \sondernachbedingung{
	\item Client hat den entsprechenden Fehler erhalten.
	\item Fehler steht im Log.
	}
	
	\sonderfall[Weiter mit normalem Betrieb]{1a}
    {Ungültige Authentifizerung}
    {
    \item Fehler wird ins Log geschrieben
  	\item Der Client erhält einen EAUTH Fehler mit HTTP-Status 401 Unauthorized
    }
     	\sonderfall[Weiter mit normalem Betrieb]{1b}
    {Keine Adminrechte aber ID angegeben}
    {
    \item Fehler wird ins Log geschrieben
  	\item Der Client erhält einen ENOTADMIN Fehler mit HTTP-Status 403 Forbidden
    }
    	\sonderfall[Weiter mit normalem Betrieb]{2a}
    {Benutzer ID nicht vorhanden}
    {
    \item Fehler wird ins Log geschrieben
  	\item Der Client erhält einen ENOID Fehler mit HTTP-Status 404 Not Found
    }
   	
  \sonderfall[Der Server nimmt keine weiteren Anfragen an bis Verbindung zur Datenbank wieder hergestellt.]{*}%Letzte Zeile und Nummer
	{Verbindung zu Abrechnungs-DB oder Benutzer-DB verloren}%Bezeichnung
  	{
	\item Der Fehler wird ins Log geschrieben
	\item Der Client erhält einen EDATABASE Fehler mit HTTP-Status 503 Service Unavailable
  	}

\sonderfall[Weiter mit normalem Betrieb]{**}%Letzte Zeile und Nummer
	{TCP/IP Verbindung abgebrochen}%Bezeichnung
	{
	\item Fehlermeldung wird ins Log geschrieben
	}
}


\usecase{Abfragenverlauf abrufen}{Client, Abrechnungs-DB, Benutzer-DB}%Name und Akteure
{Die in der Benutzer-DB gespeicherten Abfragen des Benutzers sowie deren Ergebnisse (bzw. Status) und Kosten sollen angezeigt werden}%Ziel
{%Vorbedingungen
  \item Der Server hat einen Request an /listrequests erhalten
}
{%Nachbedingungen
  \item Die entsprechenden Informationen wurden gesendet
}
{%Normalablauf
  \item Der Server prüft Benutzername und Passwort anhand der Benutzer-DB
  \item Der Server liest aus der Abrechnungs-DB alle durch die offset- und limit-Parameter vom Benutzer berechneten Requests sowie Zahlungsinformationen zu diesen Requests aus 
  \item Der Server schickt dem Client eine Liste seiner bisher berechneten und \textit{pending} Requests sowie Zahlungsinformationen
}
{%Sondernachbedingungen und Sonderfälle
  \sondernachbedingung{
	\item Client hat den entsprechenden Fehler erhalten.
	\item Fehler steht im Log.
	}
		\sonderfall[Weiter mit normalem Betrieb]{1a}
    {Ungültige Authentifizerung}
    {
    \item Fehler wird ins Log geschrieben
  	\item Der Client erhält einen EAUTH Fehler mit HTTP-Status 401 Unauthorized
    }
         	\sonderfall[Weiter mit normalem Betrieb]{1b}
    {Keine Adminrechte aber ID angegeben}
    {
    \item Fehler wird ins Log geschrieben
  	\item Der Client erhält einen ENOTADMIN Fehler mit HTTP-Status 403 Forbidden
    }
    	\sonderfall[Weiter mit normalem Betrieb]{2a}
    {Benutzer ID nicht vorhanden}
    {
    \item Fehler wird ins Log geschrieben
  	\item Der Client erhält einen ENOID Fehler mit HTTP-Status 404 Not Found
    }
    	\sonderfall[Weiter mit normalem Betrieb]{2b}%Letzte Zeile und Nummer
	{Ungültiges Offset}%Bezeichnung
  	{
	\item Der Fehler wird ins Log geschrieben
	\item Der Client erhält einen EOFFSET Fehler mit HTTP-Status 403 Forbidden
  	}
	\sonderfall[Weiter mit normalem Betrieb]{2c}%Letzte Zeile und Nummer
	{Ungültiges Limit}%Bezeichnung
  	{
	\item Der Fehler wird ins Log geschrieben
	\item Der Client erhält einen ELIMIT Fehler mit HTTP-Status 403 Forbidden
  	}
  \sonderfall[Der Server nimmt keine weiteren Anfragen an bis Verbindung zur Datenbank wieder hergestellt.]{*}%Letzte Zeile und Nummer
	{Verbindung zu Abrechnungs-DB oder Benutzer-DB verloren}%Bezeichnung
  	{
	\item Der Fehler wird ins Log geschrieben
	\item Der Client erhält einen EDATABASE Fehler mit HTTP-Status 503 Service Unavailable
  	}

\sonderfall[Weiter mit normalem Betrieb]{**}%Letzte Zeile und Nummer
	{TCP/IP Verbindung abgebrochen}%Bezeichnung
	{
	\item Fehlermeldung wird ins Log geschrieben
	}
}

\usecase{Benutzerliste abfragen}{Client, Benutzer-DB}%Name und Akteure
{Eine Liste der Benutzer wird abgefragt}%Ziel
{%Vorbedingungen
  \item Der Server hat einen Get-Request an /listusers erhalten
}
{%Nachbedingungen
  \item Die entsprechenden Informationen wurden gesendet
}
{%Normalablauf
  \item Der Server prüft Benutzername und Passwort anhand der Benutzer-DB
  \item Der Server liest aus der Benutzer-DB alle durch die offset- und limit-Parameter ausgewählten Benutzer-Informationen aus
  \item Der Server schickt dem Client die ausgelesenen Informationen
}
{%Sondernachbedingungen und Sonderfälle
  \sondernachbedingung{
	\item Client hat den entsprechenden Fehler erhalten.
	\item Fehler steht im Log.
	}
		\sonderfall[Weiter mit normalem Betrieb]{1a}
    {Ungültige Authentifizerung}
    {
    \item Fehler wird ins Log geschrieben
  	\item Der Client erhält einen EAUTH Fehler mit HTTP-Status 401 Unauthorized
    }
     	\sonderfall[Weiter mit normalem Betrieb]{1b}
    {Keine Adminrechte}
    {
    \item Fehler wird ins Log geschrieben
  	\item Der Client erhält einen ENOTADMIN Fehler mit HTTP-Status 403 Forbidden
    }
	\sonderfall[Weiter mit normalem Betrieb]{2a}%Letzte Zeile und Nummer
	{Ungültiges Offset}%Bezeichnung
  	{
	\item Der Fehler wird ins Log geschrieben
	\item Der Client erhält einen EOFFSET Fehler mit HTTP-Status 403 Forbidden
  	}
	\sonderfall[Weiter mit normalem Betrieb]{2b}%Letzte Zeile und Nummer
	{Ungültiges Limit}%Bezeichnung
  	{
	\item Der Fehler wird ins Log geschrieben
	\item Der Client erhält einen ELIMIT Fehler mit HTTP-Status 403 Forbidden
  	}
  	    
  \sonderfall[Der Server nimmt keine weiteren Anfragen an bis Verbindung zur Datenbank wieder hergestellt.]{*}%Letzte Zeile und Nummer
	{Verbindung zu Abrechnungs-DB oder Benutzer-DB verloren}%Bezeichnung
  	{
	\item Der Fehler wird ins Log geschrieben
	\item Der Client erhält einen EDATABASE Fehler mit HTTP-Status 503 Service Unavailable
  	}

\sonderfall[Weiter mit normalem Betrieb]{**}%Letzte Zeile und Nummer
	{TCP/IP Verbindung abgebrochen}%Bezeichnung
	{
	\item Fehlermeldung wird ins Log geschrieben
	}
}



\usecase{Benutzer löschen}{Client, Benutzer-DB}%Name und Akteure
{Das Benutzerkonto eines Kunden soll aus der Benutzer-DB gelöscht werden}%Ziel
{%Vorbedingungen
  \item Der Server hat einen Get-Request an /delete erhalten
}
{%Nachbedingungen
  \item Der gelöschte Benutzer ist tatsächlich gelöscht
}
{%Normalablauf
  \item Der Server prüft Benutzername und Passwort anhand der Benutzer-DB
  \item Der Server, der den Request erhält, löscht den Benutzer aus der Benutzer-DB.
  %TODO: rechtliche implikationen, nachvollziehbarkeit?
  %\item Die laufenden Jobs des gelöschten Benutzers werden abgebrochen/doch noch ausgeführt
}
{%Sondernachbedingungen und Sonderfälle
  \sondernachbedingung{
	\item Client hat den entsprechenden Fehler erhalten.
	\item Fehler steht im Log.
	}
		\sonderfall[Weiter mit normalem Betrieb]{1a}
    {Ungültige Authentifizerung}
    {
    \item Fehler wird ins Log geschrieben
  	\item Der Client erhält einen EAUTH Fehler mit HTTP-Status 401 Unauthorized
    }
    
  	\sonderfall[Weiter mit normalem Betrieb]{1b}
    {Keine Adminrechte}
    {
    \item Fehler wird ins Log geschrieben
  	\item Der Client erhält einen ENOTADMIN Fehler mit HTTP-Status 403 Forbidden
    }
    
  	\sonderfall[Weiter mit normalem Betrieb]{1c}
    {Benutzer ID nicht vorhanden}
    {
    \item Fehler wird ins Log geschrieben
  	\item Der Client erhält einen ENOID Fehler mit HTTP-Status 404 Not Found
    }

  \sonderfall[Der Server nimmt keine weiteren Anfragen an bis Verbindung zur Datenbank wieder hergestellt.]{*}%Letzte Zeile und Nummer
	{Verbindung zu Abrechnungs-DB oder Benutzer-DB verloren}%Bezeichnung
  	{
	\item Der Fehler wird ins Log geschrieben
	\item Der Client erhält einen EDATABASE Fehler mit HTTP-Status 503 Service Unavailable
  	}

\sonderfall[Weiter mit normalem Betrieb]{**}%Letzte Zeile und Nummer
	{TCP/IP Verbindung abgebrochen}%Bezeichnung
	{
	\item Fehlermeldung wird ins Log geschrieben
	}
}

\end{document}
