\documentclass[a4paper,10pt,titlepage]{article}

\usepackage[utf8]{inputenc}
\usepackage[ngerman]{babel}
\usepackage{graphicx}
\usepackage{a4wide}
\usepackage{url}
\usepackage{booktabs}
\usepackage{float}

\usepackage[pdfborder={0 0 0}]{hyperref}

\graphicspath{{./grafiken/}}

\author{Christoph Haag, Peter Vollmer, Sascha Meusel, Niklas Schnelle}
\title{Spezifikation ToureNPlaner Server\\ Stupro 2011/2012}

\newcommand{\shortcut}[1]
{\texttt{#1}}

\newcommand{\usecase}[8]
{\subsection{#1}
\begin{tabular}{|p{0.2\textwidth}|p{0.7\textwidth}|}
\hline
  Akteur & #2\\\hline
  Ziel & #3\\\hline
  Vorbedingung & #4\\\hline
  Normalablauf & #7\\\hline
  Nachbedingung & #5\\\hline
  Sonderfall & #8\\\hline
  Nachbedingung im Sonderfall& #6\\\hline
 \end{tabular}
}

\newcommand{\begriff}[7]
{\subsection{#1}
\begin{tabular}{|p{0.2\textwidth}|p{0.7\textwidth}|}
\hline
  Bedeutung & #2\\\hline
  Abgrenzung & #3\\\hline
  Gültigkeit & #4\\\hline
  Bezeichnung & #5\\\hline
  Unklarheiten & #6\\\hline
  Querverweise & #7\\\hline
 \end{tabular}}

\newcommand{\doaction}[1]
{Der Benutzer führt die Aktion „\nameref{#1}“ (\ref{#1}) aus.}

\begin{document}

\pagenumbering{alph}
\maketitle
\pagenumbering{roman}
\setcounter{page}{1}
\tableofcontents
\clearpage
\pagenumbering{arabic}
\setcounter{page}{1}

\section{Einleitung}
\subsection{Zweck der Spezifikation}
Die Spezifikation beschreibt die Anforderungen und Funktionalität von Tourenplaner. Sie ist die Grundlage für alle weiteren Dokumente; insbesondere muss sie fortlaufend mit dem Entwurf verglichen und bei Bedarf angepasst werden.\\
Bei der Erstellung der Spezifikation wurde bewusst auf die Festlegung von Implementierungsdetails verzichtet, um gewisse Freiheiten in der Umsetzung der Anforderungen zu erhalten.

\subsection{Leserkreis}
Der Leserkreis dieses Dokuments besteht aus folgenden Personengruppen:
\begin{itemize}
\item Die Entwickler der ToureNPlaner Serverkomponente
\item Die Beteiligten des Spezifikationsreviews
\item Der Kunde
\item Die Betreuer des Studienprojekts
\end{itemize}

\subsection{Motivation}
Betrachtet man das Angebot an Karten- oder Graphbasierten Webanwendungen wie OpenStreetMap und Google Maps, so fällt auf, dass
es zahlreiche Angebote zum berechnen eines kürzesten Weges von A nach B gibt, jedoch kaum allgemeinere Funktionen zur
Berechnung von (Rund-)Touren oder kürzesten Wegen unter bestimmten Bedingungen.\\
Da viele der Problemstellungen im Bereich der Graphbasierten Rundtouren NP-schwer sind, braucht es hier desweiteren Approximationsalgorithmen und
nicht zu vernachlässigenden Rechenaufwand.\\
Auch fällt auf, dass viele dieser Algorithmen ähnliche Backend-Daten wie zum Beispiel eine Graphrepräsentation benötigen. Es ist also sinnvoll, ein flexibles Framework zu schaffen, um solcherlei Probleme zu lösen.\\
Dieses sollte dabei sowohl eine Umgebung bieten, in der auch aufwendige Berechnungen durchgeführt werden können, als auch die den Algorithmen gemeinsamen Daten und Interfaces
bereitstellen und zu bündeln. Hierbei scheint es uns sinnvoll, eine Client-Server Struktur einzuführen, bei der aufwendige Berechnungen auf einem potentiell leistungsstarken Server erfolgen, während die Darstellung und Eingabe der Daten auf Webbasierten und/oder Mobilen Clients erfolgt.
%\subsection{Konventionen}

\section{Funktionale Anforderungen}
\subsection{Grundfunktionen}
Der Server stellt den implementierten Algorithmen folgende Supportfunktionen bereit:
\begin{itemize}
 \item Geteilte Graphrepräsentation
  \begin{itemize}
   \item 
    Basiert auf OpenStreetMap und einem textbasierten Graphformat, welches 
    bereits erfolgreich in der Abteilung Algorithmik der Universität Stuttgart eingesetzt wird
   \item
    Threadsicheres abstrahierendes Interface zum Zugriff auf die Graphdaten
  \end{itemize}
  \item Bereitstellung eines allgemeinen Interfaces für Algorithmen welches diese implementieren
  \begin{itemize}
   \item Ermöglicht es, mehrere Algorithmeninstanzen auch verschiedener Algorithmen gleichzeitig zu bearbeiten (Multithreading)
   \item Bietet Algorithmen die Möglichkeit, Probleminstanzen als JSON von Netzwerkclienten zu erhalten (siehe Protokollspezifikation)
  \end{itemize}
 \item Implementierung eines HTTP-basierten Netzwerkservers, der JSON-Anfragen entgegennimmt und auf das Algorithmeninterface abbildet, sowie
 die Ergebnisse des jeweiligen Algorithmus an den Client zurück sendet.
\end{itemize}
\subsection{Algorithmen}
Der Server unterstützt in erster Ausbaustufe mindestens folgende Algorithmen:
\begin{itemize}
 \item Shortest Path (z.B. Dijkstra oder Dijkstra+Contraction Hiearchies)
 \item Traveling-Salesmen
 \item Constrained Shortest Path
\end{itemize}


\section{Nichtfunktionale Anforderungen}
\subsection{Effizienz}
Die Implementierung der Graphrepräsentation sowie der Algorithmen soll auf Effizienz ausgelegt sein.
Insbesondere bedeutet dies auch, dass höhere Abstraktion in den Interfaces der nötigen Performanz
untergeordnet werden kann (wenn dies Aufgrund der eingesetzten Technologien nötig ist). Da Zugriffe auf den Graph in den meisten Algorithmen einen beträchtlichen Teil des Aufwands ausmachen, hat hier
Geschwindigkeit besondere Priorität.\\
Auch in den Algorithmen selbst, sowie in den von Ihnen verwendeten Datenstrukturen (z.B. Priority Queues) ist auf höchst mögliche Effizienz zu achten.
\subsection{Erweiterbarkeit}
Eine weitere wichtige Anforderung ist die gute Erweiterbarkeit des zu entwickelnden Systems, um neue Algorithmen, Graphdaten, aber auch um neue Clients.
Bei den Algorithmen soll es hierbei möglichst einfach zu implementierende Interfaces geben, die neuen Algorithmen die gesammte Funktionalität des Servers,
also sowohl die Erreichbarkeit über das Netzwerk, als auch den Zugriff auf die Graphdaten zugänglich machen.\\
Der Server fungiert hierbei als generisches Framework zur Implementierung verschiedener Graph- und Kartenbasierter Algorithmen.
\subsection{Mengengerüst}
\label{Mengengeruest}
Die Angaben sind gültig für das Zielsystem "Alge", das 24 Kerne sowie 96 Gigabyte Arbeitsspeicher besitzt.
\begin{itemize}
 \item Es soll mindestens beliebig auf dem gesamten Deutschlandgraph operiert werden können
 \item Anzahl gleichzeitig laufender Berechnungen sowie die Warteschlangenlänge ist frei konfigurierbar
\end{itemize}

\subsection{Entwurfseinschränkungen}
\subsubsection{Systemumgebung}

\subsubsection{Layout und Gestaltung}

\subsubsection{Datenhaltung}
\label{datenhaltung}

\subsection{Robustheit}

\subsection{Portabilität}

\subsection{Erweiterbarkeit}

\subsection{Distributionsform und Installation}

\subsection{Sprachunterstützung}

\clearpage
\section{Benutzeroberfläche}


\clearpage
\section{Anwendungsfälle (Use-Cases)}
\begin{figure}[H]
  \centering
  %TODO
  %\includegraphics[width=\linewidth]{usecases.png}
  \caption{Task}
\end{figure}

%\usecase{Name}{Akteur}{Ziel}{Vorbedingung}{Nachbedingung}{Nachbedingung im Sonderfall}{Normalablauf}{Sonderfall}


\clearpage
\appendix
\section{Begriffslexikon}

%\begriff{Begriff}{Bedeutung}{Abgrenzung}{Gültigkeit}{Bezeichnung}{Unklarheiten}{Querverweise}

\end{document}
