%server specification
\documentclass[a4paper,10pt,titlepage]{article}

\usepackage[utf8]{inputenc}
\usepackage[ngerman]{babel}
\usepackage{graphicx}
\usepackage{a4wide}
\usepackage{url}
\usepackage{booktabs}
\usepackage{float}
\usepackage{paralist}
\usepackage{array}
\usepackage{ifthen}

\usepackage[pdfborder={0 0 0}]{hyperref}

\graphicspath{{./grafiken/}}

\author{Christoph Haag, Peter Vollmer, Sascha Meusel, Niklas Schnelle}
\title{Spezifikation ToureNPlaner Server\\ Stupro 2011/2012}

\newcommand{\shortcut}[1]
{\texttt{#1}}

\makeatletter
\newcommand\novspace{\@minipagetrue}

\newenvironment{owncompactitem}{%
\compactitem
}{%
\@finalstrut\@arstrutbox
\@nameuse{endcompactitem}%
\aftergroup\let\aftergroup\@finalstrut\aftergroup\@gobble
}
\newenvironment{owncompactenum}{%
\compactenum
}{%
\@finalstrut\@arstrutbox
\@nameuse{endcompactenum}%
\aftergroup\let\aftergroup\@finalstrut\aftergroup\@gobble
}
\makeatother

\newcommand{\usecase}[7]
{\subsection{#1}
\setlength{\extrarowheight}{2pt}
\begin{tabular}{|p{0.2\textwidth}|p{0.9\textwidth}|}
\hline
  Akteure & #2\\\hline
  Ziel & #3\\\hline
  Vorbedingungen & \novspace
  	\begin{owncompactitem}[-] #4 \end{owncompactitem} \\\hline
  Normalablauf & \vspace{-7pt}
  	\begin{owncompactenum}[1.] #6 \end{owncompactenum} \\\hline
  Nachbedingungen & \novspace
  	\begin{owncompactitem}[-] #5 \end{owncompactitem} \\\hline
  #7
\end{tabular}
}

\newcommand{\sonderfall}[4][\empty]
{
Sonderfall #2 & \vspace{-10pt}
	\textit{#3}
	\begin{owncompactenum}[{#2}.1] {#4} \end{owncompactenum}
  	\ifthenelse{\equal{#1}{\empty}}
    	{\\\hline} %no opt
    	{\ensuremath{\rightarrow} #1 \\ [+1pt] \hline} %opt

}

\newcommand{\kurzersonderfall}[3][\empty]
{
Sonderfall #2 & \vspace{-10pt}
	\textit{#3}
  	\ifthenelse{\equal{#1}{\empty}}
    	{\\\hline} %no opt
    	{\\&\ensuremath{\rightarrow} #1 \\ [+1pt] \hline} %opt

}

\newcommand{\sondernachbedingung}[1]
{
Nachbedingungen im Sonderfall& \novspace
	\begin{owncompactitem}[-]
		#1
	\end{owncompactitem} \\\hline
}


\newcommand{\begriff}[7]
{\subsection{#1}
\begin{tabular}{|p{0.2\textwidth}|p{0.7\textwidth}|}
\hline
  Bedeutung & #2\\\hline
  Abgrenzung & #3\\\hline
  Gültigkeit & #4\\\hline
  Bezeichnung & #5\\\hline
  Unklarheiten & #6\\\hline
  Querverweise & #7\\\hline
 \end{tabular}}

\newcommand{\doaction}[1]
{Der Benutzer führt die Aktion „\nameref{#1}“ (\ref{#1}) aus.}

\begin{document}

\pagenumbering{alph}
\maketitle
\pagenumbering{roman}
\setcounter{page}{1}
\tableofcontents
\clearpage
\pagenumbering{arabic}
\setcounter{page}{1}

\section{Einleitung}
\subsection{Zweck der Spezifikation}
Die Spezifikation beschreibt die Anforderungen und Funktionalität von Tourenplaner. Sie ist die Grundlage für alle weiteren Dokumente; insbesondere muss sie fortlaufend mit dem Entwurf verglichen und bei Bedarf angepasst werden.\\
Bei der Erstellung der Spezifikation wurde bewusst auf die Festlegung von Implementierungsdetails verzichtet, um gewisse Freiheiten in der Umsetzung der Anforderungen zu erhalten.

\subsection{Leserkreis}
Der Leserkreis dieses Dokuments besteht aus folgenden Personengruppen:
\begin{itemize}
\item Die Entwickler der ToureNPlaner Serverkomponente
\item Die Beteiligten des Spezifikationsreviews
\item Der Kunde
\item Die Betreuer des Studienprojekts
\end{itemize}

\subsection{Motivation}
Betrachtet man das Angebot an Karten- oder Graphbasierten Webanwendungen wie OpenStreetMap und Google Maps, so fällt auf, dass
es zahlreiche Angebote zum berechnen eines kürzesten Weges von A nach B gibt, jedoch kaum allgemeinere Funktionen zur
Berechnung von (Rund-)Touren oder kürzesten Wegen unter bestimmten Bedingungen.\\
Da viele der Problemstellungen im Bereich der Graphbasierten Rundtouren NP-schwer sind, braucht es hier desweiteren Approximationsalgorithmen und
nicht zu vernachlässigenden Rechenaufwand.\\
Auch fällt auf, dass viele dieser Algorithmen ähnliche Backend-Daten wie zum Beispiel eine Graphrepräsentation benötigen. Es ist also sinnvoll, ein flexibles Framework zu schaffen, um solcherlei Probleme zu lösen.\\
Dieses sollte dabei sowohl eine Umgebung bieten, in der auch aufwendige Berechnungen durchgeführt werden können, als auch die den Algorithmen gemeinsamen Daten und Interfaces
bereitstellen und bündeln. Hierbei scheint es uns sinnvoll, eine Client-Server Struktur einzuführen, bei der aufwendige Berechnungen auf einem potentiell leistungsstarken Server erfolgen, während die Darstellung und Eingabe der Daten auf Webbasierten und/oder Mobilen Clients erfolgt.
%\subsection{Konventionen}

\section{Funktionale Anforderungen}
\subsection{Grundfunktionen}
Der Server stellt den implementierten Algorithmen folgende Supportfunktionen bereit:
\begin{itemize}
 \item Geteilte Graphrepräsentation
  \begin{itemize}
   \item 
    Basiert auf OpenStreetMap und einem textbasierten Graphformat, welches 
    bereits erfolgreich in der Abteilung Algorithmik der Universität Stuttgart eingesetzt wird
   \item
    Threadsicheres abstrahierendes Interface zum Zugriff auf die Graphdaten
  \end{itemize}
  \item Bereitstellung eines allgemeinen Interfaces für Algorithmen welches diese implementieren
  \begin{itemize}
   \item Ermöglicht es, mehrere Algorithmeninstanzen auch verschiedener Algorithmen gleichzeitig zu bearbeiten (Multithreading)
   \item Bietet Algorithmen die Möglichkeit, Probleminstanzen als JSON von Netzwerkclienten zu erhalten (siehe Protokollspezifikation)
  \end{itemize}
 \item Implementierung eines HTTP-basierten Netzwerkservers, der JSON-Anfragen entgegennimmt und auf das Algorithmeninterface abbildet, sowie
 die Ergebnisse des jeweiligen Algorithmus an den Client zurück sendet.
\end{itemize}
\subsection{Algorithmen}
Der Server unterstützt in erster Ausbaustufe mindestens folgende Algorithmen:
\begin{itemize}
 \item Shortest Path (z.B. Dijkstra oder Dijkstra+Contraction Hiearchies)
 \item Traveling-Salesmen
 \item Constrained Shortest Path
\end{itemize}


\section{Nichtfunktionale Anforderungen}
\subsection{Effizienz}
Die Implementierung der Graphrepräsentation sowie der Algorithmen soll auf Effizienz ausgelegt sein.
Insbesondere bedeutet dies auch, dass höhere Abstraktion in den Interfaces der nötigen Performanz
untergeordnet werden kann (wenn dies Aufgrund der eingesetzten Technologien nötig ist). Da Zugriffe auf den Graph in den meisten Algorithmen einen beträchtlichen Teil des Aufwands ausmachen, hat hier
Geschwindigkeit besondere Priorität.\\
Auch in den Algorithmen selbst, sowie in den von Ihnen verwendeten Datenstrukturen (z.B. Priority Queues) ist auf höchst mögliche Effizienz zu achten.
\subsection{Erweiterbarkeit}
Eine weitere wichtige Anforderung ist die gute Erweiterbarkeit des zu entwickelnden Systems, um neue Algorithmen, Graphdaten, aber auch um neue Clients.
Bei den Algorithmen soll es hierbei möglichst einfach zu implementierende Interfaces geben, die neuen Algorithmen die gesammte Funktionalität des Servers,
also sowohl die Erreichbarkeit über das Netzwerk, als auch den Zugriff auf die Graphdaten zugänglich machen.\\
Der Server fungiert hierbei als generisches Framework zur Implementierung verschiedener Graph- und Kartenbasierter Algorithmen.
\subsection{Mengengerüst}
\label{Mengengeruest}

\begin{itemize}
 \item Die Algorithmen sollen mindestens auf dem gesamten Deutschlandgraph operieren können
 \item Anzahl gleichzeitig laufender Berechnungen sowie die Warteschlangenlänge ist frei konfigurierbar
\end{itemize}

\subsection{Entwurfseinschränkungen}
\subsubsection{Systemumgebung}
Das System wird für Rechner entwickelt mit mindestens 4 GB RAM und 4 Kernen. Jedoch wird darauf geachtet, dass das System gut skaliert und auch auf größeren Rechnern die zusätzliche Leistung nutzen kann.
Als Referenz kann das System ''Alge" dienen, das 24 Kerne sowie 96 Gigabyte Arbeitsspeicher besitzt.

\begin {itemize}
 \item Nutzt Java Runtime Environment 6 and up
 \item Soll auf allen von Java SE unterstützten Systemen laufen, primär auf Unix-basierten Systemen.
\end {itemize}
\subsubsection{Datenhaltung}
\label{datenhaltung}
Benutzerkonten sowie Request-Log Daten werden in einer MySQL-Datenbank gehalten.\\
Die Graphdaten werden zur Laufzeit im RAM gespeichert. Sie werden aus einer Graphdatei geladen siehe Dokument Graphformat.

\subsection{Robustheit}
Durch softwaretechnische Maßnahmen, wie Unit-Tests und einem ausgiebigen Systemtest wird die Robustheit der Serverkomponente sichergestellt.
\subsection{Portabilität}
siehe Systemumgebung
\subsection{Erweiterbarkeit}
Wie im Abschnitt Funktionale Anforderungen beschrieben, stellt der Server eine Schnittstelle für die Implementierung von Graphalgorithmen bereit.
\subsection{Distributionsform und Installation}
Die Serverapplikation wird als ausführbares "`Jar"' ausgeliefert, außerdem werden für ausgewählte Linux Distributionen Pakete bereitgestellt. Zur Installation wird ein MySQL-Server benötigt die Einrichtung der Datenbank wird über mitgelieferte Skripte erleichtert.
\subsection{Sprachunterstützung}
Als Sprache wird Englisch unterstützt. 


\clearpage
\section{Benutzeroberfläche}
Als Benutzeroberfläche dient eine spezielle Administrationsansicht im Web-Client.

\clearpage
\section{Anwendungsfälle (Use-Cases)}
\begin{figure}[H]
  \centering
  %TODO
  %\includegraphics[width=\linewidth]{usecases.png}
  \caption{Use Cases}
\end{figure}

%\usecase{Name}{Akteur}{Ziel}{Vorbedingungen}{Nachbedingungen}{Normalablauf}{Sonderfälle und dessen Nachbedingungen}

%\sonderfall[optionale letzte Zeile für "-> Schritt ..."]{Nummer}{Bezeichnung}{Ablauf}

%\sondernachbedingung{Bedingungen}

%\usecase{}{}%Name und Akteure
{%Ziel
}{%Vorbedingungen
}{%Nachbedingungen
}{%Normalablauf
}{%Sondernachbedingungen und Sonderfälle
}

%\sonderfall[]{}%Letzte Zeile und Nummer
{%Bezeichnung
}{%Sonderfallablauf
}

%\kurzersonderfall[]{}%Letzte Zeile und Nummer
{%Bezeichnung
}

\usecase{Server-Informationen abfragen}{Client}%Name und Akteure
{%Ziel
Der Client möchte die Server-Informationen über die Protokollversion und die verfügbaren Algorithmen erhalten
}{%Vorbedingungen
\item Der Server ist in Betrieb
\item Der Server hat einen Request über \textit{/info} vom Client erhalten
}{%Nachbedingungen
\item Die Server-Informationen wurden dem Client gesendet
}{%Normalablauf
\item Der Server holt sich die aktuellsten Server-Informationen
\item Der Server sendet die aktuellsten Server-Informationen als Response
}{%Sondernachbedingungen und Sonderfälle
\sondernachbedingung{
	\item Der Server führt seine Arbeit wie gewohnt fort
	\item Die Fehlermeldungen wurden ins Log geschrieben}

\sonderfall{2a}%Letzte Zeile und Nummer
	{%Bezeichnung
	TCP/IP Verbindung abgebrochen
	}{%Sonderfallablauf
	\item Fehlermeldung "`TCP/IP Verbindung abgebrochen"' ins Log schreiben}
}



\usecase{Probleminstanz bearbeiten}{Client, Abrechnungs-DB}%Name und Akteure
{Der Client möchte eine Probleminstanz berechnet haben}%Ziel
{%Vorbedingungen
  \item Der Server ist in Betrieb
  \item Der Server hat einen Request über \textit{/algsuffix} vom Client erhalten
}
{%Nachbedingungen
  \item Der Client hat das Berechnungsergebnis erhalten
  \item Der Server hat die Berechnung in der Abrechnungs-DB gelogged
}
{%Normalablauf
  \item Der Server schreibt den zu berechnenden Request als \textit{pending} in die Abrechnungs-DB
  \item Der Server wählt die passende Algorithmusinstanz aus den vorhandenen Algorithmen
  \item Der Server berechnet mit der Algorithmusinstanz die Lösung der Probleminstanz
  \item Der Server trägt die Lösung in die Abrechnungs-DB ein (nun nicht mehr \textit{pending})
  \item Der Server sendet die Lösung an den Client
  }
{%Sondernachbedingungen und Sonderfälle
\sondernachbedingung{
	\item Der Server führt seine Arbeit wie gewohnt fort, solange kein kritischer Fehler aufgetreten ist
	\item Die Fehlermeldungen wurden ins Log geschrieben
	}
\sonderfall[Kritischer Fehler, Server ist beendet]{1a}%Letzte Zeile und Nummer
	{Verbindung zu Abrechnungs-DB verloren}%Bezeichnung
  	{
	\item Der Fehler "`Verbindung zu Abrechnungs-DB verloren"' wird ins Log geschrieben (als schwerwiegender Fehler)
	\item Der Client erhält eine entsprechende Fehlermeldung
	\item Der Server wird beendet
  	}

\kurzersonderfall[Weiter mit Schritt 5a.1]{1a.2a}%Letzte Zeile und Nummer
	{TCP/IP Verbindung abgebrochen}%Bezeichnung

\sonderfall[Weiter mit normalem Betrieb]{2a}%Letzte Zeile und Nummer
	{Keine Passende Algorithmusinstanz}%Bezeichnung
  	{
	\item Der Fehler "`Keine Passende Algorithmusinstanz"' wird ins Log geschrieben
	\item Der Client erhält eine entsprechende Fehlermeldung
  	}

\kurzersonderfall[Weiter mit Schritt 5a.1]{2a.2a}%Letzte Zeile und Nummer
	{TCP/IP Verbindung abgebrochen}%Bezeichnung

\sonderfall[Weiter mit normalem Betrieb]{3a}%Letzte Zeile und Nummer
	{Die Berechnung bricht mit einem Fehler ab}%Bezeichnung
  	{
	\item Der Fehler "`Berechnung wegen Fehler abgebrochen"' wird ins Log geschrieben
	\item Der Request wird in der Abrechnungs-DB auf \textit{failed} gesetzt und der genaue Fehler eingetragen
	\item Der Client erhält eine entsprechende Fehlermeldung
  	}

\kurzersonderfall[Weiter mit Schritt 1a.1]{4a}%Letzte Zeile und Nummer
	{Verbindung zu Abrechnungs-DB verloren}%Bezeichnung

\sonderfall[Weiter mit normalem Betrieb]{5a}%Letzte Zeile und Nummer
	{TCP/IP Verbindung abgebrochen}%Bezeichnung
  	{\item Der Fehler wird ins Log geschrieben
  	}
}

\usecase{Nutzer registrieren}{Client, Nutzerdaten DB}%Name und Akteure
{Ein Nutzer soll in der Datenbank registriert}%Ziel
{%Vorbedingungen
  \item Der Server ist in Betrieb.
  \item Der Server hat einen Request über \textit{/register} vom Client
}
{%Nachbedingungen
  \item Nutzer ist in Datenbank eingetragen aber noch nicht als aktiv gekennzeichnet.
  \item Client wurde benachrichtigt, dass der Nutzer aufgenommen wurde.
}
{%Normalablauf
  \item  Der Wert des Schlüssels "`email"' wird in die Nutzerdaten DB eingetragen.
  \item  Der Wert des Schlüssels "`password"' wird in die Nutzerdaten DB eingetragen.
  \item  Der Wert des Schlüssels "`firstname"' wird in die Nutzerdaten DB eingetragen.
  \item  Der Wert des Schlüssels "`lastname"' wird in die Nutzerdaten DB eingetragen.
  \item  Der Wert des Schlüssels "`adress"' wird in die Nutzerdaten DB eingetragen.
  \item  Der Server sendet dem Client die Antwort 200 OK.

}
{%Sondernachbedingungen und Sonderfälle
  \sondernachbedingung{
	\item Nutzer wurde nicht in die Datenbank aufgenommen.
	\item Fehler wurde ins Log geschrieben.
	}

  \kurzersonderfall[5a]{1a}%Letzte Zeile und Nummer
	  {Wert ist leer}%Bezeichnung
  
  \sonderfall[Fehler]{1b}%Letzte Zeile und Nummer
	  {Wert existiert schon (Nutzername)}%Bezeichnung
	  {
	  \item Der Fehler wird ins Log geschrieben.
	  \item Dem Client wird ein entsprechender Fehler übermittelt.
	  } 

  
  \kurzersonderfall[5a]{2a}%Letzte Zeile und Nummer
	  {Wert ist leer}%Bezeichnung
  
  \kurzersonderfall[5a]{3a}%Letzte Zeile und Nummer
	  {Wert ist leer}%Bezeichnung
	  
  \kurzersonderfall[5a]{4a}%Letzte Zeile und Nummer
	  {Wert ist leer}%Bezeichnung
	  
  \sonderfall[Fehler]{5a}%Letzte Zeile und Nummer
	  {Wert ist leer}%Bezeichnung
	  {
	  \item Der Fehler wird ins Log geschrieben.
	  \item Dem Client wird ein entsprechender Fehler übermittelt.
	  }
  \sonderfall[Fehler]{5a.2}%Letzte Zeile und Nummer
	  {Wert ist leer}%Bezeichnung
	  {
	  \item Der Fehler wird ins Log geschrieben.
	  }
	

  \sonderfall[Fehler, weiter mit Normalablauf]{6a}%Letzte Zeile und Nummer
	  {Wert ist leer}%Bezeichnung
	  {
	  \item Der Fehler wird ins Log geschrieben.
	  }
}

\usecase{Nutzer Anmelden}{Client, Nutzerdaten DB}%Name und Akteure
{Nutzer wurde authentifiziert}%Ziel
{%Vorbedingungen
  \item Der Server ist in Betrieb.
  \item Der Server hat einen Request über \textit{/login} vom Client
}
{%Nachbedingungen
  \item Der Client hat ein 200 OK erhalten.
}
{%Normalablauf
  \item Suche nach der übermittelten E-Mailadresse in der Datenbank.
  \item Überprüft ob das übermittelte Passwort zur übermittelten E-Mailadresse passt.
  \item Schickt an Client einen 200 OK.
}
{%Sondernachbedingungen und Sonderfälle
  \sondernachbedingung{
	\item Client hat 403 Access Denied erhalten.
	\item Fehler steht im Log.
	}
	
  \kurzersonderfall[2a.1]{1a}%Letzte Zeile und Nummer
	  {Benutzername nicht in der Datenbank}%Bezeichnung	
	
  \sonderfall[Fehler]{2a}%Letzte Zeile und Nummer
	  {Passwort passt nicht zu Benutzername}%Bezeichnung
	  {
	  \item Der Fehler wird ins Log geschrieben.
	  \item An den Client wird ein 403 Access Denied gesendet.
	  }
}

\usecase{Nutzerdaten abfragen}{Client, Nutzerdaten DB}%Name und Akteure
{Client bekommt Nutzerdaten}%Ziel
{%Vorbedingungen
  \item Der Server ist in Betrieb.
  \item Der Server hat einen Request über \textit{/retrieve} vom Client
}
{%Nachbedingungen
  \item Client hat die Nutzerdaten erhalten.
}
{%Normalablauf
  \item Suche nach der übermittelten E-Mailadresse in der Datenbank.
  \item Überprüft ob das übermittelte Passwort zur übermittelten E-Mailadresse passt.
  \item Nutzerdaten an Client schicken. 
}
{%Sondernachbedingungen und Sonderfälle
  \sondernachbedingung{
	\item Client hat 403 Access Denied erhalten.
	\item Fehler steht im Log.
	}
	
  \kurzersonderfall[2a.1]{1a}%Letzte Zeile und Nummer
	  {Benutzername nicht in der Datenbank}%Bezeichnung	
	
  \sonderfall[Fehler]{2a}%Letzte Zeile und Nummer
	  {Passwort passt nicht zu Benutzername}%Bezeichnung
	  {
	  \item Der Fehler wird ins Log geschrieben.
	  \item An den Client wird ein 403 Access Denied gesendet.
	  }
  \sonderfall[Fehler]{3a}%Letzte Zeile und Nummer
	  {TCP}%Bezeichnung
	  {
	  \item Der Fehler wird ins Log geschrieben.
	  \item An den Client wird ein 403 Access Denied gesendet.
	  } 
}


\usecase{Nutzerdaten verändern}{Client, Abrechnungs DB}%Name und Akteure
{}%Ziel
{%Vorbedingungen
  \item
}
{%Nachbedingungen
  \item
}
{%Normalablauf
  \item
}
{%Sondernachbedingungen und Sonderfälle
  \sondernachbedingung{
	\item
	}
  \sonderfall[Panic]{2a}%Letzte Zeile und Nummer
	  {}%Bezeichnung
	  {
	  \item .
	  }
}


\usecase{Billinginformationen Abfragen}{Client, Abrechnungs DB}%Name und Akteure
{Die bisherigen und ausstehenden Zahlungen für die bisherigen Berechnungen sollen angezeigt werden, Benutzerübersicht}%Ziel
{%Vorbedingungen
  \item Der Server hat einen Get-Request von einem als angemeldeten Benutzer (ungleich Administrator) an /billing erhalten
}
{%Nachbedingungen
  \item 
}
{%Normalablauf
  \item Der Server liest aus der Datenbank alle vom Benutzer berechnet Jobs aus
  \item Der Server liest aus der Datenbank die Zahlungsinformationen zu den Jobs aus (evtl. beides in einem Schritt)
  \item Der Server schickt dem Server eine Liste seiner bisher berechneten Jobs und den fälligen und bisherigen Zahlungen
}
{%Sondernachbedingungen und Sonderfälle
  \sondernachbedingung{
	\item 
	}
  \sonderfall[]{}%Letzte Zeile und Nummer
	  {}%Bezeichnung
	  {
	  \item
	  }
}

%TODO: Protokollspezifikation sagt hier noch gar nichts, also gibt's auch keinen Use Case
%\usecase{Billinginformationen Abfragen}{Client, Abrechnungs DB}%Name und Akteure
%{Die bisherigen und ausstehenden Zahlungen für die bisherigen Berechnungen sollen angezeigt werden, Administratorübersicht}%Ziel
%{%Vorbedingungen
%  \item Der Server hat einen Get-Request von einem als Administrator angemeldeten Benutzer an /billing erhalten
%}
%{%Nachbedingungen
%  \item
%}
%{%Normalablauf
%  \item
%}
%{%Sondernachbedingungen und Sonderfälle
%  \sondernachbedingung{
%	\item
%	}
%  \sonderfall[Panic]{2a}%Letzte Zeile und Nummer
%	  {}%Bezeichnung
%	  {
%	  \item .
%	  }
%}


\usecase{Nutzerliste Abfragen}{Client, Abrechnungs DB}%Name und Akteure
{}%Ziel
{%Vorbedingungen
  \item
}
{%Nachbedingungen
  \item
}
{%Normalablauf
  \item
}
{%Sondernachbedingungen und Sonderfälle
  \sondernachbedingung{
	\item
	}
  \sonderfall[Panic]{2a}%Letzte Zeile und Nummer
	  {}%Bezeichnung
	  {
	  \item .
	  }
}



\usecase{Benutzer löschen}{Client, Abrechnungs DB}%Name und Akteure
{Das Benutzerkonto eines Kunden soll von allen Servern gelöscht werden}%Ziel
{%Vorbedingungen
  \item Der Server hat einen Get-Request von einem als Administrator angemeldeten Benutzer an /delete erhalten
}
{%Nachbedingungen
  \item Der gelöschte Benutzer ist tatsächlich gelöscht
}
{%Normalablauf
  \item Der Server, der den Request erhält, löscht den Benutzer aus der Benutzer-Datenbank.
  %TODO: rechtliche implikationen, nachvollziehbarkeit?
  %\item Die laufenden Jobs des gelöschten Benutzers werden abgebrochen/doch noch ausgeführt
}
{%Sondernachbedingungen und Sonderfälle
  \sondernachbedingung{
	\item Nichts ist passiert.
	}
  \sonderfall[Weiter mit normalem Betrieb]{1a}%Letzte Zeile und Nummer
	  {zu löschender Benutzer existiert nicht}%Bezeichnung
	  {
	  \item Der Server schickt eine entsprechende Fehlermeldung an den Client, der die Lösch-Anfrage verursacht hat
	  }
}
\clearpage
\appendix
\section{Begriffslexikon}

%\begriff{Begriff}{Bedeutung}{Abgrenzung}{Gültigkeit}{Bezeichnung}{Unklarheiten}{Querverweise}

\end{document}
