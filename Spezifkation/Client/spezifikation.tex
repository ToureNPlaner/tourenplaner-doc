\documentclass[a4paper,10pt,titlepage]{article}

\usepackage[utf8]{inputenc}
\usepackage[ngerman]{babel}
\usepackage{graphicx}
\usepackage{a4wide}
\usepackage{url}
\usepackage{booktabs}
\usepackage{float}

\usepackage[pdfborder={0 0 0}]{hyperref}

\graphicspath{{./grafiken/}}

\author{ToureNPlaner Team}
\title{Spezifikation ToureNPlaner\\ Stupro 2011/2012}

\newcommand{\shortcut}[1]
{\texttt{#1}}

\newcommand{\usecase}[8]
{\subsection{#1}
\begin{tabular}{|p{0.2\textwidth}|p{0.7\textwidth}|}
\hline
  Akteur & #2\\\hline
  Ziel & #3\\\hline
  Vorbedingung & #4\\\hline
  Normalablauf & #7\\\hline
  Nachbedingung & #5\\\hline
  Sonderfall & #8\\\hline
  Nachbedingung im Sonderfall& #6\\\hline
 \end{tabular}
}

\newcommand{\begriff}[7]
{\subsection{#1}
\begin{tabular}{|p{0.2\textwidth}|p{0.7\textwidth}|}
\hline
  Bedeutung & #2\\\hline
  Abgrenzung & #3\\\hline
  Gültigkeit & #4\\\hline
  Bezeichnung & #5\\\hline
  Unklarheiten & #6\\\hline
  Querverweise & #7\\\hline
 \end{tabular}}

\newcommand{\doaction}[1]
{Der Benutzer führt die Aktion „\nameref{#1}“ (\ref{#1}) aus.}

\begin{document}

\pagenumbering{alph}
\maketitle
\pagenumbering{roman}
\setcounter{page}{1}
\tableofcontents
\clearpage
\pagenumbering{arabic}
\setcounter{page}{1}

\section{Einleitung}
\subsection{Zweck der Spezifikation}
Die Spezifikation beschreibt die Anforderungen und Funktionalitäten von Tourenplaner. Sie ist die Grundlage für alle folgenden Dokumente; insbesondere muss sie ständig mit dem Entwurf verglichen und bei Bedarf angepasst werden.

Bei der Erstellung der Spezifikation wurde bewusst auf die Festlegung von Implementierungsdetails verzichtet, um gewisse Freiheiten in der Umsetzung der Anforderungen zu erhalten.

\subsection{Leserkreis}
Der Leserkreis dieses Dokuments besteht aus folgenden Personengruppen:
\begin{itemize}
\item Die Entwickler
\item Die Beteiligten des Spezifikationsreviews
\item Der Kunde
\item Die Betreuer des Studienprojekts
\end{itemize}

\subsection{Einsatzbereich und Ziele}

In diesem Project sollen die Clients für das ToureNPlaner-Projekt realisiert werden.
Diese dienen der einfachen Nutzung der Funktionalitäten, die der ToureNPlaner-Server zur Verfügung steht.
Dazu soll es möglich sein, einfach Strecken und Touren zu definieren und vom Server berechnen zu lassen.

\noindent Das Ziel ist es, zu kommerziellen bzw. marktführenden Lösungen konkurrenzfähige Clients zu erstellen.

\subsection{Fachbegriffe und Abkürzungen}

Alle in dieser Spezifikation verwendeten Fachbegriffe werden im sich im Anhang befindlichen Begriffslexikon erläutert.

\subsection{Aufbau dieses Dokuments}

Neben einer allgemeinen Beschreibung des Systems sollen die Anforderungen an die Funktionen des Systems und die geforderten Qualitäten hinsichtlich der Software dokumentiert werden. 
Im Anschluß hierzu wird auf die Benutzeroberfläche und im nächsten Schritt auf die Anwendungsfälle des Systems näher eingegangen. 
Das Dokument endet schließlich mit einem angehängten Begriffslexikon.

\section{Nichtfunktionale Anforderungen}
\subsection{Mengengerüst}
\label{Mengengeruest}
Die mindest Punktanzahl um einen sinnvollen Weg zu berechnen beträgt 2.
Nach obenhin soll ToureNPlaner mit bis zu 100 Punkten zurecht kommen.\\
Der Client wird immer nur von einem Benutzer gleichzeitig verwendet.

\subsection{Entwurfseinschränkungen}
\subsubsection{Systemumgebung}
Der Webclient wir auf der aktuellen Version von Firefox - Version 5 - und Chromium - Version 13 - lauffähig sein.\\
Für Android setzten wir Version 2.1 oder neuer voraus.

\subsubsection{Layout und Gestaltung}
Die Oberfläche wir beim Webclient mit Hilfe von HTML 5 und CSS 3, die des Androidclient mit XML und den Android Developer Tools umgesetzt.

\subsubsection{Datenhaltung}
\label{datenhaltung}
Im Webclient werden die benötigten Daten in Cookies abgelegt.\\
Der Androidclient benutz hierzu eine Konfigurations-API.

\subsection{Robustheit}
Der Webclient verfügt über keine besondere Robustheit.\\
Auf Android können Daten wiederhergestellt werden.

\subsection{Portabilität}
Der Webclient ist gut auf andere Browser adaptierbar.\\
Android ist wegen des ADTs nur auf Android-Geräten lauffähig.

\subsection{Erweiterbarkeit}
Alle Informationen über Algorithem kommt vom Server, deshalb können einfach neue Algorithem hinzugefügt werden.
Daher werden nur Algorithem unterstützt, welche mit Pfaden auf einem Graphen arbeiten.

\subsection{Distributionsform und Installation}
Zur Installation müssen die Daten für die Webclienten auf den Webserver kopiert werden.\\
Für den Androidclient werden Android Packages (APKs) verfügbar sein.

\subsection{Sprachunterstützung}
Beide Client haben dieselbe Übersetztungsdateien als Grundlange, somit soll Konsitenz gewährleistet werden.
Die Übersetzungen werden mit Hilfe von GNU gettext implementiert.

\clearpage
\section{Funktionale Anforderungen}

\clearpage
\section{Benutzeroberfläche}


\clearpage
\section{Anwendungsfälle (Use-Cases)}
\begin{figure}[H]
  \centering
  %TODO
  %\includegraphics[width=\linewidth]{usecases.png}
  \caption{Usecase-Diagramm}
\end{figure}

%\usecase{Name}{Akteur}{Ziel}{Vorbedingung}{Nachbedingung}{Nachbedingung im Sonderfall}{Normalablauf}{Sonderfall}

\subsection*{Benutzerverwaltungsfunktionen}
\usecase{Benutzer registrieren}
{Benutzer}{Der Benutzer will sich ein Benutzerkonto anlegen.}{Keine}{Das Konto wurde erfolgreich im System angelegt.}{Der Benutzer sieht eine Fehlermeldung.}{
\begin{enumerate}
\item Der Benutzer navigiert zum Registrierungsformular.
\item Der Benutzer gibt seine Daten ein.
\item Der Benutzer bestätigt seine Angaben.
\end{enumerate}
}{Es gab einen Fehler beim Erstellen des Kontos.}

\usecase{Anmelden}
{Benutzer}{Der Benutzer will sich am System anmelden.}{Keine}{Der Benutzer ist am System angemeldet.}{Der Benutzer sieht eine Fehlermeldung.}{
\begin{enumerate}
\item (optional) Der Benutzer wählt den Server aus.
\item Der Benutzer gibt seine Anmeldedaten ein.
\item Der Benutzer bestätigt seine Angaben.
\end{enumerate}
}{Die Anmeldedaten sind falsch.}

\usecase{Benutzerdaten anzeigen}
{Benutzer}{Der Benutzer will seine Daten einsehen.}{Der Benutzer hat sich eingeloggt.}{Der Benutzer sieht seine Daten.}{Der Benutzer sieht eine Fehlermeldung.}{
\begin{enumerate}
\item Der Benutzer navigiert zur Benutzerdatenansicht.
\item Der Benutzer sieht seine Daten.
\end{enumerate}
}{Es gibt einen Fehler.}

\usecase{Benutzerdaten bearbeiten}
{Benutzer}{Der Benutzer will seine Daten bearbeiten.}{Der Benutzer hat sich eingeloggt.}{Die Daten des Benutzers sind geändert.}{Der Benutzer sieht eine Fehlermeldung.}{
\begin{enumerate}
\item Der Benutzer navigiert zur Benutzerdatenansicht.
\item Der Benutzer ändert seine Daten.
\item Der Benutzer bestätigt seine Angaben.
\end{enumerate}
}{Es gibt einen Fehler.}

\usecase{Abrechnung anzeigen}
{Benutzer}{Der Benutzer will seine Abrechnung sehen.}{Der Benutzer hat sich eingeloggt.}{Der Benutzer sieht seine Abrechnung.}{Der Benutzer sieht eine Fehlermeldung.}{
\begin{enumerate}
\item Der Benutzer navigiert zur Abrechnungsansicht.
\item Der Benutzer sieht seine Abrechnung.
\end{enumerate}
}{Es gibt einen Fehler.}

\usecase{Alte Abfrage anzeigen}
{Benutzer}{Der Benutzer will eine alte Route sehen.}{Der Benutzer hat sich eingeloggt.}{Der Benutzer sieht seine alte Route.}{Der Benutzer sieht eine Fehlermeldung.}{
\begin{enumerate}
\item Der Benutzer navigiert zur Abrechnungsansicht.
\item Der Benutzer wählt eine alte Abfrage aus.
\item Der Benutzer sieht seine alte Abfrage.
\end{enumerate}
}{Es gibt einen Fehler.}

\clearpage
\appendix
\section{Anhang}

\subsection{Begriffslexikon}

%\begriff{Begriff}{Bedeutung}{Abgrenzung}{Gültigkeit}{Bezeichnung}{Unklarheiten}{Querverweise}

\clearpage
\subsection{Versionshistorie}

	\subsubsection*{Version 0.1 (03.08.2011)}
	\begin{itemize}
		\item Erste veröffentlichte Version
	\end{itemize}

	\subsubsection*{Version 0.2 (09.08.2011)}
	\begin{itemize}
		\item Nichtfunktionale Anforderungen hinzugefügt
	\end{itemize}

\end{document}
