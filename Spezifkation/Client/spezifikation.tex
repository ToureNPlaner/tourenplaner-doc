\documentclass[a4paper,10pt,titlepage]{article}

\usepackage[utf8]{inputenc}
\usepackage[ngerman]{babel}
\usepackage{graphicx}
\usepackage{a4wide}
\usepackage{url}
\usepackage{booktabs}
\usepackage{float}

\usepackage[pdfborder={0 0 0}]{hyperref}

\graphicspath{{./grafiken/}}

\author{ToureNPlaner Team}
\title{Spezifikation ToureNPlaner\\ Stupro 2011/2012}

\newcommand{\shortcut}[1]
{\texttt{#1}}

\newcommand{\usecase}[7]
{\subsection{#1}
\begin{tabular}{|p{0.2\textwidth}|p{0.7\textwidth}|}
\hline
  Ziel & #2\\\hline
  Vorbedingung & #3\\\hline
  Normalablauf & #6\\\hline
  Nachbedingung & #4\\\hline
  Sonderfall & #7\\\hline
  Nachbedingung im Sonderfall& #5\\\hline
 \end{tabular}
}

\newcommand{\begriff}[7]
{\subsection{#1}
\begin{tabular}{|p{0.2\textwidth}|p{0.7\textwidth}|}
\hline
  Bedeutung & #2\\\hline
  Abgrenzung & #3\\\hline
  Gültigkeit & #4\\\hline
  Bezeichnung & #5\\\hline
  Unklarheiten & #6\\\hline
  Querverweise & #7\\\hline
 \end{tabular}}

\newcommand{\doaction}[1]
{Der Benutzer führt die Aktion „\nameref{#1}“ (\ref{#1}) aus.}

\begin{document}

\pagenumbering{alph}
\maketitle
\pagenumbering{roman}
\setcounter{page}{1}
\tableofcontents
\clearpage
\pagenumbering{arabic}
\setcounter{page}{1}

\section{Einleitung}
\subsection{Zweck der Spezifikation}
Die Spezifikation beschreibt die Anforderungen und Funktionalitäten von Tourenplaner. Sie ist die Grundlage für alle folgenden Dokumente; insbesondere muss sie ständig mit dem Entwurf verglichen und bei Bedarf angepasst werden.

Bei der Erstellung der Spezifikation wurde bewusst auf die Festlegung von Implementierungsdetails verzichtet, um gewisse Freiheiten in der Umsetzung der Anforderungen zu erhalten.

\subsection{Leserkreis}
Der Leserkreis dieses Dokuments besteht aus folgenden Personengruppen:
\begin{itemize}
\item Die Entwickler
\item Die Beteiligten des Spezifikationsreviews
\item Der Kunde
\item Die Betreuer des Studienprojekts
\end{itemize}

\subsection{Einsatzbereich und Ziele}

In diesem Project sollen die Clients für das ToureNPlaner-Projekt realisiert werden.
Diese dienen der einfachen Nutzung der Funktionalitäten, die der ToureNPlaner-Server zur Verfügung steht.
Dazu soll es möglich sein, einfach Strecken und Touren zu definieren und vom Server berechnen zu lassen.

\noindent Das Ziel ist es, zu kommerziellen bzw. marktführenden Lösungen konkurrenzfähige Clients zu erstellen.

\subsection{Fachbegriffe und Abkürzungen}

Alle in dieser Spezifikation verwendeten Fachbegriffe werden im sich im Anhang befindlichen Begriffslexikon erläutert.

\subsection{Aufbau dieses Dokuments}

Neben einer allgemeinen Beschreibung des Systems sollen die Anforderungen an die Funktionen des Systems und die geforderten Qualitäten hinsichtlich der Software dokumentiert werden. 
Im Anschluß hierzu wird auf die Benutzeroberfläche und im nächsten Schritt auf die Anwendungsfälle des Systems näher eingegangen. 
Das Dokument endet schließlich mit einem angehängten Begriffslexikon.

\section{Nichtfunktionale Anforderungen}
\subsection{Mengengerüst}
\label{Mengengeruest}

\subsection{Entwurfseinschränkungen}
\subsubsection{Systemumgebung}

\subsubsection{Layout und Gestaltung}

\subsubsection{Datenhaltung}
\label{datenhaltung}

\subsection{Robustheit}

\subsection{Portabilität}

\subsection{Erweiterbarkeit}

\subsection{Distributionsform und Installation}

\subsection{Sprachunterstützung}

\clearpage
\section{Funktionale Anforderungen}

\clearpage
\section{Benutzeroberfläche}


\clearpage
\section{Anwendungsfälle (Use-Cases)}
\begin{figure}[H]
  \centering
  %TODO
  %\includegraphics[width=\linewidth]{usecases.png}
  \caption{Task}
\end{figure}

%\usecase{Name}{Ziel}{Vorbedingung}{Nachbedingung}{Nachbedingung im Sonderfall}{Normalablauf}{Sonderfall}


\clearpage
\appendix
\section{Anhang}

\subsection{Begriffslexikon}

%\begriff{Begriff}{Bedeutung}{Abgrenzung}{Gültigkeit}{Bezeichnung}{Unklarheiten}{Querverweise}

\clearpage
\subsection{Versionshistorie}

	\subsubsection*{Version 0.1 (03.08.2011)}
	
	\begin{itemize}
		\item Erste veröffentlichte Version
	\end{itemize}

\end{document}
