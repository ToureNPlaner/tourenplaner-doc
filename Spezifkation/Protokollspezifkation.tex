\documentclass[ngerman,titlepage,parskip=true]{scrartcl}
\usepackage[utf8]{inputenc}
\usepackage{array}
\usepackage{babel}
\usepackage{wrapfig}
\usepackage{longtable}
\usepackage[unicode=true,pdfusetitle,bookmarks=true,bookmarksnumbered=false,bookmarksopen=false,breaklinks=false,pdfborder={0 0 0},backref=false,colorlinks=false]{hyperref}
\usepackage{listings}
\usepackage{color}

\lstset{ %
	language=C,                % the language of the code
	basicstyle=\footnotesize,       % the size of the fonts that are used for the code
	numbers=left,                   % where to put the line-numbers
	numberstyle=\footnotesize,      % the size of the fonts that are used for the line-numbers
	stepnumber=2,                   % the step between two line-numbers. If it's 1, each line 
	                                % will be numbered
	numbersep=5pt,                  % how far the line-numbers are from the code
	backgroundcolor=\color{white},  % choose the background color. You must add \usepackage{color}
	showspaces=false,               % show spaces adding particular underscores
	showstringspaces=false,         % underline spaces within strings
	showtabs=false,                 % show tabs within strings adding particular underscores
	frame=single,                   % adds a frame around the code
	tabsize=4,                      % sets default tabsize to 2 spaces
	captionpos=b,                   % sets the caption-position to bottom
	breaklines=true,                % sets automatic line breaking
	breakatwhitespace=false,        % sets if automatic breaks should only happen at whitespace
	title=\lstname,                 % show the filename of files included with \lstinputlisting;
	                                % also try caption instead of title
	escapeinside={\%*}{*)},         % if you want to add a comment within your code
	morekeywords={*,...}            % if you want to add more keywords to the set
}

% adjust table padding
\setlength{\tabcolsep}{8pt} \renewcommand{\arraystretch}{1.5}

\newcommand{\requestURL}[1]{\textit{#1}}

\newcommand{\request}[9]
{\subsection{#1}
\begin{tabular}{|p{0.2\textwidth}|p{0.7\textwidth}|}
\hline
  URL & \requestURL{#2}\\\hline
    Beschreibung & #3\\\hline
  HTTP-Method & #4\\\hline
  Server-Typ & #5\\\hline
  Authentifizierung & #6\\\hline
  SSL & #7\\\hline
  Parameter & #8\\\hline
  Status-Code Sonderfall & #9\\\hline
 \end{tabular}\vspace*{1em}}
%\request{}%Name
{}%URL
{}%Beschreibung
{}%HTTP-Method
{}%Server-Typ
{}%Auth
{}%SSL
{}%Parameter
{}%Status-Code
 
\title{Protokollspezifikation}
\author{ToureNPlaner Team}

\begin{document}

\maketitle

\tableofcontents

\pagebreak

\section{Einleitung}

	\subsection{Zweck}
	
	Diese Spezifikation dient als Grundlage jeglicher Kommunikation zwischen Client und Server.\\
	Sie enthält alle Funktionen, die der Server bereit stellt, und beschreibt welche Daten diese Funktionen benötigen und welche Sie zurückliefern. 
	Die Spezifikation dient den Teammitgliedern als Grundlage und Richtlinie für die Entwicklung sowohl des Clients als auch des Servers.
	
	\subsection{Leserkreis}
	
	Dieses Dokument ist für den folgenden Leserkreis bestimmt:
	
	\begin{itemize}
		\item Das gesamte Projektteam
		\item Den Kunden
		\item Die künftigen Programmierer bzw. Betreuer dieses Projektes
	\end{itemize}
	
\section{Generelles}

	In diesem Kapitel werden einige der Standards bzw. Protokolle aufgelistet, die zur Implementierung verwendet werden.

	\subsection{HTTP 1.1}

	Die Kommunikation zwischen Client und Server baut auf HTTP (Hyper Text Transport Protocol) in der Version 1.1 auf. Alle Abfragen eines nicht anonymen Servers, bis auf die /info Abfrage gehen über HTTPS.
	Zur Authentifizierung des Nutzers wird das HTTP Basic Authentication Verfahren, ein Feature von HTTP 1.1 verwendet.\\
	Dabei werden Benutzername und Passwort (bzw. in unserem Fall Email und Passwort) mit einer Base64-Kodierung als Teil des HTTP-Headers an den Server gesendet.
	
	\subsection{JSON}
	 
	JSON, kurz für JavaScript Object Notation, wird zur Repräsentation der Daten verwendet.
	Prinzipiell besteht JSON aus zwei Datentypen, die beliebig ineinander geschachtelt werden können. \\
	Arrays beginnen mit „[“ und enden mit „]“ und enthalten eine durch Komma getrennte Liste von Werten. \\
	Objekte beginnen mit „\{“ und enden mit „\}“ und enthalten eine durch Komma getrennte Liste von Schlüssel-Wert-Paaren, die jeweils durch einen Doppelpunkt getrennt sind.
	Dies ist dabei sowohl analog zu einer klassischen Hashmap als auch zu Javascript Objekten welche als Hashmaps verwaltet werden.
	
	\subsection{World Geodetic System 1984}
	
	Zur Kodierung der Längen- und Breitengrade einer Position auf der Karte wird das WGS 84 verwendet.
	
	\subsection*{Serverart}
	Der Server kann in zwei verschiedenen Modi laufen: Privat/Öffentlich. Im öffentlichen Modus kann jeder Probleminstanzen bearbeiten lassen. Im privaten ist ein Benutzerkonto nötig, um Probleminstanzen bearbeiten zu lassen.
	
	\subsection{Standardablauf einer Abfrage}
	Bei erfolgreicher Bearbeitung einer Abfrage antwortet der Server mit dem HTTP-Status-Code 200 OK und gibt eine evtl. Antwort im HTTP-Body als JSON Objekt zurück. Bei einem Fehler gibt der Server einen entsprechenden Status-Code $>$= 400 und weitere Informationen zum Fehler im HTTP-Body als JSON Objekt zurück. siehe Fehlerliste %TODO


	\subsection{Benutzerobjekt}
	\label{benutzerobjekt}
	Für jeden Benutzer müssen in einer Datenbanktabelle die Informationen gespeichert werden, die zum Login sowie zur Rechnungserstellung benötigt werden. 
	Das sind:
	
\begin{lstlisting}
{
	"email": "max.mustermann@online.de",
	"password": "1234",
	"firstname": "Max",
	"lastname": "Mustermann",
	"address": "Musterstrasse 10, 12345 Musterstadt",
	"admin": false,
	"active": true
}
\end{lstlisting}	
	Das Passwort sollte nur beim Registrieren oder beim ändern der Benutzerdaten verschickt werden.
	\subsection{Requestlogobjekt}	
	Dient der Speicherung alter Abfragen.
	\label{requestlogobjekt}

\begin{lstlisting}
{
	"timestamp": 1234567890,
	"duration": 500,
	"cost": 123.24,
	"request": { ... },
	"response": { ... }
}
\end{lstlisting}

\clearpage
\section{Server}

\request{Serverinformation}
{/info}
{Dient zur Abfrage der vorhandenen Algorithmen und anderer Eigenschaften.}
{GET}
{Öffentlich und privat}
{Nie}
{Nie}
{Keine}
{200 OK}
\subsubsection{Beispiel}
	
	\begin{lstlisting}
{
    "version": 1.0,
    "servertype": public/private,
    "sslport": 443,
    "algorithms": [
        {
            "version": 2,
            "name": "Shortest Path"
            "urlsuffix": "sp",
            "pointconstraints": [                    
                {
                    "name": "height",
                    "type": "meter",
                    "min": 0.0,
                    "max": 2000.0
                }
            ],
            "constraints": {
                   "minPoints": 2,
                   "sourceIsTarget": true/false 
            }
        }, { ... }
    ],
}
    \end{lstlisting}
    \subsubsection*{Erklärung}
    
    \begin{tabular}{|c|p{14cm}|}
    	\hline
    	\textbf{Feld} & \textbf{Beschreibung} \\ 
    	\hline \hline
    	
    	version & Die Version des Protokolls, die der Server spricht. \\ 
    	\hline
    	
    	servertype & Gibt an, ob der Server im öffentlich oder privaten Modus läuft. \\
    	\hline
    	sslport & Falls angegeben antwortet der Server dort auf SSL-Anfragen.\\
    	\hline
    	
    	algorithms & Ein Array aller Algorithmen, die der Server unterstützt. \\
    	\hline
    \end{tabular}
    
    \subsubsection*{Algorithmen}
    
    \noindent \begin{tabular}{|c|p{11.5cm}|}
    	\hline
    	\textbf{Feld} & \textbf{Beschreibung} \\ 
    	\hline \hline
    	
    	version & Die Version des des Algorithmus. \\ 
    	\hline
    	
    	name & Der Name des Algorithmus. \\
    	\hline
    	
    	urlsuffix & Der Requeststring der Anfrage. \\
    	\hline
    	
    	pointconstraints & Array von Constraints, die für jeden einzelnen Punkt benötigt werden. \\
    	\hline
    	
    	pointconstraints.name & Name des Constraints. \\
    	\hline
    	
    	pointconstraints.type & Der Typ des Constraints, um dem Nutzer entsprechende Anzeigen/Hilfen geben zu können. Koordinatenconstraints werden immer angenommen.\\ 
    	\hline
    	
    	pointconstraints.min & Minimaler Wert des Constraints. \\
    	\hline
    	
    	pointconstraints.max & Maximaler Wert des Constraints. \\
    	\hline
    	
    	constraints & Constraints, die nur einmal pro Route benötigt werden und nicht Punktspezifisch sind. Kann Objekte wie in pointconstraints enthalten. \\
    	\hline
    	
    	constraints.minPoints & Die minimale Anzahl an Punkten, die benötigt wird um eine sinnvolle Antwort vom Algorithmus zu erhalten.\\ 
		\hline
		
		constraints.sourceIsTarget & Gibt an, ob Start und Ziel der gleiche Punkt sind.\\    		\hline    		
   	\end{tabular}
   	\clearpage
\section{Benutzer}

Auf private Servern benötigt ein Benutzer ein Konto und muss sich authentifizieren, um einen Algorithmus auf dem Server laufen lassen zu können.
Deshalb werden dem Benutzer Formulare zum Registrieren, Einloggen sowie Bearbeiten seines Accounts gegeben, sowie eine Auflistung der durchgeführten Requests sowie ihrer Kosten.

\request{Registrierung}%Name
{/registeruser}%URL
{Dient zur Erstellung eines Benutzerkontos.\par Dazu werden E-Mail, Passwort, Vor- und Nachname sowie die Rechnungsadresse übermittelt. 
	Eine Validierung der Eingaben findet sowohl auf dem Server als auch auf dem Client statt.}%Beschreibung
{POST}%HTTP-Method
{Privat}
{Notwendig für direktes Freischalten (Admin)}%Auth
{Immer}%SSL
{Keine}%Parameter
{401 Unauthorized: Admin Authentifizierung fehlgeschlagen \par 403 Forbidden: Benutzer schon vorhanden}%Status-Code
		
		\subsubsection{Request}
Ein \texttt{\$\{BenutzerObjekt\}}, siehe \nameref{benutzerobjekt}.
		
\request{Authentifizierung}%Name
{/authuser}%URL
{Überprüfung von Benutzername und Passwort.}%Beschreibung
{GET}%HTTP-Method
{Privat}%Server-Typ
{Immer}%Auth
{Immer}%SSL
{Keine}%Parameter
{401 Unauthorized: Benutzername/Passwort falsch}%Status-Code		
\clearpage

\request{Benutzerdaten abfragen}%Name
{/getuser}%URL
{Holt die eigenen Benutzerdaten. Falls der Parameter ID angegeben ist, werden die Daten des Benutzers mit dieser ID geliefert (Admin).}%Beschreibung
{GET}%HTTP-Method
{Privat}%Server-Typ
{Immer}%Auth
{Immer}%SSL
{ID}%Parameter
{401 Unauthorized: Benutzername/Passwort falsch. Keine Rechte}%Status-Code
\subsubsection{Request}
Leer.
\subsubsection{Response}
Ein \texttt{\$\{BenutzerObjekt\}}, siehe \nameref{benutzerobjekt}.
\clearpage


\request{Benutzerdaten ändern}%Name
{/updateuser}%URL
{Ändert die eigenen Benutzerdaten. Falls der Parameter ID angegeben ist, werden die Daten des Benutzers mit dieser ID geändert (Admin)}%Beschreibung
{POST}%HTTP-Method
{Privat}%Server-Typ
{Immer}%Auth
{Immer}%SSL
{ID}%Parameter
{401 Unauthorized: Benutzername/Passwort falsch.\par
 403 Forbidden: Keine Admin Rechte}%Status-Code		
\subsubsection{Request}
Ein \texttt{\$\{BenutzerObjekt\}}, siehe \nameref{benutzerobjekt}.
Wenn der Benutzer sein Passwort ändern will setzt er das Passwort Feld im Benutzerobjekt. Wenn er es nicht setzen will darf er das Feld nicht angeben.
\clearpage

\request{Abfragenverlauf abrufen}%Name
{/listrequests}%URL
{Ruft die eigenen Abfragen ab. Falls der Parameter ID angegeben ist, werden die Daten des Benutzers mit dieser ID abgerufen. Limit limitiert die Anzahl der zurückgelieferten Datensätze. Offset gibt den Offset zum ersten Datensatz an. (Admin)}%Beschreibung
{POST}%HTTP-Method
{Privat}%Server-Typ
{Immer}%Auth
{Immer}%SSL
{ID, Limit, Offset}%Parameter
{401 Unauthorized: Benutzername/Passwort falsch.\par
 403 Forbidden: Keine Admin Rechte}%Status-Code		
	
\subsubsection{Response}		
		\begin{lstlisting}
{
	"number": 100
	"requests": [
		REQUESTLOGOBJEKT,
		...
	]
}
		\end{lstlisting}
Siehe \nameref{requestlogobjekt}.
		
	

\section{Administration}

Benutzer, bei denen das Adminflag gesetzt ist, können die anderen Benutzer im System verwalten. 
Dazu gibt es zum einen eine Übersicht über alle Benutzer, sowie Einzelansichten des ausgewählten Benutzers, die prinzipiell wie die entsprechende Funktion für normale Benutzer aufgebaut sind.
Es wird nur noch ein weiterer Parameter (id) übergeben, um den Benutzer auszuwählen.

Zusätzlich gibt es für Administratoren die Möglichkeit Benutzer zu löschen und Benutzer ohne Registrierungsformular anzulegen.

\request{Benutzerliste abrufen}%Name
{/listusers}%URL
{Limit limitiert die Anzahl der zurückgelieferten Datensätze. Offset gibt den Offset zum ersten Datensatz an.}%Beschreibung
{GET}%HTTP-Method
{Privat}%Server-Typ
{Immer}%Auth
{Immer}%SSL
{Limit, Offset}%Parameter
{401 Unauthorized: Benutzername/Passwort falsch. Keine Rechte\par
404 Not Found: Kein Benutzer zur angegebenen ID gefunden}%Status-Code
\subsubsection{Response}		
		\begin{lstlisting}
{
	"number": 100
	"requests": [
		BENUTZEROBJEKT,
		...
	]
}
		\end{lstlisting}
Siehe \nameref{benutzerobjekt}.
\clearpage

\request{Benutzer löschen}%Name
{/listusers}%URL
{Löscht den Benutzer mit der angegeben ID.}%Beschreibung
{GET}%HTTP-Method
{Privat}%Server-Typ
{Immer}%Auth
{Immer}%SSL
{ID}%Parameter
{401 Unauthorized: Benutzername/Passwort falsch.\par
 403 Forbidden: Keine Admin Rechte}%Status-Code
\clearpage
   	
\section{Algorithmen}

Da die benötigten Daten je nach Algorithmus unterschiedlich sind, wird hier nur ein generelles Format vorgegeben. Die korrekten Parameter können der Dokumentation des jeweiligen Algorithmus bzw. den Serverinformationen entnommen werden.

\request{Abfrage starten}%Name
{/alg\texttt{\$\{shortname\}}}%URL
{Startet einen Berechnungsvorgang für Algorithmus \texttt{\$\{shortname\}} mit der mitgegebenen Probleminstanz.}%Beschreibung
{POST}%HTTP-Method
{Privat und öffentlich}%Server-Typ
{Nötig wenn privat}%Auth
{Nötig wenn privat}%SSL
{Keine}%Parameter
{Gemeinsam:\par
400 Bad Request: Die Probleminstanz ist fehlerhaft\par
503 Service Unavailable: Server ausgelastet\par
Privat:\par
 401 Unauthorized: Benutzername/Passwort falsch\par
 403 Forbidden: Keine Admin Rechte}%Status-Code

\subsubsection{Beispiel}
	\begin{lstlisting}
{
    "version": 1,
    "points": [
         { "lt": -9.5123, "ln": 8.12, "k": 100 },
         { ... }
    ],
    "constraints": { "t": 100 }
}    	
	\end{lstlisting}
	
		\subsubsection*{Erklärung}
	
	    \noindent \begin{tabular}{|c|p{12cm}|}
	    	\hline
	    	\textbf{Feld} & \textbf{Beschreibung} \\ 
	    	\hline \hline
	    	
	    	version & Die Protokollversion, die der Client verwendet.\\
	    	\hline
	    	
	    	points & Array aller Punkte. Der erste Punkt ist der Startpunkt, der Letzte falls nötig der Zielpunkt. \\ 
	    	\hline
	    	
	    	points.lt & Latitude: Der Breitengrad des Punktes nach WGS 84. \\ 
	    	\hline
	    	
	    	points.ln & Longitude: Der Längengrad des Punktes nach WGS 84. \\
	    	\hline
	    	
	    	points.k & Ein Beispielconstraint, der für jeden Punkt benötigt wird.\\ 
	    	\hline
	    	
	    	constraints & Array globaler Constraints.\\ 
	    	\hline
	    	
	    	constraints.t & Ein Beispielconstraint, der an keinen Punkt gebunden ist.\\ 
	    	\hline
	    \end{tabular}
    
    \subsubsection{Antwort}	
	\begin{lstlisting}
{
	"points": [
		{"lt": -9.5123, "ln": 8.12 },
		{ ... }
	],
	"misc": {
		"distance": 100,
		"apx": 0.5
	}
}
	\end{lstlisting}
	
		\subsubsection*{Erklärung}
		
		\begin{tabular}{|c|p{12cm}|}
			\hline
			\textbf{Feld} & \textbf{Beschreibung} \\ 
			\hline \hline
			
			points & Ein Array aller Punkte.\\
			\hline
			
	    	points.lt & Latitude: Der Breitengrad des Punktes nach WGS 84. \\ 
	    	\hline
	    	
	    	points.ln & Longitude: Der Längengrad des Punktes nach WGS 84. \\
	    	\hline
	    	
	    	misc & Ein JSON-Objekt mit algorithmusspezifischen Ergebnissen, welche nicht zum eigentlichen Pfad gehören \\
	    	\hline
	    	
	    	misc.distance & Ein Beispielergebnis, in diesem Fall die Distanz zwischen Start und Ziel.\\
	    	\hline
		\end{tabular}

\section{Fehlermeldungen}
\subsection{Allgemeine Form}
Am Beispiel eines falsch formatierten JSON Requests.
Der \textit{details} Parameter ist dabei Optional und dient ausschließlich dem leichteren Debuggen von 
Clients.
	\begin{lstlisting}
{
	"errorid": "EBADJSON",
	"message" : "Failed to parse request",
	"details" : "Expected , found . at position 7"
}
	\end{lstlisting}

\subsection{Übersicht}
\begin{tabular}{|c|p{5cm}|p{5cm}|}
\hline
Error Id	& Message			& Beschreibung \\
\hline
EAUTH		& Wrong username or password	& Authenifizierung fehlgeschlagen \\
\hline
EBADJSON	& There was an error parsing the request & Es trat ein Fehler beim parsen des gelieferten JSON auf \\
\hline
EBUSY		& This server is currently too busy to fullfill the request	& Der Server ist ausgelastet und kann den Request derzeit nicht bearbeiten \\
\hline
EDATABASE	& The server can't contact it's database	& Die Benutzer-DB bzw Abrechnungs-DB ist nicht erreichbar \\
\hline
ELIMIT		& The given limit is invalid			& Das angegebene Limit ist ungültig (z.B. negativ) \\
\hline
ENOID		& The given user id is unknown to this server	& Die Benutzer-ID ist unbekannt \\
\hline
ENOTADMIN	& You are not an admin				& Der Benutzer ist kein Admin \\
\hline
EOFFSET		& The given offset is invalid 			& Das angegebene Offset ist ungültig (z.B. negativ) \\
\hline
EREGISTERED	& This username is already registred	& Der Benutzername ist schon vergeben \\
\hline
EUNKNOWNALG	& An unknown algorithm was requested	& Angeforderter Algorithmus unbekannt \\
\hline
EUNKNOWNURL	& An unknown URL was requested	& Angeforderte URL unbekannt \\
\hline
\end{tabular}


\end{document}
